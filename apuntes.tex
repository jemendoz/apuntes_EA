\documentclass[10pt,a4paper]{article}

% == codificacion y espanol ==

\usepackage[T1]{fontenc}
\usepackage[spanish]{babel}

% == paquetes ==

\usepackage{amsmath}
\usepackage[hidelinks]{hyperref}
\usepackage{amssymb}
\usepackage{tcolorbox}
\usepackage{mathtools}
\usepackage{graphicx}
\usepackage{eso-pic}
\usepackage{tikz}
\usepackage{tikz-cd}
\usepackage{faktor}
\usepackage{cancel}

% == nada de sangrado ==

\setlength{\parindent}{0cm}

% == foto portada ==

\newcommand\BackgroundPic{
	\put(0,0){
		\parbox[b][\paperheight]{\paperwidth}{
			\vfill
			\centering
			\includegraphics[width=\paperwidth,height=\paperheight]{fondo.jpg}
			\vfill
		}
	}
}

% == definiciones de comandos ==

\definecolor{pastelpink}{RGB}{253,181,194}

\newtcolorbox{defin}{colback=cyan!15,colframe=cyan!80,coltitle=black,title=Definición}
\newtcolorbox{teoma}{colback=red!15,colframe=red,coltitle=black,title=Teorema}
\newtcolorbox{corol}{colback=orange!15,colframe=orange!80,coltitle=black,title=Corolario}
\newtcolorbox{propo}{colback=green!15,colframe=green!80!black,coltitle=black,title=Proposición}
\newtcolorbox{lema}{colback=blue!15,colframe=blue!60,coltitle=black,title=Lema}
\newtcolorbox{prdad}{colback=violet!15,colframe=violet!65,coltitle=black,title=Propiedad}

\newcommand{\dem}{\textit{Demostraci\'{o}n: }}
\newcommand{\ej}[2]{\textbf{Ejercicio #1:} \textit{#2}}
\newcommand{\gen}[1]{\langle #1 \rangle}
\newcommand{\normleq}{\trianglelefteq}
\newcommand{\normle}{\triangleleft}
\newcommand{\im}{\mathrm{Im}\,}
\newcommand{\verteq}{\rotatebox{90}{$\,=$}}
\newcommand{\equalto}[2]{\underset{\scriptstyle\overset{\mkern4mu\verteq}{#2}}{#1}}

\renewcommand{\leq}{\leqslant}
\renewcommand{\geq}{\geqslant}

% == metadatos ==

\title{Estructuras algebr\'{a}icas}
\author{Jes\'{u}s Mendoza}
\date{2025}

% == documento ==

\begin{document}
	\begin{titlepage}
		\AddToShipoutPicture*{\BackgroundPic}
		\vspace*{1cm}
	\end{titlepage}

	\clearpage

	\tableofcontents
	\newpage

	\section{Grupos}
	\subsection{Monoides}
	Los monoides van a ser una estructura algebr\'{a}ica \'{u}til para entender los grupos.

	\begin{defin}
		Un monoide es un conjunto no vac\'{i}o, $M$, dotado de una operaci\'{o}n interna binaria, $(\cdot)$. El monoide se representa
		como $(M,\cdot)$, aunque se puede sobreentender como $M$. La operaci\'{o}n sigue los siguientes axiomas:
		\begin{enumerate}
			\item Asociatividad. $a(bc) = (ab)c \quad \forall a,b,c \in M$
			\item Elemento neutro. $\exists e \text{ t.q. } \forall a \in M, \; ae = ea = a$
		\end{enumerate}
	\end{defin}

	Adem\'{a}s, si la operaci\'{o}n es conmutativa, se dice que el monoide es \textit{conmutativo}.

	\begin{prdad}
		Un monoide tiene un solo neutro.
	\end{prdad}

	\dem Sean $e$ y $e'$ elementos neutros del monomio $(M,\star)$.
	\begin{gather*}
		e \star e' = \begin{cases}
			e \; (\text{por ser neutro}) \\
			e' \; (\text{por ser neutro})
		\end{cases} \implies e=e'
	\end{gather*}
	

	\begin{prdad}
		Si un elemento $a$ es invertible ($\exists b \text{ t.q. } ab = e$), el inverso es \'{u}nico.
	\end{prdad}

	\dem Sean $b$ y $c$ inversos distintos de $a$ en el monoide $(M,\star)$. Entonces
	\begin{gather*}
		\begin{cases}
			(b \star a) \star c = e \star c = c \\
			b \star (a \star c) = b \star e = b
		\end{cases}
		\implies b=c
	\end{gather*}

	\begin{prdad}
		Al multiplicar $a_1,\ldots,a_n$ elementos de un monoide, da igual el orden de operar, si se respeta el orden de los factores.
	\end{prdad}

	\dem Por inducci\'{o}n. Consideramos el caso $n=1$ y $n=2$ triviales. El caso $n=3$, por axioma de la operaci\'{o}n, se cumple.
	Con el caso $n-1$ supuesto, vemos el caso $n$: manteniendo el orden de los factores, podemos separar el produto en la siguiente operaci\'{o}n:
	\begin{gather*}
		x = x_1 \cdot x_2 = (a_1 \cdots a_i) \cdot (a_{i+1} \cdots a_n)
	\end{gather*}
	Por hip\'{o}tesis:
	\begin{gather*}
		x_1 = ((a_1 \cdot x_2) \cdots)\cdot a_i \\
		x_2 = ((a_{i+1} \cdot a_{i+2}) \cdots) \cdot a_n
	\end{gather*}
	Si $x_2 = a_n$, entonces
	$$
	x = x_1 \cdot x_2 = (((a_1 \cdot a_2)\cdots)\cdot a_{n-1}) \cdot a_n
	$$
	y hemos acabado. Si $x_2$ tiene m\'{a}s de un factor:
	$$
	x = x_1 \cdot x_2 = \overbrace{((a_1 \cdot a_2) \cdots ) a_i)}^{(1)} \cdot \overbrace{((a_{i+1} \cdot a_{i+2}) \cdots ) a_{n-1})}^{(2)}
	\cdot \overbrace{(a_n)}^{(3)}
	$$
	Podemos operar primero (1) con (2), por hip\'{o}tesis, y luego con (3).
	
	\begin{prdad}
		Sea $(M,\cdot)$ un monoide (que por comoidad, denotamos multiplicativamente), y $e$ es neutro. Entonces
		\begin{enumerate}
			\item $e \cdot e = e \implies e$ es invertible, con inverso $e$
			\item Sea $a$ invertible, con inverso $a^{-1}$. Se cumple que $a \cdot a^{-1} = e = a^{-1} \cdot a$. Es decir, el inverso
				de $a^{-1}$ es $a$.
			\item Sea $a,b \in M$ invertibles.
				\begin{gather*}
				(ab)\cdot(b^{-1}a^{-1}) = a(b \cdot b^{-1}) a^{-1} = a \cdot e \cdot a^{-1} = \\
				= a \cdot a^{-1} = e = \ldots = (b^{-1} a^{-1}) \cdot (ab)
				\end{gather*}
				Es decir, el inverso de $ab$ es $b^{-1}a^{-1}$
		\end{enumerate}
	\end{prdad}

	\subsection{Grupos}
	Los grupos son una estructura muy similar a los monoides, pero con axiomas extra.

	\begin{defin}
		Un grupo es un conjunto no vac\'{i}o, $G$, dotado de una operaci\'{o}n interna binaria. Sigue lo siguientes axiomas:
		\begin{enumerate}
			\item Asociatividad. $a(bc) = (ab)c \quad \forall a,b,c \in G$
			\item Elemento neutro. $\exists e \text{ t.q. } \forall a \in G, \; ae = ea = a$
			\item Inverso. $\forall a \in G, \; \exists b \in G \text{ t.q. } ab=ba=e$
		\end{enumerate}
	\end{defin}

	Adem\'{a}s, si la operaci\'{o}n de un grupo es conmutativa, lo llamamos \textit{grupo conmutativo}, o \textit{grupo abeliano}. La cantidad de elementos
	del grupo se le conoce como \textit{orden}.
	
	\subsection{Homomorfismos e isomorfismos}
	Vamos a ver como se deben comportar las aplicaciones entre grupos.

	\begin{defin}
		Sean los grupos $(G_1, \cdot),(G_2,\star)$. Una aplicaci\'{o}n $f: G_1 \to G_2$ es un \textit{homomorfismo} si $f(ab) = f(a) \star f(b)
		\; \forall a,b \in G_1$. \\

		Si un homomorfismo es biyectivo, decimos que es un \textit{isomorfismo}. \\

		Si un isomorfismo es de un grupo a si mismo, decimos que es un \textit{automorfismo}.

		Si existe un isomorfismo entre dos grupos $f: G_1 \to G_2$, decimos que $G_1$ es \textit{isomorfo} a $G_2$, escribiendo $G_1 \cong G_2$.
	\end{defin}

	\begin{propo}
		\begin{enumerate}
			\item Sea $f:G_1 \to G_2$ isomorfismo. Entonces $f^{-1}:G_2 \to G_1$ es isomorfismo.
			\item Sean $f:G_1 \to G_2$ y $g:G_2 \to G_3$ isomorfismos, con $(G_1, \cdot), (G_2, \star), (G_3, \circ)$. Entonces
				$gf: G_1 \to G_3$ es isomorfismo.
		\end{enumerate}
	\end{propo}

	\dem Para demostrar 1), basta ver que $f^{-1}$ respeta las operaciones, es decir, $f^{-1}(b_1 \star b_2) = f^{-1}(b_1) \cdot f^{-1}(b_2)$.
	Como f es biyectiva, $\exists a_1, a_2 \in g_1 \text{ t.q. } b_1 = f(a_1), b_2 = f(a_2)$.
	$$
	f^{-1}(b_1 \star b_2) = f^{-1}(f(a_1) \star f(a_2)) = f^{-1}(f(a_1 a_2)) = a_1 a_2 = f^{-1}(b_1) \cdot f^{-1}(b_2)
	$$
	Para ver 2), usamos una l\'{o}gica similar:
	\begin{gather*}
		gf(a_1 a_2) = g(f(a_1 a_2)) = g(f(a_1) \star f(a_2)) = g(f(a_1)) \circ g(f(a_2)) = \\
		= gf(a_1) \circ gf(a_2)
	\end{gather*} \\
	
	\textit{Dato curioso. } Mientras que $(\mathbb{R},+) \cong (\mathbb{R}^{+},\cdot)$, no es cierto que
	$(\mathbb{Q},+) \cong (\mathbb{Q}^{+},\cdot)$. \\
	\dem Para lo primero, basta con tomar $f: \mathbb{R} \to \mathbb{R}^{+}, f(x) = e^{x}$. Para lo segundo, supongamos
	que existe $f:\mathbb{Q} \to \mathbb{Q}^{+}$ isomorf\'{i}a. Sean $a,b \in \mathbb{Z}, b > 0, \mathrm{mcd}(a,b) = 1,
	\dfrac{a}{b} = f(1)$. Sea $N$ par, $N > |a|$, $N > b$. Entonces tenemos que
	\begin{gather*}
		\dfrac{a}{b} = f(1) = f(\dfrac{1}{N} + \cdots \dfrac{1}{N}) = f(\dfrac{1}{N})^{N}
	\end{gather*}
	Ahora, sea $c,d \in \mathbb{Z}, d > 0, \mathrm{mcd}(a,b) = 1, \dfrac{a}{b} = f(\dfrac{1}{N})$. Entonces
	\begin{gather*}
		\dfrac{a}{b} = f(\dfrac{1}{N})^{N} = \dfrac{c^{N}}{d^{N}} \implies d^{N}a = c^{N}b \implies \\
		\implies \begin{cases}
			d^{N} | b \overset{N > b}{\implies} d = 1 \\
			c^{N} | a \overset{N > |a|}{\implies} c = \pm 1
		\end{cases}
	\end{gather*}
	Lo que significa que $\forall N$ suficientemente grande, $f(\dfrac{1}{N}) = 1$, lo que la hace no-biyectiva.
	
	\begin{corol}
		El conjunto de los automorfismo (isomorfismos de un grupo al mismo) forman en s\'{i} un grupo.
		\begin{gather*}
			\mathrm{Aut}(g) = \{f:G \to G | f \text{ es un isomorfismo}\}
		\end{gather*}
		La operaci\'{o}n es la composici\'{o}n, y el elemento neutro es la aplicaci\'{o}n identidad.		
	\end{corol}

	\subsection{Subgrupos}

	\begin{defin}
		Dado un grupo $(G,\cdot)$, llamamos \textit{subgrupo} $H$ a un conjunto no vac\'{i}o, subconjunto de $G$, que cumple:
		\begin{enumerate}
			\item $ab \in H, \; \forall a,b \in H$
			\item $a^{-1} \in H, \; \forall a \in H$
		\end{enumerate}
	\end{defin}

	\begin{defin}
		Llamamos \textit{subgrupo de automorfismos internos}, denotado por $\mathrm{inn}(G)$ subgrupo de $Aut(G)$ formado
		por los automorfismos de la forma:
		\begin{align*}
			\mathrm{Ad}_a: G & \to G \\
			b & \mapsto aba^{-1}
		\end{align*}

		A estas aplicaciones, las llamamos \textit{automorfismos internos}.
	\end{defin}

	\dem Vamos a demostrar que los automorfismos internos son homomorfismos.
	\begin{gather*}
		\mathrm{Ad}_a(b_1 b_2) = a(b_1 b_2) a^{-1} = ab_1 a^{-1} a b_2 a^{-1} = \mathrm{Ad}_a(b_1)\mathrm{Ad}_a(b_2)
	\end{gather*}
	
	\dem Vamos a demostrar que los automorfismos internos son biyecciones.
	\begin{gather*}
		\mathrm{Ad}_{a^{-1}} \mathrm{Ad}_a (b) = \mathrm{Ad}_{a^{-1}} (aba^{-1}) = a^{-1} (aba^{-1}) (a^{-1})^{-1} = \\
		= a^{-1} a b a^{-1} a = e b e = b
	\end{gather*}
	Por lo que $\mathrm{Ad}_{a^{-1}} \mathrm{Ad}_{a} = \mathrm{Id}$, lo que significa que el automorfismo interno tiene inversa, y es $\mathrm{Ad}_{a^{-1}}$.

	\begin{prdad}
		Los subgrupos contienen al elemento neutro del grupo.
	\end{prdad}

	\dem Sea $H$ subgrupo de $(G,\cdot)$.
	\begin{gather*}
		H \neq \varnothing \implies \exists h \in H \implies \exists h^{-1} \in H \implies hh^{-1} \in H \implies e \in H
	\end{gather*}

	\begin{defin}
		En un grupo $G$, los conjuntos $\{e\}$ y $G$ son subgrupos de $G$. Al primero se le llama \textit{subgrupo trivial},
		y al otro \textit{subgrupo propio}.
	\end{defin}

	\begin{propo}
		Sea un grupo $G$ y $H$ subconjunto finito no vac\'{i}o de $G$. Se tiene que $H$ es subgrupo de $G$ si y solamente si $ab \in H \; \forall a,b \in H$
	\end{propo}

	\dem Aqui, $(G,\cdot)$ es un grupo, $H \subset G$, y $H \neq \varnothing$, y $H$ es cerrado por productos. Veamos que $H$ es cerrado
	por inversos. Como $H$ es finito, $H = \{h_1,\ldots,h_n\}$. Dado un $h_i \in H$, consideramos
	$$
	h_i h_1,\ldots,h_i h_n
	$$
	que pertenecen todos en $H$, ya que $H$ es cerrado por productos. Comprobamos que no hay repeticiones. Supongamos que
	\begin{gather*}
		h_i h_k = h_i h_k, \; j \neq k \underset{\exists h^{-1}_i \in G}{\implies} h_i^{-1} (h_i h_j) = h_i^{-1} (h_i h_k) \implies \\
		\underset{\text{asociativ.}}{\implies} (h_i^{-1} h_i) h_j = (h_i^{-1} h_i) h_k \implies h_j = h_k
	\end{gather*}
	Lo que es imposible, si $j \neq k$, por lo que en el conjunto $\{h_i h_1,\ldots,h_i h_n\}$ no tiene repetidos, y por lo tanto, hay
	$n$ elementos, como en $H$. Entonces,
	$$
	\exists h_j \in H \text{ t.q. } h_i h_j = h_i \implies \ldots \implies h_j = e
	$$
	Entonces
	$$
	\exists h_k \in H \text{ t.q. } h_i h_k = e
	$$
	ya que el neutro est\'{a} en $H$, pero entonces, tiene que ser uno de los $\{h_i h_1, \ldots, h_i h_n\}$, por lo que tiene que
	ser ese $h_k$ el inverso de $h_i$, y esto se cumple para cualquier $h_i$. Por lo que $H$ es cerrado por inversos.
	
	\begin{defin}
		Podemos representar relaciones de contenidos usando un diagrama de ret\'{i}culos. En \'{a}lgebra, los usamos para
		indicar que subgrupos est\'{a}n contenidos en otros sin tener ning\'{u}n otro subgrupo por medio.
		Como el siguiente ejemplo:

		\begin{tikzcd}
			                           & {\{1,2,3\}}                                 &                            \\
			{\{1,2\}} \arrow[ru]       & {\{1,3\}} \arrow[u]                         & {\{2,3\}} \arrow[lu]       \\
			\{1\} \arrow[u] \arrow[ru] & \{2\} \arrow[lu] \arrow[ru]                 & \{3\} \arrow[lu] \arrow[u] \\
			                           & \varnothing \arrow[lu] \arrow[u] \arrow[ru] &                           
		\end{tikzcd}
	\end{defin}

	\begin{defin}
		Dado un subconjunto no vac\'{i}o de $X$ de un grupo $G$ y $X^{-1} = \{a^{-1} | a \in X\}$, el conjunto
		\begin{gather*}
			\langle X \rangle = \{a_1 \cdots a_n | a_1,\ldots,a_n \in X \cup X^{-1}, n \in \mathbb{N}\}
		\end{gather*}
		es un subgrupo de $G$ al cual llamaremsos subgrupo generado por $X$.
		Cualquier subgrupo de $G$ que contiene a $X$ contiene tambi\'{e}n a $\langle X \rangle$.
	\end{defin}

	\begin{defin}
		El caso m\'{a}s sencillo de subgrupo generado es el de un solo elemento. A estos los llamamos \textit{grupos c\'{i}clicos}.
		\begin{gather*}
			\langle a \rangle \underset{X = \{a\}}{=} \{a_1 \cdots a_n | a_1,\ldots,a_n \in X \cup X^{-1}, n \geq 1\} = \{a^{k} | k \in \mathbb{Z}\}
		\end{gather*}
	\end{defin}

	Como en espacio vectoriales, vamos a poder operar juntando e intersecando subgrupos, pero como en espacio vectoriales,
	vamos a tener que buscar un sustituto para la uni\'{o}n de subgrupos, ya que unirlos no es un grupo.

	\begin{propo}
		Sea el grupo $(G,\cdot)$.
		\begin{enumerate}
			\item La intersecci\'{o}n de cualquier familia arbitraria de subgrupos de $(G,\cdot)$ es un subgrupo
			\item El subgrupo generado por $X \subseteq G$ es la intersecci\'{o}n de todos los subgrupos de $G$ que contienen a $X$.
		\end{enumerate}
	\end{propo}

	\dem Para la primera parte, sea $\{H_i\}_{i \in I}$ una familia de subgrupos de un grupo $G$. Veamos que $H := \bigcap_{i \in I} H_i$ es un subgrupo
	de $G$.	Primero, $H \neq \varnothing$, ya que $e \in H_i \; \forall i \in I$. Adem\'{a}s, $H$ es cerrado por productos. Dados $h, h' \in H$,
	tenemos que
	$$
	h, h' \in H_i \; \forall i \in I \implies hh' \in H_i \; \forall i \in I \implies hh' \in H.
	$$
	De la misma forma, vemos que es cerrado por inversos
	$$
	h \in H \implies h \in H_i \; \forall i \in I \implies h^{-1} \in H_i \; \forall i \in I \implies h^{-1} \in H
	$$	

	\dem Para la segunda parte, veamos que $\langle X \rangle$ coincide con la intersecci\'{o}n de todos los subgrupos de $G$ que contienen a $X$.
	Llamemos a esta intersecci\'{o}n $H'$, que, por 1), es subgrupo de $G$, y $H'$ contiene a $X$. Falta ver que $X \subseteq H'$.
	Es claro $\langle X \rangle$ es subgrupo de $G$, y adem\'{a}s $X \subseteq \langle X \rangle$, por lo que $\langle X \rangle$ es uno
	de los subgrupos que forman $H$ por intersecci\'{o}n. Por ello, tenemos que $H' = \langle X \rangle$. \\

	Vamos a ver un ejemplo curioso de generadores. Sean:
	\begin{gather*}
		a = \begin{pmatrix}
			0 & 1 \\
			1 & 0
		\end{pmatrix} \quad b = \begin{pmatrix}
			0 & 2 \\
			\dfrac{1}{2} & 0
		\end{pmatrix}
	\end{gather*}

	Vemos que $a^2 = \mathrm{Id}$, $a^{3} = a$, $a^{4} = \mathrm{Id},\ldots$. Si lo hacemos con $b$, tenemos $b^2 = \mathrm{Id}$. Parecer\'{i}a
	que el generador $\langle ab \rangle$ no es infinito, pero si nos ponemos a echar cuentas, vemos que:
	\begin{gather*}
		(ab)^{n} = \begin{pmatrix}
			\dfrac{1}{2^{n}} & 0 \\
			0 & 2^{n}
		\end{pmatrix}
	\end{gather*}
	por lo que el generador es infinito.

	\begin{propo}
		Sea $\{H_i | i \in \Lambda\}$ una familia de subgrupos de un grupo $(G, \cdot)$.
		\begin{enumerate}
			\item La intersecci\'{o}n $\bigcap_{i \in \Lambda} H_i$ est\'{a} contenida en todos los subgrupos $H_i$ y
				contiene a cualquier subgrupo contenido en todos ellos.
			\item El subgrupo $\langle \bigcup_{i \in \Lambda} H_i\rangle$ contiene a todos los subgrupos $H_i$ y
				est\'{a} contenido en cualquier subgrupo que los contenga.
		\end{enumerate}
	\end{propo}

	\dem Para la parte 2). Sea $H := \langle \bigcup_{i \in \Lambda} H_i \rangle$. Veamos que $H_i \subseteq H \; \forall i \in \Lambda$. Sabemos
	que $H$ es el subgrupo m\'{a}s peque\~{n}o que contiene $\bigcup_{i \in \Lambda} H_i$, por lo que contiene
	a cada $H_i$. Cualquier otro subgrupo que contenga a cada $H_i$ contiene a $\bigcup_{i \in \Lambda} H_i$, por lo que
	contiene a $\langle \bigcup_{i \in \Lambda} H_i \rangle = H$

	\begin{propo}
		Sea $H \leq G$ (subgrupo de $G$) y $a \in G$. El conjunto $aHa^{-1}$ = $\{aha^{-1} | h \in H\}$ tambi\'{e}n es
		subgrupo de G, al cual llamamos subgrupo \textit{conjugado de $H$ por $a$}. Adem\'{a}s, tiene el mismo
		orden que $H$.
	\end{propo}

	\dem Claramente $aHa^{-1} \neq \varnothing$ por serlo $H$. Dados $aha^{-1}$ y $ah'a^{-1} \in aHa^{-1}$
	\begin{gather*}
		(aha^{-1})(ah'a^{-1}) = ahh'a^{-1} \in aHa^{-1}
	\end{gather*}
	por lo que es cerrado por productos. Adem\'{a}s
	\begin{gather*}
		(aha^{-1})^{-1} = (a^{-1})^{-1} h^{-1} a^{-1} = ah^{-1}a^{-1} \in aHa^{-1}
	\end{gather*}
	por lo que es cerrado por inversos, y por lo tanto, grupo.

	Como $\mathrm{Ad}_a : G \to G$ es biyectiva, y $aHa^{-1} = \mathrm{Ad}_a(H)$, entonces $|aHa^{-1}| = |H|$.

	\begin{teoma}
		El orden de cualquier subgrupo de un grupo finito divide el orden del grupo.
	\end{teoma}

	\dem Aqu\'{i} $G$ es un grupo finito y $H$ es un subgrupo de $G$. DEclaramos que para $x,y \in G$,
	\begin{gather*}
		x \equiv y \iff \exists h \in H \text{ t.q. } y = xh
	\end{gather*}
	
	Vamos a ver alguna propiedades
	\begin{itemize}
		\item Reflexiva: cogiendo el elemento neutro, $x = xe \; \forall x \in G$
		\item Sim\'{e}trica: cogiendo el inverso, $y = xh \implies x = yh^{-1}$
		\item Transitiva: si $y = xh, \; z = yh' \implies z = x(hh')$
	\end{itemize}
	Por lo que la relaci\'{o}n es de equivalencia. Adem\'{a}s, la clase de equivalencia tiene tantos
	elementos como $H, \; \forall x \in H$. Si $[x_1],\ldots,[x_k]$ son las distintas clases de equivalencia,
	entonces sabemos que
	\begin{gather*}
		G = [x_1] \sqcup \ldots \sqcup [x_k] \\
		|G| = |[x_1]| \sqcup \ldots \sqcup |[x_k]| = \\
		= |H| + \ldots + |H| = k |H|
	\end{gather*}
	donde $k$ es el n\'{u}mero distinto de clases de equivalencia que hay. Por ello,
	$|H|$ divide a $|G|$

	\begin{defin}
		Sea $H$ un subgrupo de un grupo $G$. Los conjuntos $aH = \{ah | h \in H\}$, con $a \in G$ se les llama \textit{clases
		laterales a la izquierda} de $H$ en $G$. En un grupo finito $G$ hay $|G : H| := |H|/|G|$ de estas
		clases. A este n\'{u}mero se le llama \textit{\'{i}ndice de $H$ en $G$}.
	\end{defin}

	\begin{corol}
		Si dos subgrupos de un grupo finito tienen ordenes coprimos, su intersecci\'{o}n es el subgrupo trivial.
	\end{corol}

	\dem Aqu\'{i} $G$ es grupo finito, $H_1,H_2$ son subgrupos de $G$ con ordenes coprimos entre s\'{i}. Veamos que $H_1 \cap H_2 = \{e\}$.
	Sabemos que $H_1 \cap H_2$ es subgrupo, que adem\'{a}s es subgrupo tanto de $H_1$ como de $H_2$. Por el teorema de Lagrange,
	su tama\~{n}o divide al tama\~{n}o de $H_1$, y tambi\'{e}n al de $H_2$, pero como estos son coprimos, el \'{u}nico factor que
	tienen en com\'{u}n es el 1.

	\subsection{Ejemplos de grupos}

	\paragraph{Grupo multiplicativo de los complejos}
	Denotado $(\mathbb{C} \setminus \{0\},\cdot)$, es un grupo del que podemos destacar los subgrupos de raices de la unidad tal que:
	\begin{gather*}
		\{w,w^2,\ldots,w^{k} = 1\}
	\end{gather*}

	\paragraph{Grupos de permutaciones}
	En un conjunto de $n$ elementos, las aplicaciones biyectivas (permutaci\'{o}n), junto con la composici\'{o}n son un grupo. Por ejemplo:
	\begin{gather*}
		X = \{1,2,3,4\} \\
		f \in S_X \text{ t.q. } \begin{cases}
			1 \to 2 \\
			2 \to 4 \\
			3 \to 3 \\
			4 \to 1
		\end{cases}
	\end{gather*}
	Este grupo es c\'{i}clico, por lo que para toda aplicaci\'{o}n, existe una potencia que acabe siendo el neutro.

	\paragraph{Grupo general lineal}
	Dado un cuerpo $F$, el conjunto
	$$
	GL_n(F) = \{A \in M_n(F) | \mathrm{det}(A) \neq 0\}
	$$
	con el producto de matrices es un grupo.

	\paragraph{Grupo cuaternio}
	Es un grupo, $Q_8$, es isomorfo a las siguiente matrices:
	$$
	\pm \begin{pmatrix}
		1 & 0 \\ 0 & 1
	\end{pmatrix}
	, \quad \pm \begin{pmatrix}
		i & 0 \\ 0 & -i
	\end{pmatrix}
	, \quad \pm \begin{pmatrix}
		0 & 1 \\ -1 & 0
	\end{pmatrix}
	, \quad \pm \begin{pmatrix}
		0 & i \\ i & 0
	\end{pmatrix}
	$$
	llamadas: $\pm 1, \pm i, \pm j, \pm k$, respectivamente. Tienen la curiosa propiedad: $i^2 = j^2 = k^2 = -1$. El grupo
	no es abeliano, y puedes usar esta tabla para multiplicar elementos: \\
	\begin{tikzcd}
	             & i \arrow[rd] &              \\
	k \arrow[ru] &              & j \arrow[ll]
	\end{tikzcd}

	\paragraph{Grupos de simetr\'{i}as}
	Dado un conjunto $X$ espacio af\'{i}n eucl\'{i}deo $E$, el conjunto
	\begin{gather*}
		\mathrm{Sim}(X) = \{f:E \to E | f \text{ es aplicaci\'{o}n af\'{i}n biyectivas, } \\
		d(f(x),f(y)) = d(x,y) \forall x,y \in E, \; f(X) = X\}
	\end{gather*}

	El grupo di\'{e}drico, $D_n$, de orden $2n$ es el grupo de simetr\'{i}as del pol\'{i}gono regular de $n$ caras. Podemos considerar
	estos pol\'{i}gonos como los inscritos en la circunferencia de radio 1, y con el v\'{e}rtice principal en (1,0).
	El grupo est\'{a} formado por el conjunto
	\begin{gather*}
		D_n = \{\rho, \rho^2, \ldots, \rho^{n} = \mathrm{Id}, \sigma, \sigma \rho, \ldots, \sigma \rho^{n-1}\}
	\end{gather*}
	donde $\rho^{k}$ es la rotaci\'{o}n positiva de $2\pi/k$ radianes, y $\sigma$ es la reflexi\'{o}n respecto al eje que pasa
	por el v\'{e}rtice principal y el centro de la figura. El operador el la composici\'{o}n de simetr\'{i}as.

	\textit{Nota: } dada $f$ simetr\'{i}a del pol\'{i}gono de $n$ v\'{e}rtices, podemos elegir $i=0,1,\ldots$ de modo
	que $\sigma^{i} f$ sea rotaci\'{o}n que fija al pol\'{i}gono, por lo que $\sigma^{i} f = \rho^{j}$ para alg\'{u}n $j$. Por ello,
	$f = \sigma ^{-i} \rho^{j}$, por lo que $D_n$ son todas las simetr\'{i}as.

	Observar tambi\'{e}n
	\begin{gather*}
		\underbrace{\text{rotaci\'{o}n}}{\sigma \rho \sigma^{-1}} = \rho^{n-1} = \rho^{-1}
	\end{gather*}
	y, en general, para echar cuentas est\'{a} bien saber que
	\begin{gather*}
		\begin{cases}
			\rho^{n} = \mathrm{Id}, \quad \sigma^2 = \mathrm{Id} \\
			\sigma \rho \sigma^{-1} = \rho^{-1}
		\end{cases}
	\end{gather*}


	\subsection{Grupos c\'{i}clicos}

	\begin{defin}
		Sea el grupo $G$, $a \in G$. El \textit{orden} de $a$ es, en caso de existir, el m\'{i}nimo $n \in \mathbb{N}$ tal que
		$a^{n} = e$. Si tal m\'{i}nimo no existe, se dice que es de orden infinito. Se denota como $|a| = n$.
	\end{defin}

	\begin{propo}
		En un grupo finito, ning\'{u}n elemento puede tener orden infinito.
	\end{propo}

	\begin{propo}
		Sea $G$ grupo finito, y $a \in G$. Se tiene que $a^{n} = e \iff |a|$ divide a $n$. En particular, $a^{n} = a^{m} \iff n \equiv m 
		\; (\mathrm{mod} \; n)$.
	\end{propo}

	\dem Aqu\'{i}, $G$ es grupo y $a \in G$ es un elemento de orden finito. Si $a^{n} = e$, dividimos $n$ entre $|a|$
	$$
	n = k \cdot |a| + r, \quad \text{con} \; 0 \leq r < |a| 
	$$
	Luego
	\begin{gather*}
		e = a^{n} = a^{k\cdot |a| +r} = (a^{|a|})^{k} \cdot a^{r} = e^{k} \cdot a^{r} = a^{r}
	\end{gather*}
	Por ello, $a^{r} = e$ y $0 \leq r \leq |a|$, por minimalidad de $|a|$, $r=0$, y $n = k|a|$, y llegamos a que $|a|$ divide a $n$.
	Claramente tambi\'{e}n, si $|a|$ divide a $n$ entonces $a^{n} = e$. En particular
	\begin{gather*}
		a^{n} = a^{m} \iff a^{n-m} = e \iff |a| \; | \; n-m \iff n \equiv m \; (\mathrm{mod} \; n)
	\end{gather*}

	\begin{propo}
		Sea $G$ grupo y $a \in G$ un elemento con orden $|a| = n$. Entonces $a^{-1} = a^{n-1}$.
	\end{propo}
	
	\dem
	\begin{gather*}
		|a| = n \iff a \underbrace{\cdots}_{n \text{ veces}} a = a \cdot a^{n-1} = e \implies a^{-1} = a^{n-1}
	\end{gather*}

	\begin{prdad}
		Si $\nexists a \in G$ tal que $|a| = |G|$, entonces $G$ no es un grupo c\'{i}clico.
	\end{prdad}
	
	\begin{teoma}
		El orden de un elemento coincide con el orden del subgrupo que genera. Notar que entonces los elementos $e, a, a^2, \ldots, a^{|a|-1}$
		son distintos entre s\'{i}.
	\end{teoma}

	\dem Si $|a| = n$ finito, entonces
	$$
	\langle a \rangle = \{a, a^2, \ldots, a^{|a|} = e\}
	$$
	tiene a lo sumo $|a|$ elementos. Si existen $i,j \in \mathbb{N}$ distintos tal que $a^{i} = a^{j}$, con $1 \leq i < j \leq |a|$, entonces $a^{j-i} = e$,
	con $1 \leq j-i \leq |a|-1$. Esto es imposible, por la minimalidad de $a^{|a|} = e$. Por lo que $\langle a \rangle$ tiene
	$|a|$ elementos. Si $|a| = \infty$, entonces $\langle a \rangle = \{a^{n} | n \in \mathbb{Z}\}$ tiene orden $\infty$.

	\begin{corol}
		En un grupo $G$, el orden de cualquiera de sus elementos tiene que dividir al orden del grupo. Adem\'{a}s, $a^{|G|} = e \; \forall a \in G$.
	\end{corol}

	\dem Sea $G$ grupo finito, y $a \in G$. Sabemos que $|a| = |\langle a \rangle|$ que por el teorema de Lagrange, divide a $|G|$. Por ello
	$$
	|a| \; | \; |G|
	$$
	y por tanto, $a^{|G|} = e$

	\begin{corol}
		\begin{enumerate}
			\item Teorema de Euler-Fermat. Sea $n \geq 2, a \in \mathbb{Z}$, coprimos entre s\'{i}. Se tiene que $a^{\varphi(n)} \equiv 1
			\; (\mathrm{mod} \; n)$
			\item Peque\~{n}o teorema de Fermat. Sea $p$ primo que no divide a $a \in \mathbb{Z}$. Entonces $a^{p-1} \equiv 1 \; (\mathrm{mod} \; p)$
		\end{enumerate}
	\end{corol}

	\dem Para $U(n) = \{\overline{a} \in \mathbb{Z}_n | \; \overline{a} \text{ es invertible}\}$. Sabemos que $|U(n)| = \varphi(n)$. Por el resultado
	anterior
	$$
	\overline{a}^{|U(n)|} = \overline{1}
	$$
	Por lo que
	$$
	a^{\varphi(n)} \equiv 1 \; (\text{mod }n)
	$$
	si $\mathrm{mcd(a,n) = 1}$

	\begin{corol}
		Si el orden de un grupo es un n\'{u}mero primo, entonces es un grupo c\'{i}clico.
	\end{corol}

	\dem Aqu\'{i} $|G| = p$ primo. Sea $a \in G$, $a \neq e$. Como $|\langle a \rangle| \; | \; |G| = p$, por el teorema de Lagrange,
	$|\langle a \rangle| = 1$ \'{o} $p$. Como $e,a \in \langle a \rangle$, $|\langle a \rangle| \geq 2$, por lo que $|\langle a \rangle| = p = |G|$, y
	llegamos a que $G = \langle a \rangle$, y por tanto, $G$ es c\'{i}clico.

	\begin{teoma}
		Sea $a$ un elemento de orden finito de un grupo, y $k \in \mathbb{N}$. Se tiene que $|a^{k}| = \dfrac{|a|}{\mathrm{mcd}(|a|,k)}$
	\end{teoma}

	\dem
	\begin{gather*}
		(a^{k})^{n} = e \iff a^{kn} = e \iff |a| \; | \; kn \iff \dfrac{|a|}{\mathrm{mcd}(|a|,k)} \; | \; \dfrac{k}{\mathrm{mcd}(|a|,k)} \cdot n \iff \\
		\iff \dfrac{|a|}{\mathrm{mcd}(|a|,k)} \; | \; n
	\end{gather*}
	por lo que $|a^{k}| = \dfrac{|a|}{\mathrm{mcd}(|a|,k)}$

	\textit{Ejemplo} $(\mathbb{Z}_5, +)$ \\
	$\langle \overline{1} \rangle = \{\overline{1},\overline{2},\overline{3},\overline{4},\overline{5} = \overline{0}\}$, por lo que ese grupo
	es c\'{i}clico generado por $\overline{1}$. Del mismo modo, $\langle \overline{2} \rangle = \mathbb{Z}_5$. \\

	Sin embargo, si probamos con $(\mathbb{Z}_6, +)$, vemos que solo el $\langle \overline{1} \rangle$ y $\langle \overline{5} \rangle$, ya que
	son los \'{u}nicos coprimos con 6. Vamos a verlo

	\begin{corol}
		Sea $G$ un grupo c\'{i}clico de orden $n$, y $a$ un generador de $G$. El conjunto de generadores de $G$ es $\{a^{k} | \mathrm{mcd}(k,n)
		= 1, 1 \leq k \leq n\}$, por lo que $G$ tiene $\varphi(n)$ generadores.
	\end{corol}

	\dem Aqu\'{i} $G$ es c\'{i}clico de orden $n$, por lo que $\exists a \in G \text{ t.q. } G = \langle a \rangle$ y $|a| = n$. Un elemento
	$a^{k}$ genera $G \iff \langle a^{k} \rangle = G \iff \dfrac{n}{\mathrm{mcd}(n,k)} = n \iff \mathrm{mcd}(n,k) = 1$. Es decir,
	hay tantos generadores de $G$ como coprimos hay de $n$, menores que $n$, que es $\varphi(n)$.

	\begin{propo}
		Sean $a,b$ elementos de orden finito de un grupo. Si son coprimos entre s\'{i} y conmutan, entonces $|ab| = |a||b|$.
	\end{propo}

	\dem Aqu\'{i} $G$ es grupo y $a,b \in G$ tienen orden finito, $ab = ba$ y $\mathrm{mcd}(|a|,|b|) = 1$.
	\begin{gather*}
		(ab)^{n} = e \implies a^{n}b^{n} = e \implies a^{n} = b^{-n} \\
	\end{gather*}
	Por ello,
	\begin{gather*}
		(a^{n})^{|b|} = \begin{cases}
			a^{n|b|} \\
			(b^{-n})^{|b|} = b^{-|b|n} = e
		\end{cases} \implies
		|a| \; | \; n|b| \underset{\mathrm{mcd}(|a|,|b|=1)}{\implies} |a| \; | \; n
	\end{gather*}
	Haciendo lo mismo con $b$, sacamos que $|b| \; | \; n$, y por ello, $n$ es m\'{u}ltiplo de $|a||b|$. Como $(ab)^{|a||b|} = a^{|a||b|} b^{|a||b|} = e$,
	entonces $|a||b|$ es el menor $n \geq 1$ tal que $(ab)^{n} = e$, por lo que $|ab| = |a||b|$.

	\begin{lema}
		Todo subgrupo de un grupo c\'{i}clico es c\'{i}clico
	\end{lema}
	
	\dem Aqu\'{i} $G$ es un grupo c\'{i}clico, por lo que $G = \langle a \rangle$ para alg\'{u}n $a \in G$, y $H$ es un subgrupo de $G$. Vamos a ver que $H$
	es c\'{i}clico. Si $H = \{e\}$, entonces es c\'{i}clico. Si $H \neq \{e\}$, consideramos $a^{k} \in H$ con $k$ m\'{i}nimo, $k \geq 1$.
	Dado $b \in H$, $b = a^{j}$ para alg\'{u}n $j \in \mathbb{Z}$. Dividimos $j$ entre $k$.
	$$
	j = qk + r, \text{ con } 0 \leq r < k
	$$
	por lo que
	\begin{gather*}
		b = a^{j} = a^{qk+r} = (a^{k})^{q} a^{r} \therefore a^{r} = (a^{k})^{-q} \cdot b \in H
	\end{gather*}
	Por minimalidad de $k$, necesitamos que $r=0$, lo que hace que
	\begin{gather*}
		b = a^{j} = (a^{k})^{q} \in \langle a^{k} \rangle \therefore H \subseteq \langle a^{k} \rangle \subseteq H \\
		\therefore H = \langle a^{k} \rangle
	\end{gather*}
	Por lo que $H$ es c\'{i}clico.
	
	\begin{teoma}
		Sea un grupo c\'{i}clico $G$ de orden $n$ y $a$ un generador suyo. Para cada divisor $d$ de $n$, existe un \'{u}nico subgrupo de $G$
		de orden $d$, que es $\langle a^{n/d} \rangle$. Adem\'{a}s, estos son los \'{u}nicos subgrupos de $G$.
	\end{teoma}

	\dem Aqu\'{i} $G = \langle a \rangle$ con $|G| = n$, y $d$ es divisor de $n$. Veamos que $|\langle a^{n/d} \rangle| = d$.
	\begin{gather*}
		|\langle a^{n/d} \rangle| = |a^{n/d}| = \dfrac{|a|}{\mathrm{mcd}(|a|,n/d)} = \dfrac{n}{\mathrm{mcd}(n,n/d)} = \dfrac{n}{n/d} = d
	\end{gather*}

	Veamos que $\langle a^{n/d} \rangle$ es el \'{u}nico subgrupo de $G$ con $d$ elementos. Sea $H \leq G$, $|H| = d$. Sabemos que $H$
	debe ser c\'{i}clico $\therefore H = \langle a^{m} \rangle$ para alg\'{u}n $m$. Tenemos
	\begin{gather*}
		d = |H| = |a^{m}| = \dfrac{|a|}{\mathrm{mcd}(n,m)} = \dfrac{n}{\mathrm{mcd}(n,m)} \therefore \\
		\mathrm{mcd}(n,m) = \dfrac{n}{d} \therefore \dfrac{n}{d} | m \therefore \exists k \text{ t.q. } m=\dfrac{n}{d}k \therefore \\
		a^{m}=a^{\frac{n}{d} \cdot k} = (a^{\frac{n}{d}})^{k} \therefore a^{m} \in \langle a^{\frac{n}{d}} \rangle \therefore \\
		H \subseteq \langle a^{\frac{n}{d}} \rangle
	\end{gather*}
	Como $|H| = d = |\langle a^{\frac{n}{d}} \rangle|$, entonces $H = \langle a^{\frac{n}{d}} \rangle$.

	\textit{Ejemplo:} $(\mathbb{Z}_{12},+)$ \\

	$\mathbb{Z}_{12}$ es grupo c\'{i}clico, generado por $\langle \overline{1} \rangle$. Los divisores de 12 son 1,2,3,4,6,12.
	\begin{itemize}
		\item Orden 1: $\langle \overline{0} \rangle$
		\item Orden 2: $\langle \overline{6} \rangle$
		\item Orden 3: $\langle \overline{4} \rangle$
		\item Orden 4: $\langle \overline{3} \rangle$
		\item Orden 6: $\langle \overline{2} \rangle$
		\item Orden 12: $\langle \overline{1} \rangle$
	\end{itemize}

	\begin{teoma}
		Sea $F$ un cuerpo y $G \subseteq F \setminus \{0\}$ un subgrupo multiplicativo finito. $G$ es un grupo c\'{i}clico.
	\end{teoma}

	\dem $F$ es cuerpo, y $G \leq (F \setminus \{0\}, \cdot)$
	Veamos que $G$ es c\'{i}clico. Observamos que:
	\begin{enumerate}
		\item $G$ es abeliano, por serlo $(F \setminus \{0\},\cdot)$
		\item Sobre un cuerpo, un polinomio tiene a lo sumo tantas ra\'{i}ces como su grado.
	\end{enumerate}

	Como $G$ es finito, existe $a \in G$ con $|a|$ m\'{a}ximo. Veamos que $\forall b \in G, |b|$ divide a $|a|$ (\textit{no demostrado del todo}).
	As\'{i} $\forall b \in G, b^{|a|} = 1 \therefore$ todo elemento de $G$ es un elemento de $F$ ra\'{i}z de
	$$
	x^{|a|} - 1
	$$
	Como en $F$ el polinomio $x^{|a|} - 1$ tiene a lo sumo $|a|$ ra\'{i}ces entonces
	\begin{gather*}
		|G| \leq |a| = |\langle a \rangle| \leq |G| \therefore |G| = |\langle a \rangle| \therefore G = \langle a \rangle
	\end{gather*}

	\textit{Ejemplo: } Calculamos los subgrupos de $D_n$. \\

	Sea $S \leq D_n = \{\underbrace{R,R^2,\ldots,R^{n}=I}_{\text{rotaciones}},\underbrace{H,HR,\ldots,HR^{n-1}}_{\text{reflexiones}}\}$.
	Pueden ocurrir dos cosas:
	\begin{enumerate}
		\item $S$ no contiene reflexiones $\therefore S \leq \langle R \rangle \therefore S = \langle R^{d} \rangle$ donde $d$ es un
			divisor de $n$ (y tiene $\frac{n}{d} elementos)$.
		\item $S$ contiene alguna reflexi\'{o}n $\sigma$. Sean
			\begin{gather*}
				S_+ := S \cap \{R,R^2,\ldots,R^{n}\} = S \cap \langle R \rangle \leq \langle R \rangle \\
				S_- := S \cap \{H,HR,\ldots,HR^{n-1}\} \\
				\therefore S = S_+ \sqcup S_-
			\end{gather*}
	\end{enumerate}
	Veamos en este segundo caso que $S_- = \sigma S_+$
	\begin{gather*}
		\underbrace{\sigma}_{\in S} \underbrace{S_+}_{\subseteq S} \underbrace{\subseteq}_{S \text{ subg.}} S_- \\
		\sigma S_- \subseteq S_+ \implies S_- \subseteq \sigma S_+ \\
		\therefore S_- \subseteq \sigma S_+ \subset S_- \\
		\therefore S_- = \sigma S_+
	\end{gather*}

	Como $S_+ \leq \langle R \rangle$ entonces $S_+ = \langle R^{d} \rangle$ para alg\'{u}n $d$ divisor de $n$.
	$$
	S = \langle R^{d} \rangle \sqcup \sigma \langle R^{d} \rangle
	$$
	Como $\sigma$ es reflexi\'{o}n $\sigma = HR^{i}$ para alg\'{u}n $i \in \{0,1,\ldots,n-1\}$. Multiplicando $\sigma$ a derecha
	por $R^{-d} \in S_+$ las veces precisas encontramos $\sigma_0 \in S_-$ con $\sigma_0 = HR^{i}$, con $i \in \{0,1,\ldots,d-1\}$.
	Por tanto $S_{d,i}$ con $0 \leq i \leq d$, $S_{d,i} := S = \langle R^{d} \rangle \sqcup \sigma_0 \langle R^{d} \rangle$.
	Y $\sigma_0 = HR^{i}$ con $i \in \{0,\ldots,n-1\}$. Veamos que estos subgrupos son distintos entre s\'{i}, y
	son todos los de $D_n$.

	\textbf{No acabado en clase}

	\subsection{Grupos sim\'{e}tricos}

	\begin{defin}
		Dado un conjunto $X$, una \textit{permutaci\'{o}n} de $X$ es una biyecci\'{o}n de $X$ en $X$. El \textit{grupo de permutaciones}
		de $X$ es el conjunto $S_X = \{\alpha:X \to X | \alpha \text{ es biyecci\'{o}n}\}$, con la composici\'{o}n. El \textit{grupo
		sim\'{e}trico de grado $n$} es el grupo de permutaciones de $X_n = \{1,\ldots,n\}$ y se denota $S_n$.
	\end{defin}

	\textit{Ejemplo: } Con $S_5$, podemos considerar la permutaci\'{o}n:
	\begin{align*}
		f: \{1,2,3,4,5\} & \to \{1,2,3,4,5\} \\
		1 & \mapsto 4 \\
		2 & \mapsto 3 \\
		3 & \mapsto 5 \\
		4 & \mapsto 2 \\
		5 & \mapsto 1 \\
	\end{align*}
	Que podemos escribir como:
	$$
	f = \begin{bmatrix}
	1 & 2 & 3 & 4 & 5 \\
	4 & 3 & 5 & 2 & 1 \\
	\end{bmatrix}
	$$

	\begin{propo}
		Sea $X = \{x_1,\ldots,x_n\}$ un conjunto con $n$ elementos. La aplicaci\'{o}n $f: X \to X_n$ definida por $f(x_i) = i$ induce un isomorfismo
		de grupos $S_X \to S_n$ dado por $\alpha \mapsto f \alpha f^{-1}$
	\end{propo}

	\dem Aqui $X = \{x_1,\ldots,x_n\}$, y $f(x_i) = i$. Veamos que $\varphi:S_X \to S_n, \; \varphi(\alpha) \mapsto f \alpha f^{-1}$ es biyecci\'{o}n.

	\begin{gather*}
		f \alpha f^{-1} \in S_n \\
		f \alpha f^{-1} : \{1,\ldots,n\} \overset{f^{-1}}{\to} X \overset{\alpha}{\to} X \overset{f}{\to} \{1,\ldots,n\}
	\end{gather*}
	$f \alpha f^{-1}$ es biyectiva con inversa $f \alpha^{-1} f^{-1}$
	\begin{gather*}
		f \alpha f^{-1} \circ f \alpha^{-1} f^{-1} = \mathrm{Id} = f \alpha^{-1} f^{-1} \circ f \alpha f^{-1}
	\end{gather*}
	
	Veamos que $\varphi$ es biyectiva.
	\begin{itemize}
		\item \textit{Inyectiva:} si $\varphi(\alpha) = \varphi(\alpha')$ entonces
			$$
			f \alpha f^{-1} = f \alpha' f^{-1} \therefore \alpha = \alpha'
			$$
		\item \textit{Suprayectiva:} dado $\beta \in S_n$
			\begin{gather*}
				\beta = f (f^{-1} \beta f) f^{-1} = \varphi(f^{-1} \beta f) \\
				f^{-1} \beta f \in S_X \therefore \alpha := f^{-1} \beta f
			\end{gather*}
	\end{itemize}
	Por lo que $\varphi$ es suprayectiva. \\

	Veamos que $\varphi$ es isomorfismo de grupos. Dado $\alpha,\alpha' \in S_X$
	\begin{gather*}
		\varphi(\alpha \alpha') = f \circ (\alpha \alpha') \circ f^{-1} = f \alpha f^{-1} f \alpha' f^{-1} = \varphi(\alpha) \varphi(\alpha')
	\end{gather*}
	Como $\varphi$ es biyectiva, $\varphi$ es isomorfismo de grupos, por lo que
	\begin{gather*}
		S_X \cong S_n, \quad n = |X|
	\end{gather*}
	
	\begin{teoma}
		Todo grupo es isomorfo a un subgrupo de un grupo de permutaciones.
	\end{teoma}

	\dem Sea $G$ un grupo cualquiera. Para cada $g \in G$, consideramos
	\begin{align*}
		L_g: G & \to G \\
		g' & \mapsto gg'
	\end{align*}

	Veamos que $L_g$ es biyectiva. Su inversa es $\left( L_g \right)^{-1} = L_{g^{-1}}$
	\begin{gather*}
		L_g \circ L_{g^{-1}}(g') = L_g(g^{-1}g') = g(g^{-1}g') = g' = L_{g^{-1}} L_g(g')\\
		\therefore \left( L_g \right)^{-1} = L_{g^{-1}}
	\end{gather*}
	Sea $L_G := \{L_g | g \in G\} \subseteq S_G$. Veamos que $L_G$ es subgrupo de $S_G$. Claramente $L_G \neq \varnothing$, ya que
	$G \neq \varnothing$. Dados $g_1,g_2 \in G$
	\begin{gather*}
		L_{g_1} L_{g_2}(g') = L_{g_1}(g_2 g') = g_1(g_2 g') = (g_1 g_2)g' = L_{g_1 g_2}(g') \\
		\therefore L_{g_1} L_{g_2} = L_{g_1 g_2} \in L_G \therefore \; \text{cerrado por productos} \\
		\left( L_{g_1} \right)^{-1} = L_{g_1^{-1}} \in L_G \therefore \; \text{cerrado por inversos}
	\end{gather*}
	Luego $L_G$ es subgrupo de $S_G$. Ahora, veamos que $L_G \cong G$. Consideramos la aplicaci\'{o}n
	\begin{align*}
		L: G & \to L_G \\
		g & \mapsto L_g
	\end{align*}
	que es isomorfismo de grupos. Claramente es suprayectiva. Tambi\'{e}n es inyectiva ya que
	\begin{gather*}
		L_{g_1} = L_{g_2} \implies g_1e = g_2e \implies g_1 = g_2
	\end{gather*}
	por lo que $L$ es biyectiva. Ahora, vemos que conserva el producto, ya que $\forall g_1,g_2 \in G$
	\begin{gather*}
		L_{g_1 g_2} = L_{g_1} L_{g_2}
	\end{gather*}
	por lo que ya hemos visto. Por ello, $G \cong L_G \leq S_G$
	
	\begin{defin}
		Un elemento $x \in X$ es \textit{fijo} por $\alpha \in S_X$, o $\alpha$ fija $x$ si $\alpha(x) = x$. Si no, decimos que
		$\alpha$ \textit{mueve} $x$. Dos permutaciones son \textit{disjuntas} si mueven conjuntos disjuntos
		de elementos.
	\end{defin}

	\begin{propo}
		Dos permutaciones disjuntas de un mismo conjunto conmutan.
	\end{propo}

	\dem Sean $\alpha,\beta: X \to X$ permutaciones disjuntas. Veamos que $\alpha(\beta(x)) = \beta(\alpha(x)) \forall x \in X$. Distinguimos
	tres casos.
	\begin{enumerate}
		\item Si $\alpha(x) \neq x$. Entonces
			\begin{gather*}
				\alpha(\alpha(x)) \neq \alpha(x) \therefore \beta(x) = x \land \beta(\alpha(x)) = \alpha(x)
			\end{gather*}
			por ser $\alpha,\beta$ disjuntos $\therefore$
			\begin{gather*}
				\alpha(\beta(x)) = \alpha(x) = \beta(\alpha(x))
			\end{gather*}

		\item Si $\beta(x) \neq x$. Es al caso an\'{a}logo.
		\item Si $\alpha(x) = \beta(x) = x$.
			\begin{gather*}
				\alpha(\beta(x)) = \alpha(x) = x = \beta(x) = \beta(\alpha(x))
			\end{gather*}
	\end{enumerate}
	$\therefore$ siempre ocurre que $\alpha(\beta(x)) = \beta(\alpha(x)) \; \forall x \in X$

	\begin{defin}
		Un \textit{ciclo de longitud r} o un \textit{r-ciclo} es una permutaci\'{o}n de un conjunto $X$ para la cual
		existe $x \in X$ tal que $\{x,\alpha(x),\ldots,\alpha^{r-1}(x)\}$ son distintos, pero $\alpha^{r}(x) = x$,
		y el resto de elementos de $X$ son fijos. Se suele denotar con la tupla:
		$$
		(x,\alpha(x),\ldots,\alpha^{r-1}(x))
		$$
		Un 2-ciclo es una \textit{transposici\'{o}n}.
	\end{defin}
	
	\begin{propo}
		El orden de un r-ciclo es $r$.
	\end{propo}

	\dem Sea $\sigma$ un r-ciclo:
	$$
	\sigma = (a_1,\ldots,a_r)
	$$
	y
	\begin{gather*}
		\begin{rcases}
			\sigma(a_1) = a_2 \\
			\sigma^{2}(a_1) = a_3 \\
			\vdots \\
			\sigma^{r-1}(a_1) = a_r \\
			\sigma^{r}(a_1) = a_1
		\end{rcases} \implies \sigma^{i}(a_1) \neq a_1 \forall i=1,\ldots,r-1 \therefore \sigma^{i} \neq \mathrm{I} \forall i=1,\ldots,r-1 \\
		\therefore |\sigma^{i}| \geq r
	\end{gather*}
	Adem\'{a}s
	\begin{gather*}
		\sigma^{r}(a_i) = a_i \forall i = 1,\ldots,r \\
		\sigma^{r}(x) = x, \; x \notin \{a_1,\ldots,a_r\} \\
		\therefore \sigma^{r } = \mathrm{I} \therefore |\sigma| = r
	\end{gather*}
	
	

	\begin{propo}
		Dada $\alpha$ permutaci\'{o}n de $X$, $x_0 \in X$ tal que
		$$
		\underbrace{x_0,\alpha(x_0),\ldots,\alpha^{r-1}(x_0)}_{\text{distintos}}
		$$
		pero $\alpha^{r}(x_0)$ est\'{a} repetido, entonces $\alpha^{r}(x_0) = x_0$.
	\end{propo}

	\dem Esto pasa ya que si $\alpha^{r}(x_0) = \alpha^{i}(x_0)$ para alg\'{u}n $0 \leq i \leq r-1$,
	podemos hacer $\alpha^{r-i}(x_0) = x_0 \therefore$ al ser todos distintos, solo queda que $\alpha^{r}(x_0) = x_0$


	\begin{teoma}
		Toda permutaci\'{o}n de un conjunto finito distinto de la identidad es producto de uno o m\'{a}s ciclos disjuntos
		de longitud $\geq 2$. Adem\'{a}s, esta descomposici\'{o}n es \'{u}nica salvo el orden de los factores (por ser a su vez
		permutaciones disjuntas).
	\end{teoma}

	\dem Dada una permutaci\'{o}n $\alpha$ de $X$ definimos:
	$$
	M_{\alpha} := \{x \in X | \alpha(x) \neq x\}
	$$
	Vamos a probar por induci\'{o}n en el tama\~{n}o de $M_{\alpha}$ que, o bien $\alpha = \mathrm{I}$, o bien
	$\alpha$ es producto de ciclos disjuntos de longitud $\geq 2$ con entradas en $M_{\alpha}$. \\
	Si $|M_{\alpha}| = 0$, entonces todos los elementos son fijos, y $\alpha = \mathrm{I}$. Si suponemos demostrado el caso 0, vemos el paso
	de inducci\'{o}n. Asumimos que $|M_{\alpha}| \geq 1 \therefore \exists x_0 \in M_{\alpha}$, es decir, $x_0 \neq \alpha(x_0)$. Como $X$ es finito,
	$\exists r \geq 2 \text{ t.q. } x_0,\alpha(x_0),\ldots,\alpha^{r-1}(x_0)$ son distintos, pero no lo es $\alpha^{r}(x_0)$. Como hemos visto,
	al llegar este caso, tenemos que $\alpha^{r}(x_0) = x_0$. Formamos un ciclo $\sigma$
	$$
	\sigma := (x_0,\alpha(x_0),\ldots,\alpha^{r-1}(x_0)).
	$$
	Consideramos $\alpha \sigma^{-1}$. Observamos que
	\begin{gather*}
		\alpha \sigma^{-1}(\alpha^{i}(x_0)) = \alpha(\alpha^{i-1}(x_0)) = \alpha^{i}(x_0) \\
		\therefore x_0,\ldots,\alpha^{r-1}(x_0) \notin M_{\alpha \sigma^{-1}}
	\end{gather*}
	Veamos que $M_{\alpha \sigma^{-1}} \subseteq M_{\alpha} \setminus \{x_0,\ldots,\alpha^{r-1}(x_0)\}$.
	En efecto, dado $x \in M_{\alpha \sigma^{-1}}$, por lo anterior, $x \notin \{x_0,\ldots,\alpha^{r-1}(x_0)\} \therefore \sigma(x) = x \therefore
	\sigma^{-1}(x) = x \therefore x \neq \alpha \sigma^{-1}(x) = \alpha(x) \therefore x \in M_{\alpha} \therefore
	M_{\alpha \sigma^{-1}} \subseteq M_{\alpha} \setminus \{x_0,\ldots,\alpha^{r-1}(x_0)\}$. As\'{i}, $|M_{\alpha \sigma^{-1}}| < |M_{\alpha}| \therefore$
	por la hip\'{o}tesis de inducci\'{o}n o bien $\alpha \sigma^{-1} = \mathrm{Id}$ (en cuyo caso $\alpha = \sigma$ y hemos acabado)
	o bien $\alpha \sigma^{-1} = \sigma_1 \cdots \sigma_{l-1}$ producto de ciclos de longitud $ \geq 2$ disjuntos
	con entradas en $M_{\alpha \sigma^{-1}} \therefore \alpha = \sigma_1 \cdots \sigma_{l-1} \sigma$ producto de ciclos
	disjuntos de longitud $\geq 2$ con entradas en $M_{\alpha}$. Con esto, aseguramos existencia. Vamos con la unicidad. \\

	Sea $\alpha$ una permutaci\'{o}n de $X$, $\alpha \neq \mathrm{Id}$ y 
	$$
	\alpha = \sigma_1 \cdots \sigma_{l} = \sigma'_1 \cdots \sigma'_{l'}
	$$
	con $\sigma_1 \cdots \sigma_{l}$ ciclos disjuntos de lo longitud $\geq 2$, y lo mismo para los $\sigma'$. Como $\alpha \neq \mathrm{Id},
	\exists x_0 \text{ t.q. } \alpha(x_0) \neq x_0 \therefore$ existe $\sigma_i$ tal que $\sigma_i \text{ t.q. } \sigma_i(x_0) \neq x_0$.
	Como $\sigma_1 \cdots \sigma_{l}$ conmutan entre s\'{i}, sin p\'{e}rdida de generalidad podemos asumir que $\sigma_1(x_0) \neq x_0 \therefore$
	los dem\'{a}s $\sigma$ lo fijan. Igualmente, podemos asumir que $\sigma'_1(x_0) \neq x_0$. Por esto,
	$\alpha(x_0) = \sigma_1 \cdots \sigma_{l}(x_0) = \sigma_1(x_0) \therefore \alpha(x_0)$ es una entrada de $\sigma_1$ por lo que 
	$\sigma_2 \cdots \sigma_{l}$ lo fijan $\therefore$
	$$
	\alpha^2(x_0) = \alpha(\alpha(x_0)) = \alpha(\sigma_1(x_0)) = \sigma_1(\sigma_1(x_0))
	$$
	As\'{i}, podemos ir construyendo y llegar a que
	$$
	\sigma_1 = (x_0,\alpha(x_0),\alpha^2(x_0),\ldots,\alpha^{r-1}(x_0))
	$$
	donde $r$ es el menor natural $\geq 2$ tal que $\alpha^{r}(x_0) = x_0$. Podemos hacer lo mismo con $\sigma'_1$
	$$
	\sigma'_1 = (x_0,\alpha(x_0),\alpha^2(x_0),\ldots,\alpha^{r-1}(x_0)) = \sigma_1
	$$
	y llegamos a que
	\begin{gather*}
		\sigma_1 = \sigma'_1 \\
		\sigma_2 \cdots \sigma_l = \sigma'_2 \cdots \sigma'_{l'}
	\end{gather*}
	Sin p\'{e}rdida de generalidad, podemos asumir que $l \leq l'$. Repitiendo para $\sigma_2 = \sigma_2',\ldots$ y
	\begin{gather*}
		\mathrm{Id} = \sigma_{l+1}' \cdots \sigma_{l'}'
	\end{gather*}
	

	\textit{Nota: }
	\begin{gather*}
		\sigma = (a_1,\ldots,a_r) \implies \sigma^{-1} = (a_r,a_{r-1},\ldots,a_1)
	\end{gather*} \\

	\textit{Nota: } Si $\sigma = \sigma_1 \cdots \sigma_l$ producto de ciclos disjuntos, entonces $|\sigma| = \mathrm{mcm}(|\sigma_1|,\ldots,|\sigma_l|)$.
	Por un lado
	\begin{gather*}
		\sigma^{\mathrm{mcm}(|\sigma_1|,\ldots,|\sigma_l|)} = \sigma_1^{\mathrm{mcm}(|\sigma_1|,\ldots,|\sigma_l|)} \cdots
		\sigma_l^{\mathrm{mcm}(|\sigma_1|,\ldots,|\sigma_l|)} = \mathrm{Id} \cdots \mathrm{Id} = \mathrm{Id} \\
		\therefore |\sigma| \leq \mathrm{mcm}(|\sigma_1|,\ldots,|\sigma_l|)
	\end{gather*}
	Por otro lado, si $\sigma^{k} = \mathrm{Id}$, entonces
	\begin{gather*}
		\sigma_1^{k} \cdots \sigma_l^{k} = \mathrm{Id} \quad (*)
	\end{gather*}
	Analizamos $\sigma_1^{k}$. Dadaa $x \in X$:
	\begin{gather*}
		\begin{cases}
			\text{si} \; \sigma_1(x) = x \; \text{entonces} \; \sigma_1^{k} = x \\
			\text{si} \; \sigma_1(x) \neq x \; \text{entonces} \; \sigma_2(x) = \ldots = \sigma_l(x) = x
		\end{cases}
	\end{gather*}
	$\therefore$ por (*), tenemos
	\begin{gather*}
		\sigma_1^{k}(x) = x \therefore \sigma_1^{k}(x) = x \; \forall x \therefore \sigma_1^{k} = \mathrm{Id}
	\end{gather*}
	An\'{a}logamente, $\sigma_i^{k} = \mathrm{Id}$, por lo que $k$ es m\'{u}ltiplo de
	$|\sigma_1|,\ldots,|\sigma_l|$, y $|\sigma| \geq \mathrm{mcm}(|\sigma_1|,\ldots,|\sigma_l|)$, y $|\sigma| = \mathrm{mcm}(|\sigma_1|,\ldots,|\sigma_l|)$
	
	\begin{corol}
		Toda permutaci\'{o}n de un conjunto finito es producto de trasposiciones
	\end{corol}

	\dem Observamos que si $a_1,\ldots,a_r$ son elementos distintos.
	\begin{gather*}
		(a_1,a_2)(a_2,a_3)\cdots(a_{r-2},a_{r-1})(a_{r-1},a_r) \\
		a_1 \mapsto a_2 \\
		a_2 \mapsto a_3 \\
		\vdots \\
		a_{r-1} \mapsto a_r \\
		a_r \mapsto a_1 \\
		x \notin \{a_1,\ldots,a_r\} \mapsto x
	\end{gather*}
	$\therefore$ eso es el ciclo $(a_1,a_2,\ldots,a_r)$, por lo que todo ciclo es un producto de trasposiciones. Como toda permutaci\'{o}n es
	producto de ciclos, entonces toda permutaci\'{o}n es producto de trasposiciones.

	\begin{defin}
		La \textit{signatura} de una permutaci\'{o}n $\sigma \in S_n$ es
		$$
		\varepsilon(\sigma) = \prod_{1 \leq i < j \leq n}^{} \frac{\sigma(i) - \sigma(j)}{i-j} \in \{-1,1\}
		$$
	\end{defin}

	\textit{Nota: } Observamos que si $\sigma \in S_n$.
	\begin{gather*}
		\{1,\ldots,n\} = \{\sigma(1),\ldots,\sigma(n)\} \\
		\therefore P := \{S \subset X_n | |S| = 2\} \\
		= \{\{i,j\} | 1 \leq i < j \leq n\} \quad \text{(descr. 1)} \\
		= \{\{\sigma(i),\sigma(j)\} | 1 \leq i < j \leq n\} \quad \text{(descr. 2)}
	\end{gather*}
	Luego, tenemos
	\begin{gather*}
		\varepsilon(\sigma)^2 = \prod_{1 \leq i < j \leq n}^{} \frac{(\sigma(i) - \sigma(j))^2}{(i-j)^2} =
		\frac{\prod_{1 \leq i < j \leq n}^{} (\sigma(i) - \sigma(j))^2}{\prod_{1 \leq i < j \leq n}^{} (i - j)^2}
	\end{gather*}
	Considerar $f(\{i,j\}) := (i-j)^2$ para cada $\{i,j\} \in P$ (bien definida por el cuadrado). $\therefore$
	\begin{gather*}
		\varepsilon(\sigma)^2 = \frac{\prod_{S \in P}^{} f(S) \leftarrow \text{desc. 2}}{\prod_{S \in P}^{} f(S) \leftarrow \text{desc. 1}} = 1
	\end{gather*}
	Luego $\varepsilon(\sigma) \in \{-1,1\}$
	
	\begin{propo}
		Para cualesquiera $\sigma,\tau \in S_n$ se tiene que $\varepsilon(\sigma \tau) = \varepsilon(\sigma) \varepsilon(\tau)$
	\end{propo}

	\dem Dadas $\sigma,\tau \in S_n$
	\begin{gather*}
		\varepsilon(\sigma \tau) = \prod_{1 \leq i < j \leq n}^{} \frac{\sigma \tau(i) - \sigma \tau(j)}{i - j} = \\
		= \prod_{1 \leq i < j \leq n}^{} \frac{\sigma \tau(i) - \sigma \tau(j)}{\tau(i) - \tau(j)} \cdot \frac{\tau(i) - \tau(j)}{i-j}= \\
		= \left( \prod_{1 \leq i < j \leq n}^{} \frac{\sigma \tau(i) - \sigma \tau(j)}{\tau(i) - \tau(j)} \right) \varepsilon(\tau) = (*)
	\end{gather*}
	Consideramos $f_{\sigma}(\{i,j\}) = \frac{\sigma(i) - \sigma(j)}{i-j}$, que est\'{a} bien definida y de hecho, sabemos
	\begin{gather*}
		\varepsilon(\sigma) = \prod_{S \in P}^{} f_{\sigma}(S) = \prod_{1 \leq i < j \leq n}^{} f_{\sigma}(\{i,j\}) = \\
		= \prod_{1 \leq i < j \leq n}^{} \frac{\sigma(\tau(i)) - \sigma(\tau(j))}{\tau(i) - \tau(j)} \\
		\therefore (*) = \varepsilon(\sigma) \varepsilon(\tau)
	\end{gather*}

	\textit{Nota: } $\varepsilon((1,2)) = -1$. Sea $\sigma = (1,2)$
	\begin{gather*}
		\varepsilon(\sigma) = \prod_{1 \leq i < j \leq n}^{} \frac{\sigma(i) - \sigma(j)}{i-j} = \\
		= \frac{\sigma(1) - \sigma(2)}{1-2} \cdots \frac{\sigma(1) - \sigma(n)}{1-n} \cdot \\
		\cdot \frac{\sigma(2) - \sigma(3)}{2-3} \cdots \frac{\sigma(2) - \sigma(n)}{2-n} \cdot \\
		\vdots \\
		\cdot \frac{\sigma(n-1) - \sigma(n)}{n-1 - n} = \\
		= \frac{2-1}{1-2} \cdot \frac{2-3}{1-3} \cdot \frac{2-4}{1-4} \cdots \frac{2-n}{1-n} \cdot \\
		\cdot \frac{1-3}{2-3} \cdot \frac{1-4}{2-4} \cdots \frac{1-n}{2-n} \cdot \\
		\cdot \frac{3-4}{3-4} \cdots \frac{3-n}{3-n} \cdot \\
		\vdots \\
		\cdot \frac{n-1 - n}{n-1-n} = -1
	\end{gather*}
	Adem\'{a}s, $\varepsilon(\mathrm{Id}) = 1$, obvio. Veamos que si $i \neq j$ entonces $\varepsilon((i,j)) = -1$. Consideramos
	la permutaci\'{o}n auxiliar
	\begin{align*}
		f: \; & 1 \mapsto i \\
			 & 2 \mapsto j \\
			 & \vdots \\
			 & n \mapsto \ldots
	\end{align*}
	donde todos menos 1 y 2 van a otros que no sean $i,j$. Consideramos
	\begin{align*}
		f(1,2)f^{-1}: \; & i \mapsto j \\
				 & j \mapsto i \\
				 & k \neq i,j \mapsto k 
	\end{align*}
	Luego $f(1,2)f^{-1} = (i,j)$, y
	\begin{gather*}
		\varepsilon((i,j)) = \varepsilon(f(1,2)f^{-1}) = \varepsilon(f) \varepsilon((1,2)) \varepsilon(f^{-1}) = \\
		= \varepsilon(f) \varepsilon(f^{-1}) \varepsilon((1,2)) = \\
		= \varepsilon(\mathrm{Id}) \varepsilon((1,2)) = 1 \cdot (-1) = -1
	\end{gather*}
	Y, en general
	\begin{gather*}
		\varepsilon\left((a_1,\ldots,a_r)\right) = \varepsilon\left((a_1,a_2),(a_2,a_3) \cdots (a_{r-1},a_r))\right) = (-1)^{r-1}
	\end{gather*}

	\textit{Nota: } $A_n := \{\sigma \in S_n | \varepsilon(\sigma) = 1\}$ es subgrupo de $S_n$.
	\begin{gather*}
		\varepsilon(\mathrm{Id}) = 1 \implies A_n \neq \varnothing \\
		\sigma,\tau \in A_n, \; \varepsilon(\sigma \tau) = \varepsilon(\sigma) \varepsilon(\tau) = 1 \cdot 1 = 1 \implies \sigma \tau \in A_n \\
		\sigma \in A_n, \; 1 = \varepsilon(\mathrm{Id}) = \varepsilon(\sigma \sigma^{-1}) =
		\varepsilon(\sigma) \varepsilon(\sigma^{-1}) = \varepsilon(\sigma^{-1}) \implies \sigma^{-1} \in A_n
	\end{gather*}
	
	\begin{defin}
		Llamamos \textit{grupo alternado de grado n} al conjunto de permutaciones pares (signatura 1), $A_n$, que es subgrupo de $S_n$.
	\end{defin}

	\begin{prdad}
		Para el grupo alternado de grado $n$, tenemos que $|A_n| = \frac{n!}{2}$.
	\end{prdad}

	\dem Esto se debe a que
	\begin{align*}
		\varphi : S_n & \to S_n \\
		\sigma & \mapsto (1,2)\sigma
	\end{align*}
	es una biyecci\'{o}n, que cambia permutaciones pares por impares (y viceversa, al ser su propia inversa). Eso significa que hay tantas permutaciones
	pares como impares.

	\begin{propo}
		Dada $\sigma \in S_n$, se tiene que
		\begin{gather*}
			\sigma \cdot (a_1,\ldots,a_r) \sigma^{-1} = (\sigma(a_1),\ldots,\sigma(a_r))
		\end{gather*}
	\end{propo}

	\dem Dado un $\sigma(a_i)$
	\begin{gather*}
		\sigma(a_i) \overset{\sigma^{-1}}{\mapsto} a_i \overset{(a_1,\ldots,a_r)}{\mapsto} a_{i+1} \overset{\sigma}{\mapsto} \sigma(a_{i+1})
	\end{gather*}
	y a $x \notin \{\sigma(a_1),\ldots,\sigma(a_r)\}$
	\begin{gather*}
		x \mapsto \sigma^{-1}(x) \mapsto \sigma^{-1}(x) \mapsto x
	\end{gather*}

	\subsection{Subgrupos normales. Grupo cociente.}
	
	\begin{defin}
		Dado un grupo $G$, y subconjuntos $A,B \subseteq G$. Definimos $AB = \{ab | a \in A, b \in B\}$.
		Aun siendo $A$ y $B$ subgrupos, $AB$ no suele serlo.
	\end{defin}

	\begin{propo}
		Para cualesquiera subgrupos $H,K \leq G$, se tiene que $|HK| = \frac{|H||K|}{|H \cap K|}$
	\end{propo}

	\dem Vamos a ver que con la aplicaci\'{o}n
	\begin{align*}
		f: H \times K & \to HK \\
		(h,k) & \mapsto hk
	\end{align*}

	Contamos las preim\'{a}genes de $h_0k_0$:
	\begin{gather*}
		|\{(h,k) \in H \times K : hk = h_0k_0\}| = (*)
	\end{gather*}
	Si $hk = h_0k_0$ entonces
	\begin{gather*}
		\underbrace{h_0^{-1}h}_{\in H} = \underbrace{k_0k^{-1}}_{\in K} \in H \cap K \\
		\therefore \exists x \in H \cap K \text{ t.q. } h_0^{-1}h=k_0k^{-1}=x \therefore h = h_0x, \; k=x^{-1}k_0
	\end{gather*}
	Y rec\'{i}procamente, si se cumple $h=h_0x,k=x^{-1}k_0$, entonces $h\in H,k\in K, hk = h_0k_0 \therefore$.
	\begin{gather*}
		(*)=|\{(h_0x,x^{-1}k_0):x \in H \cap K\}| = |H \cap K|
	\end{gather*}
	y contando todas las preim\'{a}genes de todos los elementos de $HK$
	\begin{gather*}
		|H\times K| = |HK||H\cap K| \therefore |HK| = \frac{|H||K|}{|H \cap K|}
	\end{gather*}

	\begin{defin}
		Un subgrupo $H$ de un grupo $G$ es un \textit{subgrupo normal} si $aHa^{-1} = H \; \forall a \in G$.
		Es decir, si $H$ coincide con todos y cada uno de sus conjugados en $G$. Se denota $H \normleq G$.
	\end{defin}

	\textit{Ejemplos}

	\begin{itemize}
		\item $H = \{e\}$, ya que $aHa^{-1} = \{a e a^{-1}\} = \{e\} = H$
		\item $H = G$, ya que $aGa^{-1} \subseteq G = aa^{-1}Ga^{-1}a \subseteq aGa^{-1}$
	\end{itemize}

	\begin{defin}
		Dado un grupo $G$, el \textit{centro} de $G$ es el subgrupo normal
		\begin{gather*}
			\mathrm{Z}(G) = \{a \in G:ab=ba \; \forall b \in G\}
		\end{gather*}
	\end{defin}

	\begin{prdad}
		\begin{itemize}
			\item $e \in \mathrm{Z}(G)$, por lo que $\mathrm{Z}(G) \neq \varnothing$
			\item Dados $a_1,a_2 \in \mathrm{Z}(G)$
				\begin{gather*}
					(a_1a_2)b = a_1(a_2b) = a_1(ba_2) = (a_1b)a_2 = \\
					(ba_1)a_2 = b(a_1a_2) \therefore a_1a_2 \in \mathrm{Z}(G)
				\end{gather*}
			\item $\forall g \in G, \; g\mathrm{Z}(G)g^{-1} = \mathrm{Z}(G)$
			\item Dado $G$ abeliano, $\mathrm{Z}(G) = G$
			\item Si $N$ es subgrupo normal de $G$, entonces $aN = Na, \; \forall a \in G$
		\end{itemize}
	\end{prdad}

	\textit{Ejercicio}
	
	Para $n\geq 3$, demuestra
	\begin{gather*}
		\begin{cases}
			\mathrm{Z}(D_n) = \{e\} & n \text{ impar} \\
			\mathrm{Z}(D_n) = \{e, R^{n/2}\} & n \text{ par} \\
		\end{cases}
	\end{gather*}

	\begin{propo}
		Sean $H,K$ subgrupos de $G$. Equivale decir
		\begin{enumerate}
			\item $HK$ es un subgrupo de $G$
			\item $HK = KH$
			\item $KH$ es subgrupo de $G$
		\end{enumerate}
	\end{propo}

	\dem Veamos que si $HK$ es subgrupo, entonces $HK = KH$. Procedemos por doble contenido. $HK \subseteq KH$? Dados $h, \in H$ y $k \in K$.
	\begin{gather*}
		hk \in HK \implies (hk)^{-1} = \underbrace{k^{-1}}_{\in K} \underbrace{h^{-1}}_{\in H} \in KH  
	\end{gather*}
	Y el otro contenido se hace de forma paralela. Y as\'{i} $HK \leq G \implies HK = KH$. Las implicaciones entre 2 y 3 son iguales.
	Vamos con la implicaci\'{o}n entre 2 y 1. Ahora, $HK = KH$, y queremos ver que $HK \leq G$.
	\begin{gather*}
		e = e\cdot e \in HK \\
		hk \cdot hk = hhkk \in HK \\
		(hk)^{-1} = k^{-1} h^{-1} \in KH = HK
	\end{gather*}
	Luego $HK$ es subgrupo de $G$

	\begin{propo}
		Sea $G$ grupo, $N \normleq G$, y $H \leq G$. Se tiene que $NH = HN$ es un subgrupo que contiene a $N$ y a $H$. Si adem\'{a}s
		$H$ es tambi\'{e}n normal, entonces $NH$ es normal.
	\end{propo}

	\dem Dado $h \in H$.
	\begin{gather*}
		hNh^{-1} = N \therefore hN = Nh, \; \forall h \in H \\
		HN \subseteq NH \subseteq HN \therefore HN = NH \therefore HN \leq G
	\end{gather*}
	esta cadena de contenidos se da ya que
	\begin{gather*}
		h \in H, \; x \in N \\
		hx \in hN = Nh \therefore \exists y \in N \text{ t.q. } hx = yh \in Nh \subseteq NH
	\end{gather*}
	
	Si adem\'{a}s $H \normleq G$, entonces $\forall g \in G$
	\begin{gather*}
		gHNg^{-1} = \underbrace{gHg^{-1}}_{=H} \underbrace{gNg^{-1}}_{=N} = HN
	\end{gather*}
	por lo que $HN \normleq G$.

	\begin{teoma}
		Sea $G$ grupo, $N \normleq G$ y $G/N := \{aN : a \in G\}$. Se tiene que $(aN)(bN) = (ab)N$, por lo que queda definida
		la operaci\'{o}n
		\begin{align*}
			(\cdot): G/N \times G/N & \to G/N \\
			(aN,bN) & \mapsto (ab)N
		\end{align*}
		que dota a $G/N$ de estructura de grupo. El elemento neutro es $eN$ y el inverso es $a^{-1}N$. El grupo $(G/N,\cdot)$ se llama
		\textit{grupo cociente de G por N}.
	\end{teoma}

	\dem En efecto, $G/N \neq \varnothing$, ya que $N = eN \in G/N$.
	\begin{itemize}
		\item Asociatividad.
			\begin{gather*}
				(aN bN) cN = abN cN = (ab)cN = a(bc)N = aN bcN = aN (bN cN)
			\end{gather*}
		\item Neutro
			\begin{gather*}
				aN\cdot eN = aeN = aN \\
				eN \cdot aN = eaN = aN
			\end{gather*}
		\item Inverso
			\begin{gather*}
				aN \cdot a^{-1}N = aa^{-1}N = eN = a^{-1}N \cdot aN
			\end{gather*}
	\end{itemize}

	\begin{propo}
		Sean $H$ un subgrupo de $G$, y $N \normleq G$. Se tiene que:
		\begin{enumerate}
			\item $aH = bH \iff a^{-1}b \in H$ \\
			\item Si $G$ es finito entonces $|G/N| = |G:N| = |G|/|N|$
		\end{enumerate}
	\end{propo}

	\dem Veamos la parte 1. Si $aN = bN$ entonces $a = ae \in aN = bN$. Luego tiene que $\exists x \in N \text{ t.q. } a = bx
	\therefore b^{-1}a = x \in N$. Rec\'{i}procamente, procedemos por doble contenido. Dado $x \in N$
	\begin{gather*}
		ax = b \cdot \underbrace{b^{-1} \cdot a}_{\in N} \cdot \overbrace{x}^{\in N} \in bN \therefore aN \subseteq bN \\
		by = a \cdot \underbrace{a^{-1} \cdot b}_{\in N} \cdot \overbrace{y}^{\in N} \in aN \therefore bN \subseteq aN
	\end{gather*} \\

	\textit{Nota:} Vimos que si $H \leq G$, entonces
	\begin{gather*}
		a \equiv b \iff b = ah \text{ para alg\'{u}n } h \in H
	\end{gather*}
	es la relaci\'{o}n de equivalencia. La clase de equivalencia de $a \in G$ es
	\begin{gather*}
		[a] = \{ah : h \in H\} = aH
	\end{gather*}
	$\therefore$ por las propiedades de las clases de equivalencia
	\begin{gather*}
		aH = bH \text{ \'{o} } aH \cap bH = \varnothing \\
		G = \bigsqcup_{aH \in G/H} aH
	\end{gather*}
	donde $G/H := \{aH | a \in G\}$

	\begin{propo}
		Si $G$ es abeliano, todo subgrupo suyo es normal.
	\end{propo}

	\dem Sea $H \leq G, \, \forall \; x \in G$
	\begin{gather*}
		xHx^{-1} = xx^{-1}H = H
	\end{gather*}
	por lo que $H \normleq G$.

	\begin{propo}
		Sea $G$ grupo, $H \normleq G$, con $|H|=n$, entonces los elementos de $\faktor{G}{H}$ ``colapsan'' de $n$ en $n$. Es decir,
		$\overline{a_1} = \overline{a_2} = \ldots = \overline{a_n}$ con $a_i$ distintos.
	\end{propo}

	\dem Sea $a \in G$, entonces $\overline{a} = a\cdot H = \{ah : h \in H\}$. Notemos que como $e \in H$, entonces $a \in aH$.
	Como $H$ tiene $n$ elementos distintos, $aH$ tambi\'{e}n tendr\'{a} otros $n$ elementos distintos. As\'{i}, para todo $g \in G$, $gH = aH \iff
	g \in aH$. Por ello, los $n$ elementos que colapsan en $aH$ son los elementos de $aH$.
	
	\begin{propo}
		Si el \'{i}ndice de un subgrupo en un grupo es 2 entonces es un subgrupo normal.
	\end{propo}

	\dem Tenemos $H \leq G$, $|G:H| = 2$ por lo que solo hay 2 clases laterales. Una de ellas es $eH = H$. Otra ser\'{a}
	$aH$, donde $e^{-1}a \notin H$, es decir $a \notin H$. Con esto, las \'{u}nicas clases laterales son $eH$ y $aH$. Veamos que
	$H \normleq G$, es decir, $gHg^{-1} = H \; \forall g \in G$. Como sabemos que $G = eH \sqcup aH$, dado $h \in H$, $ha \notin H$,
	ya que $a \notin H$. Entonces
	\begin{gather*}
		ha \in aH \therefore Ha \subseteq aH
	\end{gather*}
	y como $|Ha| = |H| = |aH|$ entonces $aH = Ha$. Dado $g \in G$
	\begin{itemize}
		\item Caso 1. $g \in H$. Aqu\'{i} $gHg^{-1} \subseteq H \therefore gHg^{-1} = H$
		\item Caso 2. $g \in aH$. Aqu\'{i} $g =ah$ para alg\'{u}n $h \in H \therefore$
			\begin{gather*}
				gHg^{-1} = a\underbrace{hHh^{-1}}_{\subseteq H}a^{-1} = aHa^{-1} \underset{aH=Ha}{=} H \therefore \\
				gHg^{-1} = H \; \forall g \in G \therefore H \normleq G
			\end{gather*}
	\end{itemize}
	
	\textit{Nota:} Podemos usar estas dos notaciones, siendo la primera m\'{a}s expl\'{i}cita, y la segunda m\'{a}s c\'{o}moda
	\begin{gather*}
		[a] = aN \\
		\overline{a} = aN
	\end{gather*}
	Y las notaciones nos quedan
	\begin{gather*}
		\overline{a} \overline{b} = \overline{ab}, \quad (\overline{a})^{-1} = \overline{a^{-1}} \\
		[a][b] = [ab], \quad [a]^{-1} = [a^{-1}]
	\end{gather*}

	\begin{teoma}
		Sean $G$ un grupo y $N \normleq G$. Dado un subgrupo $H \leq G$ que contiene a $N$, el conjunto
		\begin{gather*}
			\overline{H} = H/N = \{hN : h \in H\}
		\end{gather*}
		es un subgrupo de $G/N$. De hecho, la aplicaci\'{o}n
		\begin{align*}
			\{\text{subgrupos de $G$ que contienen a $N$}\} & \to \{\text{subgrupos de } G/N\} \\
			H & \mapsto \overline{H}
		\end{align*}
		es una biyecci\'{o}n que preserva contenidos y normalidad.
	\end{teoma}
	
	\dem Aqu\'{i} $N \normleq G$. Adem\'{a}s, definimos las aplicaciones:
	\begin{align*}
		\varphi: \{H \leq G : N \subseteq H\} & \to \{\Omega \leq G/N\} \\
		H & \mapsto \overline{H} = H/N = \{hN : h \in H\}
	\end{align*}
	\begin{align*}
		\psi:  \{\Omega \leq G/N\} & \to \{H \leq G : N \subseteq H\}  \\
		\Omega & \mapsto \{a \in G : aN \in \Omega\}
	\end{align*}

	Veamos que $\varphi$ y $\psi$ son inversas una de la otra, y que cumplen lo dicho en el teorema.
	\begin{itemize}
		\item \underline{$\varphi$ bien definida}. Debemos ver que si $H \leq G$, entonces $\varphi(H) \leq G/N$.
			$\varphi(H) = \overline{H} \neq \varnothing$. Dados $h_1N,h_2N \in \overline{H}$ con $h_1,h_2 \in H$:
			\begin{gather*}
				\begin{rcases}
					(h_1N)(h_2N) = \overbrace{h_1h_2}^{\in H}N \in \overline{H} \\
					(h_1N)^{-1} = \underbrace{h_1^{-1}}_{\in H}N \in \overline{H}
				\end{rcases} \therefore \overline{H} \leq G/N
			\end{gather*}
		\item \underline{$\psi$ bien definida}. Dado $\Omega \leq G/N$, debemos ver que $\psi(\Omega)\leq G$. Como $\Omega \leq G/N$,
			$\Omega$ tiene al neutro $eN \therefore e \in \psi(\Omega) \therefore \psi(\Omega) \neq \varnothing$. Dados
			$a_1,a_2 \in \psi(\Omega), \; a_1N, a_2N \in \Omega \therefore$
			\begin{gather*}
				\begin{rcases}
					(a_1a_2)N = (a_1N)(a_2N) \in \Omega \therefore a_1,a_2 \in \psi(\Omega) \\
					a_1^{-1}N = (a_1N)^{-1} \in \Omega \therefore a_1^{-1} \in \psi(\Omega)
				\end{rcases} \therefore \psi(\Omega) \leq G
			\end{gather*}
			Dado $x \in N$, $xN = eN \in \Omega \therefore x \in \psi(\Omega) \therefore N \subseteq \psi(\Omega)$
		\item \underline{Veamos que $\psi\varphi = \mathrm{Id}$}. Dado $H \leq G$, $N \subseteq H$
			\begin{gather*}
				\psi\varphi(H) = \psi\left( \{hN : h \in H\} \right) = \{a \in G : \exists h \in H \text{ t.q. } aN=hN\} = \\
				= \{a \in G : \exists h \in H \text{ t.q. } h^{-1}a \in N\} = \{a \in G : \exists h \in H \text{ t.q. } a \in hN\} = \\
				\bigcup_{h \in H} hN \underset{N \subseteq H}{\subseteq} H \subseteq \bigcup_{h \in H} hN \implies \\
				\implies \bigcup_{h \in H} hN = H \therefore \psi\varphi(H) = H \therefore \psi\varphi = \mathrm{Id}
			\end{gather*}
		\item \underline{Veamos que $\varphi\psi = \mathrm{Id}$}. Dado $\Omega \leq G/N$:
			\begin{gather*}
				\varphi\psi(\Omega) = \varphi \left( \{a \in G : aN \in \Omega\} \right) = \{aN : aN \in \Omega\} = \Omega \\
				\therefore \varphi\psi(\Omega) = \Omega \therefore \varphi\psi = \mathrm{Id}
			\end{gather*}
	\end{itemize}
	As\'{i}, hemos visto que $\varphi$ y $\psi$ son inversas, por lo que son biyecciones. Vamos con sus propiedades:

	\begin{itemize}
		\item \underline{Veamos que $\varphi$ preserva contenidos}. Dados $H_1, H_2 \leq G$ con $H_1 \subseteq H_2$
			\begin{gather*}
				\varphi(H_1) = \{h_1N:H_1 \in H_1\} \subseteq \{h_2N:h_2 \in H_2\} = \varphi(H_2)
			\end{gather*}
		\item \underline{Veamos que $\psi$ preserva contenidos}. Dados $\Omega_1,\Omega_2 \leq \faktor{G}{N}$, con $\Omega_1 \subseteq \Omega_2$
			\begin{gather*}
				\psi(\Omega_1) = \{a \in G:aN \in \Omega_1\} \subseteq \{a \in G : aN \in \Omega_2\} = \psi(\Omega_2)
			\end{gather*}
		\item \underline{Veamos que $\varphi$ preserva normalidad}. Dado $M \normleq G$, $N \subseteq M$. Debemos comprobar que $\forall \overline{g} =
			gN \in \faktor{G}{N}$ se tiene que $\overline{g}\varphi(M)\overline{g}^{-1} = \varphi(M)$. Por doble contenido.
			Dado $yN$ con $y \in M$
			\begin{gather*}
				\overline{g} \; \overline{y} \; \overline{g}^{-1} = \overline{gyg^{-1}} \underbrace{\in}_{gyg^{-1} \in M} \varphi(M) \therefore
				\overline{g}\varphi(M)\overline{g}^{-1} \subseteq \varphi(M).
			\end{gather*}
			Si cambiamos $\overline{g}$ por $\overline{g}^{-1}$ tenemos
			\begin{gather*}
				\overline{g}^{-1}\varphi(M)\overline{g} \subseteq \varphi(M) \therefore \varphi(M) \subseteq \overline{g}\varphi(M)\overline{g}^{-1}
				\therefore\overline{g} \varphi(M) \overline{g}^{-1} = \varphi(M) \therefore \\ \varphi(M) \normleq \faktor{G}{N}
			\end{gather*}
			
		\item \underline{Veamos que $\psi$ preserva normalidad}. Dado $\Omega \normleq \faktor{G}{N}$. Debemos comprobar que $\forall g \in G$
			se tiene que $g\psi(\Omega)g^{-1} = \psi(\Omega)$. Por doble contenido.	Dado $a \in \psi(\Omega)$
			\begin{gather*}
				gag^{-1}N = \overline{gag^{-1}} = \overline{g} \; \overline{a} \; \overline{g}^{-1} \underbrace{\in}_{aN \in \Omega} \Omega
				\implies gag^{-1} \in \psi(\Omega) \therefore g\psi(\Omega)g^{-1} \subseteq \psi(\Omega).
			\end{gather*}
			Si cambiamos $g$ por $g^{-1}$ tenemos
			\begin{gather*}
				g^{-1}\psi(\Omega)g \subseteq \psi(\Omega) \therefore \psi(\Omega) \subseteq g\psi(\Omega)g^{-1} \therefore
				g\psi(\Omega)g^{-1} = \psi(\Omega) \therefore \\ \psi(\Omega) \normleq \faktor{G}{N}
			\end{gather*}
	\end{itemize}

	\begin{defin}
		Un grupo no trivial es \textit{simple} si no posee subgrupos normales propios no triviales.
	\end{defin}

	\begin{teoma}
		\begin{itemize}
			\item $A_2$ es el grupo trivial
			\item $A_3$ es grupo abeliano simple
			\item $A_4$ no es ni abeliano ni simple
			\item $A_n$ es un grupo simple no abeliano para $n \geq 5$
		\end{itemize}
	\end{teoma}

	En el caso de $A_4$, vemos que
	\begin{gather*}
		N := \{I,(12)(34),(13)(24),(14)(23)\}
	\end{gather*}
	es subgrupo de $A_4$
	\begin{gather*}
		(ij)(kl) \cdot (ab)(cd) \underset{SPDG, a=i}{=} (ij)(kl)(ib)(cd) \underset{SPDG}{=}
		\begin{cases}
			\mathrm{Id} \in N & ,b=j \\
			(ij)(kl)(ik)(jl) = (il)(kj) \in N & ,b \neq j
		\end{cases}
	\end{gather*}
	$\therefore N$ es subgrupo de $A_4$. Adem\'{a}s $\forall \sigma \in A_4$
	\begin{gather*}
		\sigma(ij)(kl)\sigma^{-1} = (\sigma(i)\sigma(j))(\sigma(k)\sigma(l)) \in N
	\end{gather*}
	$\therefore N$ es subgrupo normal de $A_4 \therefore A_4$ no es simple. \\

	Para el caso de $A_3$, tenemos que $A_3 \cong C_3$, que es simple y abeliano, y $A_2 = \{\mathrm{Id}\}$. \\

	Ahora, para $A_n$ con $n \geq 5$. Tenemos que $N \normleq A_n$ no trivial. Al no ser trivial, $\exists \alpha \in N$, y cogiendo cualquier
	3-ciclo de $A_n$, resulta que
	\begin{gather*}
		\alpha(a \; b \; c)^{-1} \alpha^{-1} (a \; b \; c) = (\alpha(c) \; \alpha(b) \; \alpha(a))(a \; b \; c) \quad (*)
	\end{gather*}
	es elemento de $N$, por su normalidad. Para evitar que $N = A_n$, y por lo tanto, llegar a una contradicci\'{o}n, tenemos que evitar tener ciertos
	elementos en $N$, ya que su pertenencia hacen que $N$ sea el grupo ambiente. Estos son:
	\begin{enumerate}
		\item Los 3-ciclos. En efecto, si tenemos que $(x \; y \; z) \in N$, entonces $(x \; y \; z)^2 = (x \; z \; y) \in N$, y podemos cambiar cualquier
			elemento del ciclo, por ejemplo $x$, por $a \notin \{x,y,z\}$, y seguir\'{a} estando en $N$. Esto se debe a que
			\begin{gather*}
				\underbrace{(x \; y \; a)}_{\in A_n} \underbrace{(x \; y \; z)}_{\in N} \underbrace{(x \; y \; z)^{-1}}_{\in A_n} 
				\underset{(*)}{=} (y \; a \; z) \in N
			\end{gather*}
			y al elevar al cuadrado, tenemos que $(a \; y \; z) \in N$. Este proceso se puede repetir para obtener todos los 3-ciclos, y por
			ende, $N = A_n$.
		\item El producto de 2 trasposiciones distintas, entonces contiene alg\'{u}n 3-ciclo. Para trasposiciones no disjuntas $(a \; b)$ y $(b \; c)$
			, su producto es $(a \; b)(b \; c) = (a \; b \; c)$. En el caso de que sean disjuntas, $\alpha := (a \;b)(c \; d) \in N$,
			observamos que
			\begin{gather*}
				\alpha (a \; x)(c \; d)\alpha^{-1}(a \; x)(c \; d) \underset{(*)}{=}  (a \; b \; x)
			\end{gather*}
			por normalidad de $N$.
		\item Si $N$ tiene alg\'{u}n 5-ciclo, entonces tiene alg\'{u}n 3-ciclo. Si $\alpha := (a_1 \; a_2 \; a_3 \; a_4 \; a_5) \in N$, podemos aplicar (*)
			al 3-ciclo $(a_1 \; a_2 \; a_3)$, para obtener que $(a_1 \; a_4 \; a_3) \in N$
	\end{enumerate}

	Ahora bien, tenemos nuestro $N \normleq A_n$, demostremos gracias a esta preparaci\'{o}n que no existe m\'{a}s all\'{a} de subgrupos triviales. Sea
	$\alpha \in N$. Si $\alpha$ fija a todos los elementos que no sean $\{\alpha^{-1}(x),x,\alpha(x),\alpha^2(x)\}$, entonces disponemos
	de 4 elementos para formar su descomposici\'{o}n en ciclos disjuntos. Esto hace que sea, o un 3-ciclo, o un producto de trasposiciones distintas,
	lo cual hace que $N$ sea trivial, lo que es absurdo. Ahora, $\alpha$ no fija a un $z \notin \{\alpha^{-1}(x),x,\alpha(x),\alpha^2(x)\}$. Al aplicar
	(*) al ciclo $(x \; \alpha(x) \; z)$ obtenemos $\beta := (\alpha(z) \; \alpha^2(x) \; \alpha(x))(x \; \alpha(x) \; z) \in N$. Si $\alpha^2(x) = x$,
	entonces $\beta = (x \; \alpha(z))(z \; \alpha(x))$, y hemos vuelto a acabar, o que $\alpha^2(x) \neq x$, que entonces $\beta = (x \; \alpha(z)
	\; \alpha^2(x) \; \alpha(x) \; z)$, que es un 5-ciclo, y hemos acabado. \\

	As\'{i}, hemos visto que suponer que existe $N \normleq A_n$ no trivial es absurdo. \\

	\textit{Ejemplos:} \\

	\underline{Grupo cociente de un grupo c\'{i}clico $C_n$}, donde $C_n = \gen{a}, |a| = n$. Como $C_n$ es abeliano, entonces todo subgrupo suyo es normal,
	ya que
	\begin{gather*}
		\sigma H \sigma^{-1} = \sigma \sigma^{-1} H = H
	\end{gather*}
	Que pasa para todos los abelianos. Si $d$ es divisor de $n$, sabemos que $N := \gen{a^{d}}$ tiene orden $\frac{n}{d}$. El cociente
	tendr\'{a} orden $|G/N| = \frac{|G|}{|N|} = \frac{n}{n/d} = d$. Adem\'{a}s,
	\begin{gather*}
		G/N = \{a^{i}N : i=0,\ldots,n-1\} \underset{\overline{a} := aN}{=} \{\overline{a}^{i} : i=0,\ldots,n-1\} = \gen{\overline{a}}
	\end{gather*}
	$\therefore G/N$ es c\'{i}clico $\therefore G/N \cong C_d$. Lo que pasa, es que al tomar las potencias $i=0,\ldots,n-1$, estamos
	repitiendo mucho, cuando solo hace falta tomar $i=0,\ldots,d-1$, ya que si vamos haciendo potencias
	\begin{gather*}
		\overline{e},\overline{a},\overline{a}^2,\ldots,\overline{a}^{d} = a^{d}N \underset{a^{d} \in N}{=} eN = \overline{e}
	\end{gather*}
	Por ello, escribimos
	\begin{gather*}
		G/N = \{\overline{a}^{i} : i =0,\ldots,d-1\}
	\end{gather*} \\

	Grupo cociente de un di\'{e}drico. Sea $N \normleq D_n, \; N \neq \{\mathrm{Id}\}$.
	\begin{itemize}
		\item Caso 1. $\exists H$ reflexi\'{o}n, $H \in N$.
			\begin{gather*}
				R^2H = RHR^{-1} \in N \text{, por ser } N \normleq D_n \\
				\therefore R^2 = \underbrace{(R^2H)}_{\in N} \underbrace{H}_{\in N} \in N \\
				\therefore R^{2i}, HR^{2i} \in N \; \forall i
			\end{gather*}
			Si $n$ es impar, entonces $|R^2| = n$ hace que $N$ contenga $R^{i} \; \forall i \therefore$ tambi\'{e}n $HR^{i} \therefore N = D_n$ y
			$|D_n/N| = 1 \therefore D_n/N = \{\mathrm{Id}\}$.

			Si $N$ contiene alg\'{u}n elemento m\'{a}s que $R^{2i},HR^{2i} \; \forall i$ entonces tiene rotaciones impares, por lo
			que tendr\'{i}a todas las rotaciones, y $N = D_n$, y $D_n/N = \{\mathrm{Id}\}$.

			Si $N = \gen{R^2,H}$ y $n$ es par, entonces $|N| = 2 \cdot \frac{n}{2} = n$, y $|D_n/N| = 2$, por lo que
			$D_n/N \cong C_2$

			$\gen{R^2,H}$ s\'{i} es un subgrupo normal
			\begin{gather*}
				R^{i} R^{2k} R^{-i} = R^{2k} \in N \\
				(HR^{i}) R^{2k} (HR^{i})^{-1} = R^{-2k} \in N \\
				R^{i} HR^{2k} R^{-i} \in N \\
				(HR^{i}) HR^{2k} (HR^{i})^{-1} \in N
			\end{gather*}
		\item Caso 2. $N$ no contiene ninguna reflexi\'{o}n. $N \subseteq \gen{R} \therefore N = \gen{R^{d}}$ para alg\'{u}n $d$
			divisor de $n$. Notar que, en efecto, todo subgrupo del grupo de rotaciones es normal en $D_n$, ya que
			si $S = \gen{R^{d}}$ entonces (usamos $S$ como subgrupo ya que $H$ se usa en reflexiones)
			\begin{gather*}
				R^{i} R^{dk} R^{-i} = R^{dk} \in S \\
				(HR^{i}) R^{dk} (HR^{i})^{-1} = R^{-dk} \in S \\
				\therefore \forall g \in D_n, gSg^{-1} \subseteq S \land g^{-1}Sg \subseteq S \therefore S \subseteq gSg^{-1} \\
				\therefore gSg^{-1} = S
			\end{gather*}
			Para $N = \gen{R^{d}}$, $|D_n/N| = \frac{|D_n|}{|N|} = \frac{2n}{n/d} = 2d$.
			\begin{gather*}
			D_n/N = \{R^{i}N, HR^{i}N : i=0,\ldots,n-1\} = \{\overline{R}^{i}, \overline{H} \; \overline{R}^{i} : i=0,\ldots,n-1\}
			\end{gather*}
			Por lo que hay repeticiones, al haber m\'{a}s de $2d$ elementos. Observamos que
			\begin{gather*}
				\overline{e},\overline{R},\overline{R}^2,\ldots,\overline{R}^{d} \underset{R^{d} \in N}{=} \overline{e} \\
			\therefore D_n/N = \{\overline{R}^{i}, \overline{H} \; \overline{R}^{i} : i=0,\ldots,d-1\}
			\end{gather*}
			Como en esta descripci\'{o}n no hay repeticiones entre los elementos:
			\begin{gather*}
				\overline{R}^{d} = \overline{e} \\
				\overline{H}^2 = \overline{H^2} = \overline{e} \\
				\overline{H} \; \overline{R} \; \overline{H}^{-1} = \overline{HRH^{-1}} = \overline{R^{-1}} = \overline{R}^{-1} \\
				\therefore D_n/N \cong D_d
			\end{gather*}
	\end{itemize}
	
	\begin{prdad}
		Producto cartesiano de grupos es grupo. Si $G_1$ y $G_2$ son grupos, entonces $G_1 \times G_2$ es grupo con la operaci\'{o}n
		\begin{gather*}
			(a_1,a_2) \cdot (b_1,b_2) := (a_1b_1,a_2b_2)
		\end{gather*}
		Si adem\'{a}s los dos son abelianos, el producto as\'{i} lo ser\'{a}. Tambi\'{e}n se puede escribir como $G_1 \bigoplus G_2$.
	\end{prdad}

	\subsection{Homomorfismos de grupo.}

	\begin{defin}
		Recordamos que, dados $(G_1,\cdot),(G_2,\ast)$, un homomorfismo es una aplicaci\'{o}n $f:G_1 \to G_2$ tal que $f(a\cdot b) = f(a) \ast f(b)$.
		Si es inyectivo, es \textit{monomorfismo}. Si es suprayectivo, es \textit{epimorfismo}. Si son biyectivos, son \textit{isomorfismos}.
	\end{defin}

	\begin{prdad}
		Sea $f:G_1 \to G_2$ homomorfismo de grupos, $a \in G_1$ y $n \in \mathbb{Z}$.
		\begin{enumerate}
			\item $f(e) = e$
			\item $f(a^{-1}) = f(a)^{-1}$
			\item $f(a^{n}) = f(a)^{n}$
			\item Si el orden de $a$ es finito entonces el orden de $f(a)$ divide al orden de $a$.
				En caso de que $f$ sea isomorfismo, los \'{o}rdenes de $a$ y $f(a)$ coinciden.
		\end{enumerate}
	\end{prdad}

	\dem Veamos 1)
	\begin{gather*}
		f(e) = f(ee) = f(e)\ast f(e) \underset{\text{por } (f(e))^{-1} \in G_2}{\implies} e = f(e)
	\end{gather*}

	Veamos 2)
	\begin{gather*}
		aa^{-1} = e \implies f(aa^{-1}) = f(e) = e \implies f(a) \ast f(a^{-1}) = e \implies f(a^{-1}) = (f(a))^{-1}
	\end{gather*}

	Veamos 3)
	\begin{gather*}
		f(a^{n}) = f(a^{n-1}a) = f(a^{n-1}) \ast f(a) = \ldots = f(a)^{n}
	\end{gather*}

	Veamos 4)
	Si $|a| < \infty$, digamos $n := |a|$.
	\begin{gather*}
		a^{n} = e \implies f(a^{n}) = f(e) = e \implies f(a)^{n} = e \implies |f(a)| < \infty
	\end{gather*}
	y $|f(a)|$ divide a $n \therefore |f(a)|$ divide a $|a|$. Si adem\'{a}s $f$ es isomorfismo, entonces para $m := |f(a)|$
	\begin{gather*}
		f(a)^{m} = e \implies f(a^{m}) = f(e) = e \implies a^{m} = e
	\end{gather*}
	$\therefore$ $m$ divide a $|a| = n$, y $n$ divide a $m$, por lo que son iguales.

	\begin{propo}
		Sean $f_1,f_2$ homomorfismos entre dos grupos $G_1,G_2$. Si $f_1$ y $f_2$ coinciden en un conjunto generador de $G_1$ entonces los
		homomorfismos son iguales.
	\end{propo}

	\dem Sean $f_1,f_2:G_1 \to G_2$ homomorfismos de grupos. Sea $X$ un conjunto generador de $G_1$. Por hip\'{o}tesis $\forall a \in X$ se tiene
	que $f_1(a) = f_2(a)$. Como $X$ genera $G_1$ entonces
	\begin{gather*}
		G_1 = \left\{ a_1 \cdots a_r : r \geq 0, \; a_i \in X \lor a_i^{-1} \in X \right\}
	\end{gather*}
	Dado $g \in G_1$, $g = a_1,\cdots,a_r$ para un cierto $r \geq 0$ y $a_1,\ldots,a_r \in X \cup X^{-1}$. $|f(a)|$ divide a $n \therefore |f(a)|$ divide
	a $|a|$.
	\begin{gather*}
		f_1(g) = f_1(a_1 \cdots a_r) = f_1(a_1) \cdots f_1(a_r) = f_2(a_1) \cdots f_2(a_r) = f_2(a_1 \cdots a_r) = f_2(g)
	\end{gather*}
	as\'{i}, $f_1 = f_2$ \\
	
	\textit{Ejemplo: } Sea $G$ un grupo y $a \in G, |a| = n$. La aplicaci\'{o}n
	\begin{align*}
		f: (\mathbb{Z}_n,+) & \to G \\
		\overline{k} & \mapsto a^{k}
	\end{align*}

	es monomorfismo de grupos. Veamos que
	\begin{itemize}
		\item $f$ est\'{a} bien definida. Si $\overline{k_1} = \overline{k_2}$ entonces $k_2 = k_1 + cn$ para alg\'{u}n $c \in \mathbb{Z}$
			\begin{gather*}
				a^{k_2} = a^{k_1 + cn} = a^{k_1} \cdot (a^{n})^{c} = a^{k_1} e^{c} = a^{k_1}
			\end{gather*}
			$\therefore f(\overline{k_2}) = f(\overline{k_1})$, y est\'{a} bien definida.
		\item $f$ es homomorfismo. Dados $\overline{k_1}, \overline{k_2} \in \mathbb{Z}_n$
			\begin{gather*}
				f(\overline{k_1} + \overline{k_2}) = f(\overline{k_1 + k_2}) = a^{k_1 + k_2} = a^{k_1} \cdot a^{k_2} =
				f(\overline{k_1})f(\overline{k_2})
			\end{gather*}
		\item $f$ es inyectiva. Si $f(\overline{k_1}) = f(\overline{k_2})$ entonces $a^{k_1} = a^{k_2} \therefore a^{k_2 - k_1} = e$.
			$n = |a|$ divide a $k_2-k_1 \therefore k_2 \equiv k_1 \; (\mathrm{mod} \; n) \therefore \overline{k_2} = \overline{k_1}$
	\end{itemize}

	\textit{Ejemplo:} Dados un grupo $G$ y un subgrupo normal $N \normleq G$, tenemos
	\begin{align*}
		\pi: G & \to G/N \\
		a & \mapsto aN = \overline{a}
	\end{align*}
	que es epimorfismo de grupos.
	\begin{gather*}
		\forall a,b \in G \\
		\pi(ab) = \overline{ab} = \overline{a}\; \overline{b} = \pi(a) \pi(b)
	\end{gather*}
	$\therefore \pi$ es homomorfismo, y claramente suprayectiva. Notar que siempre que $a \in N$ tenemos
	\begin{gather*}
		\pi(a) = \overline{a} = aN = eN
	\end{gather*}
	por lo que todos los elementos de $N$ tienen la misma imagen, que es $\overline{e}$. \\

	\begin{propo}
		Sea $G$ un grupo que posee dos \textit{subgrupos normales}, $N_1,N_2 \normleq G$ tales que
		\begin{enumerate}
			\item $N_1 \cap N_2 = \{e\}$
			\item $|G| = |N_1| \cdot |N_2|$
		\end{enumerate}
		En tal caso, $G$ es isomorfo a $N_1 \times N_2$. Consideramos
		\begin{align*}
			f: N_1 \times N_2 & \to G \\
			(g_1,g_2) & \mapsto g_1g_2
		\end{align*}
	\end{propo}

	\dem Veamos que es isomorfismo. Primero necesitamos ver que los elementos de $N_1$ conmutan con los de $N_2$. Dados
	$a_1 \in N_1, a_2 \in N_2\;$
	\begin{gather*}
		\underbrace{a_1 \underbrace{a_2 a_1^{-1} a_2^{-1}}_{\in N_1}}_{\in N_1}
	\end{gather*}
	Pero organizando de otra forma
	\begin{gather*}
		\underbrace{\underbrace{a_1 a_2 a_1^{-1}}_{\in N_2} a_2 ^{-1}}_{\in N_2}
	\end{gather*}
	Por lo que $a_1 a_2 a_1^{-1} a_2^{-1} = e \therefore a_1a_2 = a_2a_1$. Veamos que $f$ es homomorfismo
	\begin{gather*}
		f \left( (a_1,a_2)(b_1,b_2) \right) = f \left( (a_1b_1,a_2b_2) \right) = a_1b_1a_2b_2 = \\
		= a_1a_2 b_1b_2 = f \left( (a_1,a_2) \right) f \left( (b_1,b_2) \right) 
	\end{gather*}
	Veamos que es inyectiva.
	\begin{gather*}
		f((a_1,a_2)) = f((b_1,b_2)) \implies a_1a_2 = b_1b_2 \implies b_1^{-1}a_1 = b_2a_2^{-1} \in N_1 \cap N_2 = \{e\} \implies \\
		\implies b_1^{-1}a_1 = e b_2a_2^{-1} = e \implies \begin{cases}
			a_1=b_1 \\
			a_2 = b_2
		\end{cases}
	\end{gather*}
	Por lo que es inyectiva. Veamos la suprayectividad
	\begin{gather*}
		|N_1 \times N_2| = |N_1||N_2| = |G|
	\end{gather*}
	Por lo que es suprayectiva, por ser inyectiva. \\

	\textit{Ejercicio: } Descomponer as\'{i} a $D_{2m} \cong D_m \times C_2$, con $m$ impar. \\

	En este caso, podemos hacer la descomposici\'{o}n:
	\begin{gather*}
		D_{2m} = N_1 \times N_2 \\
		\begin{cases}
			N_1 = \{R^{2},R^{4},\ldots,R^{2m}=e,HR^{2},HR^{4},\ldots,HR^{2m}=H\} \cong D_m \\
			N_2 = \{e,R^{m}\} \cong C_2
		\end{cases}
	\end{gather*}
	En este caso, si demostramos que esos dos subgrupos son normales, ya hemos visto que la descomposici\'{o}n de $D_{2m}$ es cierta, ya que cumplen que
	la intersecci\'{o}n es $\{e\}$, y que $|D_{2m}| = |N_1| \cdot |N_2|$, ya que $4m = 2m \cdot 2$. Veamos que $N_1$ es normal:
	\begin{gather*}
		R^{i} \cdot R^{2k} \cdot R^{-i} = R^{2k} \in N_1 \\
		R^{i} \cdot HR^{2k} \cdot R^{-i} = R^{i} \cdot HR^{2k-i} = R^{i} \cdot R^{i-2k}\cdot H = R^{2(i-k)} \cdot H = HR^{2(k-i)} \in N_1 \\
		HR^{i} \cdot R^{2k} \cdot (HR^{i})^{-1} = HR^{i} R^{2k} R^{-i} H = H R^{2k} H = R^{-2k} \in N_1 \\
		HR^{i} \cdot HR^{2k} \cdot (HR^{i})^{-1} = R^{2k-i} \cdot R^{-i} \cdot H = R^{2(k-i)} \cdot H = HR^{2(i-k)} \in N_1
	\end{gather*}
	Por lo que $N_1 \normleq D_{2m}$. Ahora, vemos con $N_2$
	\begin{gather*}
		R^{i} \cdot e \cdot R^{-i} = e \in N_2 \\
		R^{i} \cdot R^{m} \cdot R^{-i} = R^{m} \in N_2 \\
		HR^{i} \cdot e \cdot (HR^{i})^{-1} = e \in N_2 \\
		HR^{i} \cdot R^{m} \cdot (HR^{i})^{-1} = H \cdot R^{i + m - i} \cdot H = R^{-m} = R^{m} \in N_2 \\
	\end{gather*}
	Luego, hemos encontrado una descomposici\'{o}n del tipo $D_{2m} \cong D_m \times C_2$

	\textit{Ejemplo:} Otro ejemplo puede ser descomponer $C_n \cong C_{n_1} \times C_{n_2}$, con $n = n_1 \cdot n_2, \; \mathrm{mcd}(n_1,n_2) = 1$.
	Como $n_1$ y $n_2$ dividen a $n$, existen subgrupos $N_1, N_2 \normleq C_n$ con $|N_1| = n_1, \; |N_2| = n_2$.
	Adem\'{a}s $N_1 \cap N_2 = \{e\}$, ya es subgrupo de $N_1$ y de $N_2$, y que su orden divide a $n_1$ y a $n_2$, por lo que tiene orden 1.

	\begin{propo}
		Un homomorfismo de grupos es inyectivo si y solo si su n\'{u}cleo es el subgrupo trivial.
	\end{propo}

	\dem Tomamos $f:G_1 \to G_2$ homomorfismo de grupos. Si $f$ es inyectivo, $\forall g \in \ker f, \; f(g) = e = f(e) \therefore
	g = e \therefore \ker f = \{e\}$. Si $\ker f = \{e\}$, entonces $\forall g_1,g_2 \in G$ con $f(g_1) = f(g_2)$
	\begin{gather*}
		f(g_1) \left( f(g_2) \right)^{-1} = e \therefore f(g_1)f(g_2^{-1}) = e \therefore f(g_1g_2^{-1}) = e \therefore \\
		g_1g_2^{-1} \in \ker f \therefore g_1g_2^{-1} = e \therefore g_1 = g_2
	\end{gather*}
	Luego, es inyectiva

	\begin{teoma}
		\textit{Primer teorema de isomorf\'{i}a:} Sea $f:G_1 \to G_2$ homomorfismo de grupos. Se tiene que $\ker f$ es un
		subgrupo normal de $G_1$ e $\im f$ es un subgrupo de $G_2$. Adem\'{a}s, la aplicaci\'{o}n
		\begin{align*}
			\overline{f}: G_1/\ker f & \to \im f \\
			a \ker f & \mapsto f(a)
		\end{align*}
		est\'{a} bien definida y es un isomorfismo de grupos.
	\end{teoma}
	
	\dem Veamos que $\ker f \normleq G_1$.
	\begin{gather*}
		f(e) = e \implies e \in \ker f \therefore \ker f \neq \varnothing \\
		g_1,g_2 \in \ker f \implies f(g_1g_2) = f(g_1) f(g_2) = e \cdot e = e \implies g_1g_2 \in \ker f \\
		g \in \ker f \implies f(g^{-1}) = f(g_1)^{-1} = e^{-1} = e \implies g^{-1} \in \ker f
	\end{gather*}
	Luego $\ker f \leq G_1$. Dado $a \in G$ y $g \in \ker f$
	\begin{gather*}
		f(aga^{-1}) = f(a) \underbrace{f(g)}_{e} f(a)^{-1} = e \therefore aga^{-1} \in \ker f \therefore a \cdot \ker f \cdot a^{-1} \subseteq \ker f
	\end{gather*}
	Luego $\ker f$ es normal, y podemos considerar el grupo cociente $G_1 / \ker f$. Veamos que $\im f \leq G_2$
	\begin{gather*}
		f(e) = e \implies e \in \im f \therefore \im f \neq \varnothing \\
		g_1,g_2 \in \im f, \; \exists a_1,a_2 \in G_1 \text{ t.q. } g_1 = f(a_1),g_2 = f(a_2) \therefore g_1g_2 = f(a_1a_2) \in \im f \\
		g \in \im f, \; \exists a \in G_1 \text{ t.q. } g = f(a) \therefore g^{-1} = f(a^{-1}) \in \im f
	\end{gather*}
	Luego $\im f \leq G_2$. Veamos que est\'{a} bien definida la aplicaci\'{o}n
	\begin{align*}
		\overline{f}: G_1/\ker f & \to \im f \\
		a \ker f & \mapsto f(a)
	\end{align*}
	Si $a \ker f = b \ker f$ entonces $b^{-1}a \in \ker f \therefore$
	\begin{gather*}
		f(b^{-1}a) = e \therefore f(b)^{-1}f(a) = e \therefore f(a) = f(b)
	\end{gather*}
	Ahora, veamos que $\overline{f}$ es homomorfismo
	\begin{gather*}
		\overline{f}(\overline{a} \overline{b}) = \overline{f}(\overline{ab}) = f(ab) = f(a) f(b) = \overline{f}(\overline{a}) \overline{f}(\overline{b})
	\end{gather*}
	Veamos que $\overline{f}$ es inyectiva
	\begin{align*}
		\ker \overline{f} &= \{\overline{a} \in G / \ker f : \overline{f}(\overline{a}) = e\} = \\
				  &= \{\overline{a} \in G/ \ker f : f(a) = e\} = \\
				  &= \{\overline{a} \in G/ \ker f : a \in \ker f\} = \\
				  &= \{\overline{e}\}
	\end{align*}
	Por lo que es $\overline{f}$ es inyectiva. Obviamente es suprayectiva. Luego, $\overline{f}$ es isomorfismo y tenemos que
	\begin{gather*}
		\im f \cong G / \ker f
	\end{gather*}

	\begin{teoma}
		\textit{Segundo teorema de isomorf\'{i}a:} Sea $G$ un grupo, $N \normleq G$, y $H \leq G$.
		Entonces, $NH \leq G$, $N \normleq NH$, $N \cap H \normleq H$ y
		\begin{align*}
			f: H/(N \cap H) & \to NH/N \\
			h N \cap H & \mapsto hN
		\end{align*}
		es isomorfismo.
	\end{teoma}

	\dem Ya hemos visto que $NH \leq G$. $N \leq NH$, ya que $\forall x \in N$, $x = xe \in NH \therefore N \subseteq NH$ y como $N \leq G$, $N \leq NH$.
	Adem\'{a}s, $\forall g \in NH$
	\begin{gather*}
		gNg^{-1} = N \text{ por ser } N \normleq G
	\end{gather*}
	$\therefore N \normleq NH$. Veamos que $N \cap H \normleq H$. $N \cap H \leq H$. Para todo $h \in H$ y $x \in N \cap H$.
	\begin{gather*}
		h\underbrace{x}_{\in N, \in H}h^{-1} \underbrace{\in}_{N \normleq G, x \in H} N \cap H
	\end{gather*}
	$\therefore N \cap H \normleq H$.
	
	Ahora, consideramos
	\begin{align*}
		f: H & \to NH/N \\
		h & \mapsto hN = \overline{h}
	\end{align*}
	que es homomorfismo de grupos.
	\begin{gather*}
		f(h_1h_2) = \overline{h_1h_2} = \overline{h_1}\cdot\overline{h_2}=f(h_1)f(h_2)
	\end{gather*}
	De hecho, es suprayectiva, ya que dado $x \in N, h \in H$
	\begin{gather*}
		xhN = h\underbrace{h^{-1}xh}_{\in N}N = \overline{h} \cdot \overline{h^{-1}xh} = \overline{h} \cdot \overline{e} = \overline{h} = hN \\
		\therefore xhN = hN = f(h) \therefore f(H) = NH/N
	\end{gather*}
	Ahora, analicemos el n\'{u}cleo
	\begin{align*}
		\ker f &= \{h \in H : f(h) = \overline{e}\} = \\
		       &= \{h \in H : hN = eN\} = \\
		       &= \{h \in H : h \in N\} = H \cap N
	\end{align*}
	Por el primer teorema de isomorf\'{i}a, queda inducida:
	\begin{align*}
		\overline{f}: H/(N \cap H) & \to NH/N \\
		hN\cap H & \mapsto hN
	\end{align*}

	\begin{teoma}
		\textit{Tercer teorema de isomorf\'{i}a:} Sean $N,M \normleq G$, con $N \subseteq M$, entonces $N/M$ es un subgrupo
		normal de $G/N$ y la aplicaci\'{o}n
		\begin{align*}
			\overline{f}: \faktor{G}{M} & \to \faktor{\faktor{G}{N}}{\faktor{M}{N}} \\
			gM & \mapsto [\overline{g}] := (gN)\cdot \faktor{M}{N}
		\end{align*}
		es una isomorf\'{i}a de grupos.
	\end{teoma}

	\dem En el estudio de los subgrupos normales del cociente $G/N$, vimos que $M/N$ es un subgrupo normal de $G/N$, que se corresponde
	con el subgrupo $M$ de $G$ (observar que $N \subseteq M$). Ahora, definimos
	\begin{align*}
		f: G & \to \faktor{\faktor{G}{N}}{\faktor{M}{N}} \\
		g & \mapsto [\overline{g}] := (gN) \faktor{M}{N}
	\end{align*}
	Primero, $f$ es homomorfismo de grupos.
	\begin{gather*}
		f(g_1g_2) = [\overline{g_1g_2}] = [\overline{g_1} \; \overline{g_2}] = [\overline{g_1}] [\overline{g_2}] = f(g_1)f(g_2)
	\end{gather*}
	Obviamente, $f$ es suprayectiva. Veamos el n\'{u}cleo
	\begin{align*}
		\ker f &= \{g \in G : f(g) = [\overline{e}]\} = \\
		       &= \{g \in G : [\overline{g}] = [\overline{e}]\} = \\
		       &= \{g \in G : \overline{g} = M/N\} = \\
		       &= \{g \in G : \exists x \in M \text{ t.q. } \overline{g} = \overline{x}\} = \\
		       &= \{g \in G : \exists x \in M \text{ t.q. } x^{-1}g \in N\} = \\
		       &= \{g \in G : g \in MN\} = MN \subseteq M (\land M \subseteq MN) \\
		       \therefore \ker f = M
	\end{align*}
	Por el primer teorema de isomorf\'{i}a
	\begin{align*}
		\overline{f}: G/M & \to (G/N)/(M/N) \\
		gM & \mapsto [\overline{g}] = (gN) M/N
	\end{align*}
	es una isomorf\'{i}a.
	
	\begin{teoma}
		Sea $M$ un grupo abeliano finitamente generado
		\begin{enumerate}
			\item Existen \'{u}nicos enteros $d_1,\ldots,d_r \geq 2$, factores invariantes, y un \'{u}nico $s \geq 0$ tales que
				$d_i$ divide a $d_{i+1}$ y
				$$
				M \cong \mathbb{Z}_{d_1} \times \ldots \times \mathbb{Z}_{d_r} \times \underbrace{\mathbb{Z} \times \ldots \times \mathbb{Z}}_{s}
				$$
			\item Existe una lista de enteros positivos $p_1^{e_1},\ldots,p_k^{e_k}$, divisores elementales, que es \'{u}nica salvo
				reordenaci\'{o}n, y un \'{u}nico $s \geq 0$ tales que $p_1,\ldots,p_k$ son primos, no necesariamente distintos, $e_1,
				\ldots,e_k$ son enteros positivos y
				$$
				M \cong \mathbb{Z}_{p_1^{e_1}} \times \ldots \times \mathbb{Z}_{p_k^{e_k}} \times
				\underbrace{\mathbb{Z} \times \ldots \times \mathbb{Z}}_{s}
				$$
		\end{enumerate}
	\end{teoma}

	\dem Si podemos hacer la descomposici\'{o}n de 2), pasar a 1) es f\'{a}cil. Si llamamos $S_1$ al conjunto de los $p_i^{e_i}$, entonces tendremos
	que $d_1 = \mathrm{mcm}(S_1)$, y definimos $S_2 = S_1 \setminus \{\text{divisores de } S_1\}$. De esta forma, podemos repetir el proceso con $d_2$,
	$d_3$, y as\'{i} hasta llegar a $S_{r+1} = \varnothing$, en este paso, tendr\'{a}s a los $d_1,\ldots,d_r$, que claramente se dividir\'{a}n
	en cadena. \\

	Ahora, demostremos la existencia de la factorizaci\'{o}n en divisores elementales. Procedemos por inducci\'{o}n sobre el n\'{u}mero de geneadoes de $M$.
	Nuestra estrategia ser\'{a} encontrar un conjunto generadoe $\{b_1,a_2,\ldots,a_l\}$ de modo que $\gen{b_1} \cap \gen{a_2,\ldots,a_l} = \{0\}$,
	pues esto nos asegura que $M \cong \gen{b_1} \times \gen{a_2,\ldots,a_l}$ y nos permite utilizar la hip\'{o}tesis de inducci\'{o}n en $\gen{a_2,\ldots,a_l}$,
	que tiene un generador menos. \\

	Sea $l$ el menor n\'{u}mero de generadores necesarios para generar $M$. Si $l = 0$, $M = \{0\}$, as\'{i} que aqu\'{i} $k=0,s=0$ y hemos acabado.
	Asumiremos que $l \geq 1$ y demostraremos el paso de inducci\'{o}n. Si $|M| < \infty$, basta descomponer $|M|$ en producto
	de potencias de primos. Si $|M| = \infty$, $M \cong \mathbb{Z}$ y tambi\'{e}n se tiene. Supuesto hasta $l-1$, sea
	$\{a_1,\ldots,a_l\}$ conjunto generador de $M \therefore$ (en notaci\'{o}n aditiva)
	\begin{gather*}
		M = \gen{a_1,\ldots,a_l} = \{\alpha_1a_1 + \ldots + \alpha_la_l : \alpha_i \in \mathbb{Z}\}
	\end{gather*}
	Si siempre que $\alpha_1a_1 + \ldots + \alpha_l a_l = 0$ entonces obligatoriamente $\alpha_i = 0$, y la aplicaci\'{o}n
	\begin{align*}
		f: \overbrace{\mathbb{Z} \times \ldots \times \mathbb{Z}}^l & \to M \\
		(\alpha_1,\ldots,\alpha_l) & \mapsto \alpha_1 a_1 + \ldots + \alpha_l a_l
	\end{align*}
	es isomorfismo de grupos $\therefore M \cong \underbrace{\mathbb{Z} \times \ldots \times \mathbb{Z}}_l$ y tomar\'{i}amos $k=0,s=l$. Veamos
	que $f$ es isomorfismo.
	\begin{gather*}
		f\big( (\alpha_1,\ldots,\alpha_l) + (\beta_1,\ldots,\beta_l)\big) = f(\alpha_1 + \beta_1,\ldots,\alpha_l + \beta_l) = \\
		= (\alpha_1 + \beta_1)a_1 + \ldots + (\alpha_l + \beta_l)a_l = \\
		= f(\alpha_1,\ldots,\alpha_l) + f(\beta_1,\ldots,\beta_l)
	\end{gather*}
	$f$ es suprayectiva, ya que $\{a_1,\ldots,a_l\}$ genera $M$.
	\begin{align*}
		\ker f &= \{(\alpha_1,\ldots,\alpha_l) \in \mathbb{Z} \times \ldots \times \mathbb{Z} : f(\alpha_1,\ldots,\alpha_l) = 0\} = \\
		       &= \{(\alpha_1,\ldots,\alpha_l) \in \mathbb{Z} \times \ldots \times \mathbb{Z} : \alpha_1 a_1 + \ldots + \alpha_l a_l = 0\} = \\
		       &= \{(0,\ldots,0)\} \text{(por hip\'{o}tesis)}
	\end{align*}
	Luego $f$ es isomorfismo.

	Si existiese una combinaci\'{o}n $\alpha_1a_1 + \ldots + \alpha_la_l = 0$ con alg\'{u}n $\alpha_i \neq 0$, vamos a tomar $d \geq 1$ m\'{i}nimo
	tal que existe esa combinaci\'{o}n lineal, donde $d$ aparece como un coeficiente. Cambiando el orden de los generadores, podemos
	asumir que $d = \alpha_1$. \\

	Si $\alpha_i$ no fiese m\'{u}ltiplo de $d$, entonces $\exists m,d'$
	\begin{gather*}
		\alpha_i = md + d', quad 1 \leq d' < d \\
		d (a_1 +ma_i) + \alpha_2a_2 + \ldots + d'a_i + \ldots + \alpha_la_l = \quad (*) \\
		= da_1 + \alpha_2a_2 + \ldots + (dm+d')a_i + \ldots + \alpha_la_l = \\
		= \alpha_1a_1 + \ldots + \alpha_ia_i + \ldots + \alpha_la_l = 0
	\end{gather*}
	Como $M = \gen{a_1,\ldots,a_i,\ldots,a_l} = \gen{a_1+ma_i,\ldots,a_i,\ldots,a_l} \therefore a_1+ma_i,\ldots,a_i,\ldots,a_l$ es un conjunto generador de $M$
	con $l$ elementos y (*) proporciona una expresi\'{o}n que contradice la mininmalidad de $d$. Luego
	\begin{gather*}
		\alpha_i = d\alpha'_i \; \forall \; i=1,\ldots,l, \text{ y cierto } \alpha'_1,\ldots,\alpha'_l
	\end{gather*}
	Definimos $b_1 := a_1 + \alpha'_2a_2 + \ldots + \alpha'_la_l$. Claramente
	\begin{gather*}
		db_1 = da_1 + \alpha_2a_2 + \ldots + \alpha_la_l = 0 \therefore db_1 = 0
	\end{gather*}
	y como $M = \gen{a_1,\ldots,a_l} = \gen{\overbrace{a_1 + \alpha'_2a_2 + \ldots + \alpha'_la_l}^{b_1},a_2,\ldots,a_l}$, entonces $\{b_1,a_2,\ldots,a_l\}$
	es conjunto generador de $M$ con $l$ elementos. Veamos que $\gen{b_1} \cap \gen{a_2,\ldots,a_l} = \{0\}$. Dado $b \in \gen{b_1} \cap \gen{a_2,\ldots,a_l}$,
	$b \in \gen{b_1} \implies b=\beta b_1$ para alg\'{u}n $\beta \in \mathbb{Z}$. Como $-b \in \gen{a_2,\ldots,a_l}, -b = \beta_2a_2 + \ldots +
	\beta_la_l$ para ciertos $\beta_2,\ldots,\beta_l \in \mathbb{Z}$.
	\begin{gather*}
		\underbrace{\beta b_1}_{b} + \underbrace{\beta_2a_2 + \ldots + \beta_la_l}_{-b} = 0
	\end{gather*}
	Si $\beta$ no es m\'{u}ltiplo de $d$, $\beta = md + d' \therefore 0=(md+d')b_1 + \beta_2a_2 + \ldots + \beta_la_l = d'b_1 + \beta_2a_2 + \ldots +
	\beta_la_l$. Lo que es imposible por minimalidad de $d$. Por ello, $\beta = md$ para alg\'{u}n $m \in \mathbb{Z} \therefore$
	\begin{gather*}
		b = \beta b_1 = m(db_1) = 0
	\end{gather*}
	$\therefore \gen{b_1} \cap \gen{a_2,\ldots,a_l} = \{0\}$. Adem\'{a}s,
	\begin{gather*}
		\gen{b_1} + \gen{a_2,\ldots,a_l} = M
	\end{gather*}
	ya que $\{b_1,a_2,\ldots,a_l\}$ genera a $M$. As\'{i}, $M \cong \gen{b_1} \times \gen{a_2,\ldots,a_l}$. \\

	Aplicando la hip\'{o}tesis de inducci\'{o}n a $\gen{a_2,\ldots,a_l}$ obtenemos que $\gen{a_2,\ldots,a_l}$ es isomorfo a un producto cartesiano
	(n\'{u}mero finito de factores) de grupos ciclos, por lo que $M$ tambi\'{e}n es isomorfo a un producto cartesiano de
	grupos c\'{i}clicos. Factorizando los factores que correspondan a grupos c\'{i}clicos finitos en producto cartesiano de
	grupos c\'{i}clicos de orden potencia de primo, tenemos una descomposici\'{o}n en divisores elementales (la segunda forma). \\

	Y con esto, hemos demostrado existencia de la descomposici\'{o}n. Falta ver la unicidad. Para ello, vamos a ver que podemos calcular las bases
	y exponentes de la descomposici\'{o}n solamente con $M$. Sea $f:M \to \mathbb{Z}_{p_1^{e_1}} \times \ldots \times \mathbb{Z}_{p_k^{e_k}} \times
	\overbrace{\mathbb{Z} \times \ldots \times \mathbb{Z}}^{l}$ un isomorfismo. Considero
	\begin{gather*}
		T:=\{a \in M:\exists \alpha \in \mathbb{Z}, \alpha \neq 0, \text{ t.q. } \alpha a=0\}
	\end{gather*}
	Vemos la imagen de $T$
	\begin{gather*}
		f(T) = \{(\overline{\alpha}_1,\ldots,\overline{\alpha}_k,\alpha_{k+1},\ldots,\alpha_{k+s}) \in \mathbb{Z}_{p_1^{e_1}} \times \ldots \times \mathbb{Z}_{p_k^{e_k}} \times \overbrace{\mathbb{Z} \times \ldots \times \mathbb{Z}}^{l} : f(\alpha a) = \alpha f(a) = 0\} = \\
		= \mathbb{Z}_{p_1^{e_1}} \times \ldots \times \mathbb{Z}_{p_k^{e_k}} \times \{0\} \times \ldots \times \{0\}
	\end{gather*}
	Conocemos $|T| = p_1^{e_1} \cdots p_k^{e_k}$, y conocemos los distintos primos que aparecen entre $p_1,\ldots,p_k$. Dado $p$ primo, observamos que
	\begin{gather*}
		|\{\overline{a} \in \mathbb{Z}_{p_i^{e_i}} : p^{e}\cdot\overline{a} = \overline{0}\}| = \mathrm{mcd}(p^{e},p_i^{e_i})
	\end{gather*}
	En efecto, $p^{e}\overline{a} = \overline{0} \iff p_i^{e_i}$ divide a $p^{e}a$, as\'{i}
	\begin{itemize}
		\item Si $p \neq p_i$, entonces $p_i^{e_i} | a \therefore \overline{a}=\overline{0} \therefore$ hay $1 = \mathrm{mcd}(p^{e},p_i^{e_i})$ elementos
			$\overline{a} \text{ t.q. } p^{e}\overline{a} = \overline{0}$
		\item Si $p=p_i$, y $e \geq e_i$, entonces hay $p_i^{e_i} = \mathrm{mcd}(p_e,p_i^{e_i})$ elementos $\overline{a} \text{ t.q. }
			p^{e}\overline{a} = \overline{0}$
		\item Si $p=p_i$ y $e < e_i$, entonces $p_i^{e_i}|p^{e}a \iff p_i^{e_i-e}|a \iff \overline{a} \in \gen{\overline{p^{e_i-e}}}$. Como
			$|\gen{\overline{p^{e_i-e}}}| = |\overline{p^{e_i-e}}|$, ya que es subgrupo de $\mathbb{Z}_{p_i^{e_i}}$ generado por $\overline{p^{e_i-e}}
			= p^{e_i-e} \cdot \overline{1}$. Luego $|p^{e_i-e}| = \frac{|\overline{1}|}{\mathrm{mcd}(|\overline{1}|,p^{e_i-e})} =
			\frac{p_i^{e_i}}{p^{e_i-e}}= p^{e} \therefore$ hay $p^{e} = \mathrm{mcd}(p^{e},p_i^{e_i}$ elementos $\overline{a} \text{ t.q. }
			p^{e}\overline{a} = \overline{0}$
	\end{itemize}
	Luego
	\begin{gather*}
		|\{(\overline{\alpha}_1,\ldots,\overline{\alpha}_k,0,\ldots,0) : p^{e}(\overline{\alpha}_1,\ldots,\overline{\alpha}_k,0,\ldots,0) =
		(\overline{0},\ldots,\overline{0},0,\ldots,0)\}| = \\
		= |\{(\overline{\alpha}_1,\ldots,\overline{\alpha}_k,0,\ldots,0): p^{e}\overline{\alpha}_1 = \overline{0},\ldots\}| =
		\prod_{i=1}^{k} \mathrm{mcd}(p^{e},p_i^{e_i})
	\end{gather*}
	Por el isomorfismo, $a \in T$ cumple que $p^{e}a = 0 \iff f(a) \in f(T)$ cumple que $p^{e}f(a) = 0 \therefore$
	\begin{gather*}
		|\{a \in T : p^{e}a = 0\}| = |\{(\overline{\alpha}_1,\ldots,\overline{\alpha}_k,0,\ldots,0) \in \mathbb{Z}_{p_1^{e_1}} \times \ldots
		\times \mathbb{Z}_{p_k^{e_k}} \times \{0\} \times \ldots \times \{0\}\}| = \prod_{i=1}^{k}  \mathrm{mcd}(p^{e},p_i^{e_i})
	\end{gather*}
	$\therefore$ podemos calcular estos n\'{u}meros a partir de $M \; \forall p \in \{p_1,\ldots,p_k\}$ y $e \geq 1$. Observamos
	\begin{gather*}
		\frac{\mathrm{mcd}(p^{e},p_1^{e_1}) \cdots \mathrm{mcd}(p^{e},p_k^{e_k})}{\mathrm{mcd}(p^{e-1},p_1^{e_1}) \cdots \mathrm{mcd}(p^{e-1},p_k^{e_k})} = p
		\quad (**)
	\end{gather*}
	\begin{itemize}
		\item Si $p_i \neq p$ entonces $\mathrm{mcd}(p^{e},p_i^{e_i}) = 1 = \mathrm{mcd}(p^{e-1},p_i^{e_i})$ y ``esos factores'' no contribuyen
		\item Si $p_i = p$ y $e_i < e$ entonces $\mathrm{mcd}(p^{e},p_i^{e_i}) = p_i^{e_i} = \mathrm{mcd}(p^{e-1},p_i^{e_i}) \therefore$ no contribuyen
		\item Si $p_i = p$ y $e_i \geq e$ entonces $\mathrm{mcd}(p^{e},p_I^{e_i}) = p^{e}$ y $\mathrm{mcd}(p^{e-1},p_i^{e_i}) = p^{e-1}$
	\end{itemize}

	As\'{i}, podemos ver que la descomposici\'{o}n se saca solo de $M$. Continuamos para la unicidad de $s$. Si $2M := \{2a : a \in M\}$, que es subgrupo
	de $M$, luego es normal. Entonces
	\begin{gather*}
		|\faktor{M}{T + 2M}| = 2^{s}
	\end{gather*}
	$\therefore$ podemos adivinar $s$. En efecto, escribimos
	\begin{align*}
		f: M & \to \mathbb{Z}_{p_1^{e_1}} \times \ldots \times \mathbb{Z}_{p_k^{e_k}} \times \overbrace{\mathbb{Z} \times \ldots \times \mathbb{Z}}^s \\
		a & \mapsto (f_1(a),\ldots,f_k(a),f_{k+1}(a),\ldots,f_{k+s}(a))
	\end{align*}
	Las aplicaciones $f_i:M \to \mathbb{Z}$, con $i=k+1,\ldots,k+s$ son homomorfismos de grupos por serlo $f \therefore$
	\begin{align*}
		g: M & \to \overbrace{\mathbb{Z}_2 \times \ldots \times \mathbb{Z}_2}^{s}  \\
		a & \mapsto (\overline{f_{k+1}(a)},\ldots,\overline{f_{k+s}(a)})
	\end{align*}

	Vamos a ver que $g$ es epimorfismo. Dado $(\overline{\alpha}_1,\ldots,\overline{\alpha}_s) \in \mathbb{Z}_2 \times \ldots \times \mathbb{Z}_2$ como $f$ es
	isomorfismo, existe $a \in M$ tal que $f(a) = (\overline{0},\ldots,\overline{0},\alpha_1,\ldots,\alpha_s) \therefore g(a) =
	(\overline{\alpha}_1,\ldots,\overline{\alpha}_s) \therefore$ $g$  es suprayectiva. \\

	Veamos que $\ker g = T + 2M$. Empezamos con el contenido $T + 2M \subseteq \ker g$. Dado $a \in T$, $f(a) = (\overline{\alpha}_1,\ldots,\overline{\alpha}_k,
	0,\ldots,0) \in f(T) \therefore g(a) = (\overline{0},\ldots,\overline{0}) \therefore a \in \ker g \therefore T \subseteq \ker g$. Dado $a \in 2M$,
	$a = b + b = 2b$ para alg\'{u}n $b \in M$.
	\begin{gather*}
		f(a) = f(b) + f(b) = 2f(b) = (2\overline{\alpha}_1,\ldots,2\overline{\alpha}_k,2\alpha_{k+1},\ldots,2\alpha_{k+s}) \\
		g(a) = (\overline{2\alpha_{k+1}},\ldots,\overline{2\alpha_{k+s}}) = (\overline{0},\ldots,\overline{0}) \therefore 2M \subseteq \ker g \\
		\therefore T + 2M \subseteq \ker g
	\end{gather*}

	Veamos el otro contenido. Dado $a \in \ker g$, $f_{k+1}(a) = 2\alpha_{k+1},\ldots,f_{k+s}(a)=2\alpha_{k+s}$ para ciertos $\alpha_{k+1},\ldots,\alpha_{k+s}
	\in \mathbb{Z}$. $f(a) = (\overline{\alpha_1},\ldots,\alpha_k,2\alpha_{k+1},\ldots,2\alpha_{k+s})$ para ciertos $\alpha_1,\ldots,\alpha_k,\ldots,
	\alpha_{k+s} \in \mathbb{Z}$. Como $(\overline{\alpha}_1,\ldots,\overline{\alpha}_k,0,\ldots,0) \in f(T)$ existe $b \in T$ tal que
	\begin{gather*}
		f(b) = (\overline{\alpha}_1,\ldots,\overline{\alpha}_k,0,\ldots,0)
	\end{gather*}
	Como $f$ es isomorfismo, existe $c \in M$ tal que
	\begin{gather*}
		f(c) = (\overline{0},\ldots,\overline{0},\alpha_{k+1},\ldots,\alpha_{k+s}) \\
		\therefore f(b + 2c) = (\overline{\alpha}_1,\ldots,\overline{\alpha}_k,2\alpha_{k+1},\ldots,2\alpha_{k+s}) = f(a)
	\end{gather*}
	Como $f$ es isomorfismo, esto implica $a = b + 2c \in T + 2M \therefore \ker g \subseteq T + 2M \therefore \ker g = T + 2M$. Por el primer teorema
	de isomorf\'{i}a
	\begin{gather*}
		\overbrace{\mathbb{Z}_2 \times \ldots \times \mathbb{Z}_2}^{s} = \im g \cong \faktor{M}{\ker g} = \faktor{M}{T+2M} \\
		\therefore |\faktor{M}{T+2M}| = |\overbrace{\mathbb{Z}_2 \times \ldots \times \mathbb{Z}_2}^{s}| = 2^{s} \\
		\therefore s = \log_2 |\faktor{M}{T+2M}|
	\end{gather*}
	es \'{u}nico.

	\begin{figure}[h]
		\centering
		\includegraphics[height=4cm,width=4cm]{square-107.png}
		\caption{Forma \'{o}ptima de meter 107 cuadrados en un cuadrado}
	\end{figure}
	
	\textit{Ejemplo:}
	Este ejemplo es para ver que los valores se sacan solo de $M$.Vamos a ver el caso de $7^3 7^2 7^3 7^5 7^6 11^2 11^7 11^3$.
	\begin{gather*}
		|T| = 7^{19} 11^{12}
	\end{gather*}
	Empezamos con $p=7$
	\begin{gather*}
		\begin{array}{c|c}
			e & \text{num de } 7^{e} \text{ que tengo} \\ \hline
			19 & 0 \\
			18 & 0 \\
			\vdots & \vdots \\
			7 & 0 \\
			6 & 1 \\
			5 & 1 \\
			4 & 0 \\
			3 & 2 \\
			2 & 1 \\
			1 & 0
		\end{array}
	\end{gather*}
	Para ver como sacar los n\'{u}meros de esa tabla, prueba a meter $p,e$ en (**). Puesto que cada $e_i$
	es menor o igual que el exponente de $p_i$ en la factorizaci\'{o}n de $|T|$, empezando desde ese exponente y bajando hasta 1 podemos
	encontrar los distintos $p_i^{e_i}$ gracias a (**). \\

	\begin{corol}
		Sea $M$ un grupo abeliano finito. Para todo $d$ divisor del orden de $M$, existe alg\'{u}n subgrupo de $M$ de ese orden.
	\end{corol}

	\dem Aqu\'{i} $M$ es un grupo abeliano finito
	\begin{gather*}
		\therefore M \cong \mathbb{Z}_{p_1^{e_1}} \times \ldots \times \mathbb{Z}_{p_k^{e_k}}
	\end{gather*}
	Dado $d$ divisor de $|M| = p_1^{e_1} \cdots p_k^{e_k}$,
	\begin{gather*}
		d = p_1^{e'_1} \cdots p_k^{e'_k} \quad \text{con } e'_i \leq e_i \; \forall i=1,\ldots,k
	\end{gather*}
	Como $\mathbb{Z}_{p_i^{e_i}}$ tiene un subgrupo $H_i$ de orden $p_i^{e'_i}$ entonces $H_1 \times \ldots \times H_k$ es subgrupo de $\mathbb{Z}_{p_1^{e_1}}
	\times \ldots \times \mathbb{Z}_{p_k^{e_k}}$ y $|H_1 \times \ldots \times H_k| = p_1^{e'_1} \cdots p_k^{e'_k} = d$. Por el isomorfismo
	$H_1 \times \ldots \times H_k$ se corresponde con un subgrupo de $M$ de orden $d$. \\

	\textit{Ejemplo: } $\gen{13} \leq (\mathbb{Z},+), \; \faktor{\mathbb{Z}}{\gen{13}}$ es grupo abeliano.
	\begin{align*}
		f: \mathbb{Z} & \to \mathbb{Z}_{13} \\
		a & \mapsto \overline{a}
	\end{align*}
	es epimorfismo de grupos, y $\ker f = \gen{13}$. Luego, por el primer teorema de isomorf\'{i}a $\faktor{\mathbb{Z}}{\gen{13}} \cong \mathbb{Z}_{13}$. \\

	\textit{Ejemplo: } Menos trivial. $\gen{(1,2),(3,4))} \leq \mathbb{Z} \times \mathbb{Z}$
	\begin{gather*}
		\faktor{\mathbb{Z}\times\mathbb{Z}}{\gen{(1,2),(3,4)}} \underset{(*)}{=} \gen{\overline{(1,0)},\overline{(0,1)}} \cong ??? \\
		(*) \overline{(a,b)} := (a,b) + \gen{(1,2),(3,4)}
	\end{gather*}
	En general, dado $\mathbb{Z}^{n} = \overbrace{\mathbb{Z} \times \ldots \times \mathbb{Z}}^{n}$ y $N = \gen{b_1,\ldots,b_m} \leq \mathbb{Z}^{n}$.
	A\~{n}adiendo como generador a $(0,\ldots,0)$, podemos asumir que $m \geq n$. Sea
	\begin{gather*}
		A = \left[ b_1 | \ldots | b_m \right] \in M_{n \times m}(\mathbb{Z})
	\end{gather*}
	Si existen $P \in M_n(\mathbb{Z})$ y $Q \in M_m(\mathbb{Z})$ invertibles, cuyos inversos tienen entradas en $\mathbb{Z}$, tal que
	\begin{gather*}
		PAQ = 
		\begin{bmatrix}
			\hat{d}_1 & & &0&\ldots&0 \\
			 & \ddots & &0&\ldots&0 \\
			& & \hat{d}_n &0&\ldots&0
		\end{bmatrix}
	\end{gather*}
	con $\hat{d}_i \geq 1 \; \forall d=1,\ldots,n$. Entonces definimos
	\begin{align*}
		f: \mathbb{Z}^{n} & \to \mathbb{Z}_{\hat{d}_1} \times \ldots \times \mathbb{Z}_{\hat{d}_n} \\
		v & \mapsto \overline{Pv}
	\end{align*}
	Resulta que $f$ es epimorfismo de grupos con n\'{u}cleo $N$. Por lo tanto, por el primer teorema de isomorf\'{i}a
	\begin{gather*}
		\faktor{\mathbb{Z}^{n}}{N} \cong \mathbb{Z}_{\hat{d}_1} \times \ldots \times \mathbb{Z}_{\hat{d}_n}
	\end{gather*}
	factorizando los $\hat{d}_i$ que sean mayores o iguales que 2, se alcanza la descomposici\'{o}n en divisores elementales. Veamos que $f$ es suprayectiva.
	Dado $w \in \mathbb{Z}_{\hat{d}_1} \times \ldots \times \mathbb{Z}_{\hat{d}_n}$.
	\begin{gather*}
		w =
		\begin{bmatrix}
			\overline{\alpha}_1 \\ \vdots \\ \overline{\alpha}_n
		\end{bmatrix}
	\end{gather*}
	para cierto $\alpha_1,\ldots,\alpha_n \in \mathbb{Z}$. Tomamos
	\begin{gather*}
		v := P^{-1}
		\begin{bmatrix}
			\alpha_1 \\ \vdots \\ \alpha_n
		\end{bmatrix} \in \mathbb{Z}^{n} \\
		f(v) = \overline{Pv} = \overline{PP^{-1}
		\begin{bmatrix}
			\alpha_1 \\ \vdots \\ \alpha_n
		\end{bmatrix}
		} = \begin{bmatrix}
			\overline{\alpha}_1 \\ \vdots \\ \overline{\alpha}_n
		\end{bmatrix} = w
	\end{gather*}
	Por lo que $f$ es suprayectiva. Veamos que $\ker f = N$.
	\begin{gather*}
		\ker f = \left\{v \in \mathbb{Z}^{n} : \overline{Pv} = \begin{bmatrix} \overline{0} \\ \vdots \\ \overline{0} \end{bmatrix}\right\} = \\
		\left\{v \in \mathbb{Z}^{n} : Pv = \begin{bmatrix} \hat{d}_1 \alpha_1 \\ \vdots \\ \hat{d}_n \alpha_n \end{bmatrix}
		\text{ para ciertos } \alpha_1,\ldots,\alpha_n \in \mathbb{Z}\right\} = \\
		\left\{ v \in \mathbb{Z}^{n} : Pv = D \begin{bmatrix} \alpha_1 & \vdots & \alpha_n \end{bmatrix}
		\text{ para ciertos } \alpha_1,\ldots,\alpha_n \in \mathbb{Z}\right\} = \\
		\left\{ v \in \mathbb{Z}^{n} : \cancel{P}v = \cancel{P}AQw \text{ para cierto } w \in \mathbb{m} \right\} =
		\left\{ v \in \mathbb{Z}^{n} : v = Aw \text{ para alg\'{u}n } w \in \mathbb{Z}^{m} \right\} = \\
		\{v \in \mathbb{Z}^{n} : v \text{ es una combinaci\'{o}n lineal con coeficientes enteros de las columnas de } A\} = \\
		= \gen{b_1,\ldots,b_m} = N
	\end{gather*}
	Luego, $f$ induce un isomorfismo
	\begin{gather*}
		\faktor{\mathbb{Z}^{n}}{N} \cong \mathbb{Z}_{\hat{d}_1} \times \ldots \times \mathbb{Z}_{\hat{d}_n}
	\end{gather*}
	Cabe observar que
	\begin{gather*}
		\mathbb{Z}_0 = \faktor{\mathbb{Z}}{\gen{0}} = \faktor{\mathbb{Z}}{\{0\}} \cong \mathbb{Z} \\
		\mathbb{Z}_1 = \faktor{\mathbb{Z}}{\gen{1}} = \faktor{\mathbb{Z}}{\mathbb{Z}} = \{\overline{0}\}
	\end{gather*} \\

	\textit{Ejemplo:} Describir $\faktor{\mathbb{Z}^2}{\gen{(2,2)^{T},(2,12)^{T}}}$ como producto de c\'{i}clicos. Aqu\'{i} $N = \gen{(2,2)^{T},(2,12)^{T}}$.
	Observamos que
	\begin{gather*}
		\overline{(1,1)} \neq \overline{(0,0)} \\
		2 \cdot \overline{(1,1)} = \overline{2\cdot(1,1)} = \overline{(0,0)}
	\end{gather*}
	Por lo que el orden de $\overline{(1,1)}$ es 2. Formamos la matriz $A$:
	\begin{gather*}
		A =
		\begin{bmatrix}
			2 & 2 \\
			2 & 12
		\end{bmatrix}
	\end{gather*}
	Vamos a ir haciendo transformaciones elementales cuyas inversas tienen solo entradas enteras, estas son:
	\begin{gather*}
		F_{ij} \quad \text{cambio de filas} \\
		F_i(\alpha), \alpha \in \{1,-1\} \quad \text{multiplicar la fila}
		F_{ij}(\alpha), \alpha \in \mathbb{Z} \quad \text{sumar m\'{u}ltiplo de otra fila}
	\end{gather*}
	Y sus correspondientes en columnas. As\'{i}:
	\begin{gather*}
		\begin{bmatrix}
			2 & 2 \\
			2 & 12
		\end{bmatrix} \overset{F_{21}(-1))}{\longrightarrow}
		\begin{bmatrix}
			2 & 2 \\
			0 & 10
		\end{bmatrix} \overset{C_{21}(-1)}{\longrightarrow}
		\begin{bmatrix}
			2 & 0 \\
			0 & 10
		\end{bmatrix}
	\end{gather*}
	As\'{i}, sabemos que
	\begin{gather*}
		\faktor{\mathbb{Z}^2}{\gen{(2,2)^{T},(2,12)^{T}}} \cong \mathbb{Z}_2 \times \mathbb{Z}_{10}
	\end{gather*} \\

	\textit{Otro ejemplo: } $\faktor{\mathbb{Z}^{3}}{N}$ donde
	\begin{gather*}
		N = \gen{(-18,-22,20),(-4,-4,4),(14,18,-16)}
	\end{gather*}
	\begin{gather*}
		A =
		\begin{bmatrix}
			-18 & -4 & 14 \\
			-22 & -4 & 18 \\
			20  &  4 & -16
		\end{bmatrix} \overset{C_{31}(1)}{\longrightarrow}
		\begin{bmatrix}
			-18 & -4 & -4 \\
			-22 & -4 & -4 \\
			20 & 4 & 4
		\end{bmatrix} \overset{C_{32}(-1)}{\longrightarrow} 
		\begin{bmatrix}
			-18 & -4 & 0 \\
			-22 & -4 & 0 \\
			20 & 4 & 0
		\end{bmatrix} \overset{F_{31}(1)}{\longrightarrow} \\
		\begin{bmatrix}
			2 & 0 & 0 \\
			-22 & -4 & 0 \\
			20 & 4 & 0
		\end{bmatrix} \overset{F_{32}(1)}{\longrightarrow}
		\begin{bmatrix}
			2 & 0 & 0 \\
			-22 & -4 & 0 \\
			-2 & 0 & 0
		\end{bmatrix} \overset{F_{31}(1)}{\longrightarrow} 
		\begin{bmatrix}
			2 & 0 & 0 \\
			-22 & -4 & 0 \\
			0 & 0 & 0
		\end{bmatrix} \overset{F_{21}(11)}{\longrightarrow} 
		\begin{bmatrix}
			2 & 0 & 0 \\
			0 & -4 & 0 \\
			0 & 0 & 0
		\end{bmatrix} \overset{F_{2}(-1)}{\longrightarrow} \\
		\begin{bmatrix}
			2 & 0 & 0 \\
			0 & 4 & 0 \\
			0 & 0 & 0
		\end{bmatrix}
	\end{gather*}
	Luego, nos queda que
	\begin{gather*}
		\faktor{\mathbb{Z}^{3}}{N} \cong \mathbb{Z}_2 \times \mathbb{Z}_4 \times \mathbb{Z}
	\end{gather*}

	\begin{teoma}
		Podemos clasificar grupos $U(n)$ de la siguiente forma:
		\begin{gather*}
			U(2^{n}) \cong C_2 \times C_{2^{n-2}}, \quad n \geq 2 \\
			U(p^{e}) \cong C_z, \quad p \text{ primo}, p \neq 2, z = |U(p^{e})| \\
			U(n_1n_2) \cong U(n_1) \times U(n_2), \quad \mathrm{mcd}(n_1,n_2) = 1
		\end{gather*}
	\end{teoma}

	\dem Sea $n \geq 3$. Entonces
	\begin{gather*}
		|U(2^{n})| = \varphi(2^{n}) = 2^{n-1} \therefore U(2^{n}) \cong C_{2^{e_1}} \times \ldots \times C_{2^{e_k}}
	\end{gather*}
	con $e_1 + \ldots + e_k = n-1$. Contamos el n\'{u}mero de elementos cuyo cuadrado es el neutro. En el lado de la derecha tenemos
	\begin{gather*}
		(a_1,\ldots,a_k)^2 = (e,\ldots,e) \iff (a_1^2,\ldots,a_k^2) = (e,\ldots,e) \iff a_1^2 = e, \ldots, a_k^2 = e
	\end{gather*}
	Cada $C_{2^{e_i}}$ tiene exactamente dos elementos cuyo cuadrado es el neutro $\therefore$ en $C_{2^{e_1}} \times \ldots \times C_{2^{e_k}}$ tiene
	exactamente $2^{k}$ elementos cuyo cuadrado es el neutro. En el lado de la izquierda, $\overline{a} \in U(2^{n})$
	\begin{gather*}
		\overline{a}^2 = \overline{1} \iff \overline{(a+1)(a-1)} = \overline{0} \iff 2^{n} | (a-1)(a+1)
	\end{gather*}
	Observamos que $\overline{a} \in U(2^{n}) \implies a$ es impar. As\'{i}, $2 | a-1$ y $2 | a+1$, pero si $d|a-1$ y $d|a+1$, entonces
	$d$ divide a $(a+1)-(a-1) = 2$, luego $\mathrm{mcd}(a+1,a-1) = 2$. Luego las posibilidades son que $2^{n} | a+1, 2^{n-1}|a+1,2^{n}|a-1,2^{n-1}|a-1$.
	Por ello,
	\begin{gather*}
		\overline{a} \equiv \overline{-1},\quad a=2^{n-1}-1 \implies \overline{a}^2 = 1,\quad \overline{a} = \overline{1},\quad a = 2^{n-1}+1 \implies
		\overline{a}^2 = \overline{1}
	\end{gather*}
	Luego, solo hay 4 distintos elementos cuyo cuadrado es el neutro. Por ello, sabemos que $k=2$, y as\'{i}, solo descomponemos el grupo en dos c\'{i}clicos.
	\begin{gather*}
		U(2^{n}) \cong C_{2^{e_1}} \times C_{2^{e_2}}
	\end{gather*}
	
	Basta comprobar que $\overline{a} = \overline{2^{n}-1}$ no es cuadrado en $U(2^{n})$ para concluir que $e_1=1$ o $e_2=1 \therefore$
	\begin{gather*}
		U(2^{n}) \cong C_2 \times C_{2^{n-2}}
	\end{gather*}

	\begin{teoma}
		\textit{Teorema de Cauchy:} Sea $G$ grupo finito y $p$ primo tal que $p$ divide a $|G|$. Entonces, existe alg\'{u}n elemento del grupo
		de orden $p$.
	\end{teoma}

	\dem Sea $X := \{(x_1,\ldots,x_p) \in \overbrace{G \times \ldots \times G}^{p} : x_1\cdots x_p = e\}$. La condici\'{o}n $x_1\cdots x_p = e$ equivale
	a $xp = (x_1\cdots x_{p-1})^{-1} \therefore |X| = |G|^{p-1} \therefore p$ divide a $|X|$. Nos fijamos en $S_X$, y en particular en
	\begin{align*}
		f: X & \to X \\
		(x_1,\ldots,x_p) & \mapsto (x_p,x_1,\ldots,x_{p-1})
	\end{align*}
	Claramente $f^{p} = \mathrm{Id}$. Adem\'{a}s, $f \neq \mathrm{Id}$. \textbf{FALTA DE HACER ALGO, CHEMA LO SUBIRA}. $\therefore$
	el orden de $f$ como elemento de $S_X$ es $p$ $\therefore$ $|\gen{f}|=p$. Declaramos que $\mathbb{X},\mathbb{Y} \in X$ son equivalantes si
	\begin{gather*}
		\mathbb{X} \equiv \mathbb{Y} \iff \exists i \geq 0 \text{ t.q. } \mathbb{Y}=f^{i}(\mathbb{X})
	\end{gather*}
	Esto es claramente una relaci\'{o}n de equivalencia. As\'{i}, vemos que
	\begin{gather*}
		[\mathbb{X}] = \{f^{i}(\mathbb{X}) : 0 \leq i \leq p-1\} = \{\mathbb{X}, f(\mathbb{X}), \ldots, f^{p-1}(\mathbb{X})\}
	\end{gather*}
	Sin embargo, $|[\mathbb{X}]| \leq p$. Para que $|[\mathbb{X}]| < p$, debe ocurrir que $\exists 0 \leq i < j \leq p-1 \text{ t.q. } f^{i}(\mathbb{X}) =
	f^{j}(\mathbb{X}) \therefore f^{j-i}(\mathbb{X}) = \mathbb{X}$ con $1 \leq j-i \leq p-1 \therefore f^{j-i} \neq \mathrm{Id}$ y $f^{j-i} \in \gen{f}$.
	Como $|\gen{f}| = p$ primo, $\gen{f^{j-i}} = \gen{f} \therefore$ todos los elementos $f^{k}(\mathbb{X}) \in \gen{f}$ cumplen que $f^{k}(\mathbb{X}) =
	\mathbb{X} \therefore \mathbb{X}$ tiene todas sus entradas iguales. Por ello, $[\mathbb{X}] = \{\mathbb{X}\}$ de un solo elementos $\therefore$
	podemos separar las distintas clases de equivalencia en dos tipos
	\begin{gather*}
		\begin{cases}
			[\mathbb{X}_1], \ldots, [\mathbb{X}] \to |[\mathbb{X}_i]| = 1 \\
			[\mathbb{Y}_1], \ldots, [\mathbb{Y}] \to |[\mathbb{Y}_i]| = p \\
		\end{cases}
	\end{gather*}
	Por ser clase de equivalencia
	\begin{gather*}
		X = [\mathbb{X}_1] \sqcup \ldots \sqcup [\mathbb{X}_n] \sqcup [\mathbb{Y}_1] \sqcup \ldots \sqcup [\mathbb{Y}_m]
	\end{gather*}
	Luego, m\'{o}dulo $p$
	\begin{gather*}
		0 \equiv |G|^{p-1} = |X| \equiv n + pm \equiv n \; (\mathrm{mod} \; p)
	\end{gather*}
	Luego el n\'{u}mero de clases de equivalenia de tama\~{n}o 1 es m\'{u}ltiplo de $p$. Como
	\begin{gather*}
		\begin{rcases}
			(e,\ldots,e) \in X \\
			[(e,\ldots,e)] = \{(e,\ldots,e)\}
		\end{rcases} \therefore n \geq 1
	\end{gather*}
	Como $n$ es m\'{u}ltiplo de $p$ y $n \geq 1$ entonces $\exists x \in X, \; \mathbb{X} \neq (e,\ldots,e) \text{ t.q. } [\mathbb{X}] = \{\mathbb{X}\}
	\therefore f(\mathbb{X}) = \mathbb{X} \therefore \mathbb{X} = (a,\ldots,a)$ para alg\'{u}n $a \in G, \; a \neq e$. Como $\mathbb{X} \in X$,
	$a^{p} = e$ y $a \neq e$, entonces $|a| = p$, y $G$ tiene alg\'{u}n elementos de orden $p$
	
	
	\subsection{Teorema de Jordan-H\"{o}lder}

	(Este tema no entra para examen, al menos este a\~{n}o :P) \\

	\begin{defin}
		Sea $G$ un grupo. Una \textit{serie subnormal} o simplemente \textit{serie} de $G$ es una cadena
		\begin{gather*}
			G = G_0 \supseteq G_1 \supseteq \ldots \supseteq G_r = \{e\}
		\end{gather*}
		de modo que $G_{i+1} \normleq G_i$ para cada $i=0,\ldots,r-1$. Los grupos $\faktor{G_i}{G_{i+1}}$ son los factores de la serie. Suprimiendo
		subgrupos repetidos se puede asumir que $G_i \neq G_{i+1}$, aunque no es necesario ni, en general, conveniente. El entero $r$, es decir,
		el n\'{u}mero de factores, se llama \textit{longitud} de la serie.
	\end{defin}

	\begin{defin}
		Una serie de $G$ se dice \textit{serie de composici\'{o}n} si todos sus factores son simples.
	\end{defin}

	\begin{defin}
		Un \textit{refinamiento de una serie} $G = G_0 \supseteq \ldots \supseteq G_r = \{e\}$ es otra serie $G = H_0 \supseteq \ldots
		\supseteq H_s = \{e\}$ de tal forma que $s \geq r$, y que existe una aplicaci\'{o}n inyectiva $i \mapsto i'$ tal que $G_i = H_{i'}$
	\end{defin}

	\begin{lema}
		Sea $G$ grupo, $G_0,\ldots,G_r$ una serie de $G$, y $H \leq G$. Se tiene que
		\begin{gather*}
			H = H \cap G_0 \supseteq H \cap G_1 \supseteq \ldots \supseteq H \cap G_r = \{e\}
		\end{gather*}
		es una serie de $H$.
	\end{lema}

	\dem Como $H$ y $G$ son subgrupos de $G$,
	\begin{gather*}
		H = H \cap G \geq G \cap G_1 \geq H \cap G_2 \geq \ldots \geq H \cap G_r = \{e\}
	\end{gather*}
	Como $G_{i+1} \normleq G_i$ entonces $\forall g \in H \cap G_i$
	\begin{gather*}
		g(H \cap G_{i+1})g^{-1} \subseteq H \cap G_{i+1} \therefore H \cap G_{i+1} \normleq H \cap G_i
	\end{gather*}
	Por ello, tenemos
	\begin{gather*}
		H = H \cap G \normleq H \cap G_1 \normleq \ldots \normleq H \cap G_r = \{e\}
	\end{gather*}
	por lo que es una serie de $H$.

	\begin{lema}
		Sean $Q,N,L$ subgrupos de un grupo $G$. Si $L$ es subgrupo normal de $Q$, y $gN = Ng$ para todo $g \in Q$, entonces $\faktor{QN}{LN}
		\cong \faktor{Q}{L(Q \cap N)}$
	\end{lema}

	\dem Aqu\'{i}, $Q \leq G$, $L \normleq Q$, $N \leq G$, y $gN = Ng \; \forall \; g \in Q$. Observamos que $QN = NQ$. Como $L \leq Q$,
	$LN = NL$. Adem\'{a}s, $QL = LQ$, ya que $L \normleq Q$. Veamos que $LN \normleq QN$. Como $e = e\cdot e \in LN \implies LN \neq \varnothing$
	\begin{gather*}
		LN \cdot LN = LLNN \subseteq LN
	\end{gather*}
	Dado $g \in L, x \in N$
	\begin{gather*}
		(gx)^{-1} = x^{-1} g^{-1} \in NL = LN
	\end{gather*}
	Por lo que es cerrado por productos e inversos, as\'{i} $LN \leq QN$. Veamos la normalidad. Dados $g \in Q, x \in N$
	\begin{gather*}
		(gx)LN(gx)^{-1} = gxLNx^{-1}g^{-1} \subseteq gNLNNg^{-1} = gLNNNg^{-1} \subseteq gLNg^{-1} = gLg^{-1} \cdot gNg^{-1} \subseteq LN
	\end{gather*}
	Este \'{u}ltimo contenido se debe a que $L \normleq Q, gN = Ng$. Por ello, $LN \normleq QN$. Ahora, veamos que $L(Q \cap N) \normleq Q$.
	\begin{gather*}
		\begin{rcases}
			L \normleq Q \\
			Q \cap N \leq Q
		\end{rcases} \implies L(Q \cap N) \leq Q \\
		g \in Q: \; g(Q \cap N)g^{-1} \underset{gN=Ng}{\subseteq} Q \cap N \therefore Q \cap N \normleq Q \therefore \\
		gL(Q \cap N)g^{-1} = gLg^{-1}g(Q cap N)g^{-1} \leq L(Q \cap N) \therefore L(Q \cap N) \normleq Q
	\end{gather*}
	Aplicamos el segundo teorema de isomorfia a $Q, LN \leq G$.
	\begin{gather*}
		\faktor{Q}{Q \cap LN} \underset{2TI}{\cong} \faktor{QLN}{LN} = \faktor{QN}{LN}
	\end{gather*}
	Ahora, veamos que $Q \cap LN = L(Q \cap N)$
	\begin{gather*}
		\supseteq / \quad \begin{rcases}
			L \subseteq Q,LN \\
			Q \cap N \subseteq Q,LN
		\end{rcases} \implies L(Q \cap N) \subseteq Q \cap LN \\
		\subseteq / \quad g \in Q \cap LN, \; g = g_1 x \; (g_1 \in L, x \in N) \therefore x = g_1^{-1} g \in Q \therefore x \in Q \cap N \therefore \\
		\therefore g \in L(Q \cap N) \therefore Q \cap LN \subseteq L(Q \cap N)
	\end{gather*}
	Y as\'{i}, $L(Q \cap N) = Q \cap LN \therefore \faktor{Q}{L(Q \cap N)} \cong \faktor{QN}{LN}$
	
	\begin{teoma}
		\textit{Teorema de Schreier:} Cualesquiera dos series de un mismo grupo tienen refinamientos equivalentes.
		Por fin, este teorema es del siglo 20, \textit{cine}.
	\end{teoma}

	\dem Sean
	\begin{gather*}
		G = G_0 \normleq G_1 \normleq \ldots \normleq G_m = \{e\} \\
		G = H_0 \normleq H_1 \normleq \ldots \normleq H_n = \{e\} \\
	\end{gather*}
	dos series de $G$. Partimos de la primera serie y empezamos a refinarla.
	
	
	

	\subsection{Ejercicios} % ================== EJERCICIOS =====================================================================================================

	\ej{12H2}{Da dos razones por las que el conjunto de los impares junto con la suma no es grupo.}

	\begin{enumerate}
		\item La suma de impares es par, por lo que no es una operaci\'{o}n interna.
		\item No existe elemento neutro.
	\end{enumerate}

	\ej{15H2}{Sea $a$ perteneciente a un grupo, tal que $a^{12} = e$. Expresa el inverso de los elementos $a$, $a^{6}$, $a^{8}$, y $a^{11}$
	usando la forma $a^{k}$ siendo $k$ un natural.}

	$$
	e = a^{12} = a \cdot a^{11} = a^{6} \cdot a^{6} = a^{8} \cdot a^{4} = a^{11} \cdot a
	$$

	\ej{16H2}{Muestra que el conjunto $\{5,15,25,35\}$ bajo la multiplicaci\'{o}n m\'{o}dulo 40 es grupo. Cu\'{a}l es la identidad? Hay
	alguna relaci\'{o}n con este grupo y $U(8)$}

	Vamos a hacer la tabla de Cayley a mano para el grupo.

	\begin{tabular}{l|llll}
		$\cdot$ & 5  & 15 & 25 & 35 \\ \hline
		5       & 25 & 35 & 5  & 15 \\
		15      & 35 & 25 & 15 & 5  \\
		25      & 5  & 15 & 25 & 35 \\
		35      & 15 & 5  & 35 & 25
	\end{tabular}

	Vemos que $e = 25$, y que $a^{-1} = a$ para todos los elementos. Calculando la tabla a mano de $U(8)$, vemos que son isomorfas.
	Este hecho se ve mejor si ponemos el orden de las filas y columnas como $\bar{3}, \bar{5}, \bar{1}, \bar{7}$.

	\ej{19H2}{Supongamos que $a$ y $b$ pertenecen a un grupo donde $a^{5} = b^{7} = e$. Escribe $a^{-2}b^{-4}$ y $(a^{2}b^{4})^{-2}$ sin
	usar exponentes negativos.}

	Usando las propiedades de los exponentes en grupos, podemos ver que:
	\begin{gather*}
		a^{-2}b^{-4} = ea^{-2}b^{-4}e = a^{5}a^{-2}b^{-4}b^{7} = a^{3}b^{3} \\
		(a^{2}b^{4})^{-2} = ((a^{2}b^{4})^{-1})^2 = (b^{-4}a^{-2})^2 = (b^{3}a^{3})^2
	\end{gather*}

	\ej{23H2}{Un profesor de álgebra abstracta tenía la intención de darle a una mecanógrafa una lista de nueve enteros que
	forman un grupo bajo la multiplicación módulo 91. En cambio, uno de los nueve enteros se omitió inadvertidamente, de modo
	que la lista apareció como 1, 9, 16, 22, 53, 74, 79, 81. ¿Qué entero fue omitido? (¡Esto realmente sucedió!).}

	Podr\'{i}amos hacer toda la tabla de Cayley hasta ver uno que falta, pero basta darse cuenta de que con hacer una sola
	fila o columna que no sea la del 1 (por ser el elemento neutro) lo tenemos. Esto se debe a que si un subgrupo finito tiene
	$n$ elementos, entonces el conjunto de multiplicar todos ellos por un elemento cualquiera del mismo no produce ning\'{u}n elemento
	repetido. Por ello, si nos fijamos en la fila del 9, por ejemplo, vemos que necesitamos el 29.
	
	\ej{29H2}{Prueba que para cualquier elementos $a,b \in G$ y $n \in \mathbb{N}$, se cumple que $(a^{-1}ba)^{n} = a^{-1}b^{n}a$}

	Este resultado es como multiplicamos matrices:
	\begin{gather*}
		(a^{-1}ba)^{n} = a^{-1}baa^{-1}ba \ldots a^{-1}baa^{-1}ba \underset{\text{asociativ.}}{=} a^{-1}b(aa^{-1})b(aa^{-1})\ldots (aa^{-1})ba = \\
		= a^{-1}bebe\ldots eba = a^{-1}b^{n}a
	\end{gather*}

	\ej{40H2}{Sup\'{o}n $F_1,F_2$ reflexiones distintas de un grupo dih\'{e}drico $D_n$ tal que $F_1F_2 = F_2F_1$. Prueba que $F_1F_2 = R_{180}$}
	Primero, sabemos que $n$ es par, o de lo contrario no podr\'{i}a existir $R_{180}$. Ahora, todas las reflexiones se pueden escribir
	como una reflexi\'{o}n cualquiera, llamemosla $F$, y una serie de rotaciones m\'{i}nimas, llam\'{e}moslas $R$. De esta forma,
	escribimos $F_1$ y $F_2$ como
	\begin{gather*}
		\begin{cases}
			F_1 = R^{a}F \\
			F_2 = R^{b}F
		\end{cases}
	\end{gather*}
	siendo $a,b \in \mathbb{N}, \; a \neq b, \; a \leq n \; b \leq n$. Ahora, sabemos que
	\begin{gather*}
		F_1F_2 = R^{a}HR^{b}H = R^{a}HHR^{-b} = R^{a}R^{-b} = R^{a-b} \\
		F_2F_1 = R^{b}HR^{a}H = R^{b}HHR^{-a} = R^{b}R^{-a} = R^{b-a}
	\end{gather*}

	Es decir, que $a-b \equiv b-a (\mathrm{mod} \; n) \implies 2(a-b) \equiv 0 (\mathrm{mod} \; n)$. Pero como tanto $a$ como $b$ son menores o iguales
	a $n$, no queda otra que $2(a-b) = \pm n$, es decir, $a-b = \pm n/2$, lo que significa que $F_1F_2$ es la rotaci\'{o}n $R_{180} = R^{n/2}$ \\

	\ej{1H3}{Para cada grupo en la siguiente lista, encuentra el orden del grupo y el orden de cada elemento en el grupo.
	¿Qué relación observas entre los órdenes de los elementos de un grupo y el orden del grupo?}
	\begin{gather*}
		\mathbb{Z}_{12}, \quad U(10), \quad U(12), \quad U(20), \quad D_{4}
	\end{gather*}

	\begin{enumerate}
		\item $(\mathbb{Z}_{12}, +)$
			\begin{center}
				\begin{tabular}{c|c|c|c|c|c|c|c|c|c|c|c|c}
					$\text{elem}$ & $\overline{1}$ & $\overline{2}$& $\overline{3}$ & $\overline{4}$ & $\overline{5}$ & $\overline{6}$ & $\overline{7}$ & $\overline{8}$ & $\overline{9}$ & $\overline{10}$ & $\overline{11}$ & $\overline{12}$ \\ \hline 
					\text{orden} & $12$ & $6$ & $4$ & $3$ & $12$ & $2$ & $12$ & $3$ & $4$ & $6$ & $12$ & $1$
				\end{tabular}
			\end{center}
			Usando el teorema 1.6.7, en notaci\'{o}n, es:
			$$
			|\overline{k}| = \frac{|\overline{1}|}{\mathrm{mcd}(|\overline{1}|,k)} = \frac{12}{\mathrm{mcd}(12,k)}
			$$
		\item $(U(10), \cdot)$
			\begin{center}
				\begin{tabular}{c|c|c|c|c}
					elem & $\overline{1}$ & $\overline{3}$ & $\overline{7}$ & $\overline{9}$ \\ \hline
					orden & 1 & 4 & 4 & 2
				\end{tabular}
			\end{center}
		\item $(U(12),\cdot)$
			\begin{center}
				\begin{tabular}{c|c|c|c|c}
					elem & $\overline{1}$ & $\overline{5}$ & $\overline{7}$ & $\overline{11}$ \\ \hline
	orden & 1 & 2 & 2 & 2 \\
				\end{tabular}
			\end{center}
		\item $(U(20),\cdot)$
			\begin{center}
				\begin{tabular}{c|c|c|c|c|c|c|c|c}
					elem & $\overline{1}$ & $\overline{3}$ & $\overline{7}$ & $\overline{9}$ & $\overline{11}$ & $\overline{13}$ & $\overline{17}$ & $\overline{19}$ \\ \hline
					orden & 1 & 4 & 4 & 2 & 2 & 3 & 4 & 2 \\
				\end{tabular}
			\end{center}
		\item $(D_4,\circ)$
			\begin{center}
				\begin{tabular}{c|c|c|c|c|c|c|c|c}
					elem & $R$ & $R^2$ & $R^{3}$ & Id & $H$ & $HR$ & $HR^2$ & $HR^{3}$ \\ \hline
	orden & 4 & 2 & 4 & 1 & 2 & 2 & 2 & 2 \\
				\end{tabular}
			\end{center}
	\end{enumerate}

	\ej{10H3}{Dado que $\{e, a, a^{2}, b, ab, a^{2}b\}$ es un grupo donde $|a| = 3$, $|b| = 2$ y $ba = a^{2}b$,
	determina cuál de los seis elementos es $aba^{2}$ y cuál es $a^{2}bab$.}
	\begin{gather*}
		aba^2 = a \cdot ba \cdot a = a \cdot a^2b \cdot a = eba = ba = a^2b \\
		a^2bab = a^2a^2bb = a^{4}b^2 = ab^2 = a
	\end{gather*} \\

	\ej{11H3}{Encuentra un subgrupo no abeliano de orden 6 en $D_6$}

	Probamos con $S = \{I,R^2,R^{4},H,R^2H,R^{4}H\}$.
	\begin{gather*}
		\begin{cases}
			R^{2a} \circ R^{2b} = R^{2(a+b)} \\
			R^{2a} \circ R^{2b}H = R^{2(a+b)}H \\
			R^{2a}H \circ R^{2b} = R^{2a} \circ R^{-2b} \circ H = R^{2(a-b)}H \\
			R^{2a}H \circ R^{2b}H = R^{2a} \circ R^{-2b} \circ H \circ H = R^{2(a-b)}
		\end{cases}
	\end{gather*}
	$\therefore$ es cerrado por productos $\therefore$ es cerrado por inversos. Por lo que si es un subgrupo no abeliano de orden 6. \\

	\ej{14H3}{Cuantos subgrupos de orden 4 tiene $D_4$?}

	Los \'{u}nicos son:
	\begin{gather*}
		\{I,R,R^2,R^{3}\} \\
		\{I,R^2,H,R^2H\}
	\end{gather*} \\

	\ej{23H3}{Si $a$ es un elemento de grupo, y $|a| = \infty$, prueba que $a^{n} \neq a^{m} \; \forall m \neq n$.}

	Supongamos que existen $n,m \in \mathbb{N}$ distintos tal que $a^{n} = a^{m}$. Sin perdida de generalidad, suponemos que $n > m$. Entonces
	\begin{gather*}
		a^{n} = a^{m} \iff a^{n-m} = e \implies \infty | (n-m)
	\end{gather*}
	lo cual es absurdo. \\

	\ej{7H4}{Encuentra un ejemplo de un grupo c\'{i}clico, cuyos subgrupos propios son todos c\'{i}clicos.}

	$U(12) - \{\overline{1},\overline{5},\overline{7},\overline{11}\}$ es un ejemplo, ya que
	\begin{gather*}
		|1| = 1 \quad |5| = |7| = |11| = 2
	\end{gather*}
	
	$therefore U(12)$ no es c\'{i}clico, pero por el teorema de Lagrange, $|H| \in \{1,2\}$. Si $|H| = 1,$ entonces $H = \{1\}$. Y si
	$|H| = 2$, como 2 es primo, $H$ es c\'{i}clico. \\

	\ej{9H4}{¿Cuántos subgrupos tiene $\mathbb{Z}_{20}$? Enumera un generador para cada uno de estos subgrupos.
		Supón que $G = \langle a \rangle$ y que $|a| = 20$. ¿Cuántos subgrupos tiene $G$?
		Enumera un generador para cada uno de estos subgrupos.}
	
	$Z_{20}$ es c\'{i}clico con $|Z_{20}| = 20$. Luego, para cada divisor de 20, hay un subgrupo, y esos son todos. En concreto,
	los grupos est\'{a}n generados por $\{a^{n/d} : \; d | n \; \forall d = 1,\ldots,n\}$.

	\begin{tabular}{c|c|c|c|c|c|c}
		orden & $1$ & $2$ & $4$ & $5$ & $10$ & $20$ \\ \hline
		subgrupo & $\langle \overline{20} \rangle$ & $\langle \overline{10} \rangle$ & $\langle \overline{5} \rangle$ & $\langle \overline{4} \rangle$ & 
		 $\langle \overline{2} \rangle$ & $\langle \overline{1} \rangle$ \\
	\end{tabular} \\

	\ej{10H4}{En $\mathbb{Z}_{24}$, nombra todos los generadores del subgrupo de orden 8. Sea $G = \langle a \rangle$ y sea $|a| = 24$. Nombra
	todos sus generadores del subgrupo del orden 8.}

	En $\mathbb{Z}_{24}$, $|\overline{3}| = 8$. Luego, tiene $\varphi(8)=4$, al ser tambi\'{e}n grupo c\'{i}clico. Estos son $\{3^{k}:k\in U(8)\}$ \\

	\ej{11H4}{Sea $G$ grupo con $a \in G$. Demuestra que $\langle a^{-1} \rangle = \langle a \rangle$.}
	Por definici\'{o}n
	\begin{gather*}
		\langle a^{-1} \rangle = \{(a^{-1})^{k} : k \in \mathbb{Z}\} = \{a^{k} : k \in \mathbb{Z}\} = \langle a \rangle
	\end{gather*} \\

	\ej{13H4}{En $ \mathbb{Z} $, encuentra un generador del subgrupo $ \langle 10 \rangle \cap \langle 12 \rangle $.
		En general, ¿cuál es un generador del subgrupo $ \langle m \rangle \cap \langle n \rangle $?
		Si $ a $ es un elemento del grupo de orden infinito, ¿cuál es un generador del subgrupo
		$ \langle a^m \rangle \cap \langle a^n \rangle $?}

	$(\mathbb{Z},+)$ es c\'{i}clico infinito, y $|1| = \infty$.
	\begin{gather*}
		\langle 10 \rangle = \{10^{k}:k \in \mathbb{Z}\} = 10\cdot k, k \in \mathbb{Z} = \text{m\'{u}ltiplos de 10}
		\langle 12 \rangle = \{12^{k}:k \in \mathbb{Z}\} = 12\cdot k, k \in \mathbb{Z} = \text{m\'{u}ltiplos de 12}
	\end{gather*}
	Si un elemento de $\mathbb{Z}$ tiene que ser m\'{u}ltiplo de 10 y 12, es m\'{u}ltiplo de $\mathrm{mcm}(10,12) = 60$, luego
	$\gen{10} \cap \gen{12} = \gen{60}$. En general, en $\mathbb{Z}$, $\gen{n} \cap \gen{m} = \gen{\mathrm{mcm}(n,m)}$.
	En general, con notaci\'{o}n multiplicativa en grupo infinito:
	\begin{gather*}
		\gen{a^{n}} \cap \gen{a^{m}} = \gen{a^{\mathrm{mcm}(n,m)}}
	\end{gather*} \\

	\ej{14H4}{Supón que un grupo cíclico $ G $ tiene exactamente tres subgrupos: el propio $ G $, el conjunto
		$ \{ e \} $, y un subgrupo de orden $ 7 $. ¿Cuál es el orden de $ G $ (es decir, $ |G| $)? ¿Qué puedes decir
		si $ 7 $ es reemplazado por $ p $, donde $ p $ es un número primo?}

	$G$ es c\'{i}clico, con subgrupos $\{e\}, G, H \text{ t.q. } |H|=7$. $|G|=49$, ya que si no fuese de la forma $7^{k}$, tendr\'{i}a
	alguno otro de orden que no sea potencia de 7, por el teorema de Lagrange, y tiene que tener solo 3,
	por lo que es 49, ya que sus divisores son $\{1,7,49\}$. \\

	\ej{18H4}{Sea $a$ generador del grupo c\'{i}clico $G$. Si $|G| = n$ y $\gen{a^{5}} = \gen{a^{10}}$, demuestra que $n$ es impar}

	Supongamos $n$ par:
	\begin{gather*}
		\begin{rcases}
			|\gen{a^{10}}| = \frac{n}{\mathrm{mcd}(n,10)} \\
			|\gen{a^{5}}| = \frac{n}{\mathrm{mcd}(n,5)} \\
		\end{rcases} \implies \mathrm{mcd}(n,10) = \mathrm{mcd}(n,5)
	\end{gather*}
	Y esa igualdad es entre un n\'{u}mero que es par si y solo si $n$ es par, y un n\'{u}mero que solo puede ser o 1 o 5. Luego hemos llegado
	a un absurdo. \\

	\ej{25H4}{Sea $H$ subgrupo de orden impar de $D_n$. Desmuestra que todo grupo de $H$ es c\'{i}clico.}

	$h \in H \implies |h| \; | \; |H| \implies |h|$ es impar. Si hay alguna reflexi\'{o}n, ser\'{i}a un elemento de orden 2, lo cual es
	absurdo, y $H$ est\'{a} compuesto solo de rotaciones, que es un grupo c\'{i}clico, y todo subgrupo de este ser\'{a} c\'{i}clico. \\

	\ej{26H4}{Da una cantidad infinita de ejemplos de grupos abelianos con exactamente 2 elementos de orden 3. Haz lo mismo
	para grupos no abelianos. Esos ejemplos se generalizan para exactamente $p-1$ elementos de orden $p$, con $p$ primo?}

	$\mathbb{Z}_{3k}$ es c\'{i}clico, luego es abeliano, y $3 | 3k \therefore$ tiene hay exactamente un subgrupo c\'{i}clico de orden 3, y este est\'{a} generado
	por exactamente $\varphi(3) = 2$ generadores, que van a ser todos los elementos de orden 3. En el caso de no abelianos,
	$D_{3k}$ tiene todas sus reflexiones de orden 2, y sus rotaciones son isomorfas a $\mathbb{Z}_{3k}$, que vuelve a ser el caso anterior.
	Esto se generaliza para $p-1$ elementos de orden $p$, ya que $p | pk$, y $\varphi(p) = p-1$, y se puede usar el argumento de
	$D_{pk}$ de misma forma.

	\ej{3H5}{Escribe estas permutaciones como ciclios disjuntos}

	\begin{enumerate}
		\item $(1235)(413) = (15)(234)$ \\
		\item $(12)(13)(23)(142) = (1423)$ \\
	\end{enumerate}

	\ej{5H5}{De que orden son las siguientes permutaciones?}

	\begin{enumerate}
		\item $|(124)(357869)| = \mathrm{mcm}(3,6) = 6$ \\
		\item $|(345)(245)| = |(25)(34)| = 2$
	\end{enumerate}

	\ej{9H5}{Escribe el ciclo como producto de ciclios disjuntos}
	\begin{gather*}
		[(14562)(2345)(136)(235)]^{10} = \\
		[(153)(46)]^{10} \underset{\text{disj.}}{=} (153)^{10} (46)^{10} = (153)
	\end{gather*}

	\ej{19H5}{Sean $\alpha,\beta \in S_n$. Prueba que $\alpha^{-5}\beta\alpha^{3}$ es impar si y solo si lo es $\beta$}
	\begin{gather*}
		\varepsilon(\alpha^{-5}\beta\alpha^{3}) = \varepsilon(\alpha^{-1})^{5} \varepsilon(\beta) \varepsilon(\alpha)^{3} =
		\varepsilon(\alpha^{-1}) \varepsilon(\beta) \varepsilon(\alpha) = \varepsilon(\beta)
	\end{gather*}

	\ej{22H5}{Cual es el m\'{i}nimo $n$ tal que $\exists \sigma \in S_n \text{ t.q. } |\sigma| = 30$}
	La descomposici\'{o}n en ciclos disjuntos de $\sigma$ ser\'{a}
	\begin{gather*}
		\sigma = \sigma_1 \cdots \sigma_l
	\end{gather*}
	Necesitamos $\mathrm{mcm}(|\sigma_1|,\ldots,|\sigma_l|) = 30$, es decir, hay alg\'{u}n ciclo de longitud m\'{u}ltiplo de 5, y otro de 3, y otro
	de 2. Necesitamos, como m\'{i}nimo, $S_{10}$, ya que ah\'{i} tenemos la cantidad m\'{i}nima de elementos para
	formar esos 3 ciclos disjuntos.

	Buscando en los grupos alternos, que son los que solo tienen permutaciones pares, tenemos que ir a $A_{12}$. Si cogemos
	la permutaci\'{o}n anterior y a\~{n}adimos el ciclo $(11,12)$, ya es una permutaci\'{o}n par. Y no hay ninguno m\'{a}s
	peque\~{n}o.
	
	\ej{23H5}{Cu\'{a}les son los posibles ordenes de los elementos de $S_6,A_6$? Y en $S_7$?}
	En $S_6$, solo puedes $\{1,2,3,4,5,6\}$, probando todas las combinaciones. Con $A_6$, perdemos el 6. Y con $A_7$, tenemos $1,2,3,4,5,7$.
	En general, para $S_n$, los ordenes posibles son $\{mcm(p):p \in \mathcal{P}(n)\}$

	\ej{25H5}{Sea $\beta = (1,3,5,7,9,8,6)(2,4,10)$. Que $n \in \mathbb{N}$ hacen que $\beta^{n} = \beta^{-5}$}

	 $|\beta| = 21$
	\begin{gather*}
		\beta^{n} = \beta^{-5} \iff \beta^{n+5} = e \iff 21 | n+5
	\end{gather*} \\

	\ej{28H5}{Sup\'{o}n que $H$ es un subgrupo de $S_n$ de orden impar. Prueba que $H \leq A_n$}

	\begin{gather*}
		\begin{rcases}
			H \leq S_n \\
			|H| \text{ impar}
		\end{rcases} \implies H \leq A_n
	\end{gather*}
	ya que, por el teorema de Lagrange, $\forall \; g \in H$, $|g|$ es impar, y al factorizar $g$ en ciclos disjuntos, ninguno puede
	ser par en orden, ya que $|g| = \mathrm{mcm}(|\sigma_1|,\ldots,|\sigma_r|)$. \\
	
	\ej{31H5}{Cu\'{a}ntos elementos de orden 2 hay en $A_8$ que sean de la forma en ciclos disjuntos $(a_1 \; a_2)(a_3 \; a_4)(a_5 \; a_6)(a_7 \; a_8)$}

	\begin{alignat*}{5}
		&(a_1 \; a_2) && \cdot (a_3 \; a_4) && \cdot (a_5 \; a_6) && \cdot (a_7 \; a_8) && \\
		& \frac{8 \cdot 7}{2} && \cdot \frac{6 \cdot 5}{2} && \cdot \frac{4 \cdot 3}{2} && \cdot \frac{2 \cdot 1}{2} && \cdot \frac{1}{4!}
	\end{alignat*}
	donde 4 primeras fracciones cuentan permutaciones dentro de las misma trasposiciones, y la fracci\'{o}n final tiene en cuenta
	las permutaciones de los ciclos en s\'{i}, ya que permutan entre s\'{i}, al ser disjuntos. \\

	\ej{41H5}{Sea $(a_1 a_2 a_3 a_4)$ y $(a_5 a_6)$ ciclos disjuntos en $S_{10}$. Demuestra que no existe ningún elemento
		$x$ en $S_{10}$ tal que $x^2 = (a_1 a_2 a_3 a_4)(a_5 a_6)$.}

	Notamos que $|x^2| = 4$, luego
	\begin{gather*}
		|x^2| = \frac{|x|}{\mathrm{mcd}(|x|,2)} \implies |x| = 4 \cdot \mathrm{mcd}(|x|,2)
	\end{gather*}
	y as\'{i}, $|x|$ puede ser 4 u 8, pero si es 4, entonces tenemos que $4=8$, luego $|x| = 8$. Por ello, $x = \sigma_1$ \'{o} $x = \sigma_1 \sigma_2$
	donde $|\sigma_1| = 8$ y $|\sigma_2| = 2$, disjuntos.
	\begin{gather*}
		x = \begin{cases}
			(a_1 \; \ldots \; a_8)(a_9 \; a_{10}) \\
			(a_1 \; \ldots \; a_8)
		\end{cases} \implies x^2 = 
		\begin{cases}
			(a_1 \; a_3 \; a_5 \; a_7)(a_2 \; a_4 \; a_6 \; a_8) \\
			(a_1 \; a_3 \; a_5 \; a_7)(a_2 \; a_4 \; a_6 \; a_8)
		\end{cases} \neq (a_1 \; a_2 \; a_3 \; a_4)(a_5 \; a_6)
	\end{gather*} \\

	\ej{43H5}{Sea $G$ un grupo de permutaciones sobre un conjunto $X$. Sea $a \in X$ y definamos $\mathrm{stab}(a) = \{\alpha \in G \mid \alpha(a) = a\}$.
		Llamamos a $\mathrm{stab}(a)$ el estabilizador de $a$ en $G$ (ya que consiste en todos los elementos de
		$G$ que dejan fijo a $a$). Demuestra que $\mathrm{stab}(a)$ es un subgrupo de $G$. (Este subgrupo fue
		introducido por Galois en 1832).}
	
	\begin{itemize}
		\item $\mathrm{stab}(a) \neq \varnothing$ ya que $\mathrm{Id} \in \mathrm{stab}(a)$
		\item $f,g \in \mathrm{stab}(a) \implies fg \in \mathrm{stab}(a)$ ya que $fg(a) = f(g(a)) = f(a) = a$
		\item $f \in \mathrm{stab}(a) \implies f^{-1} \in \mathrm{stab}(a)$ ya que $a = f^{-1}f(a) = f^{-1}(a)$
	\end{itemize}

	\ej{45H5}{Para $n \geq 3$, sea $H = \{\beta \in S_n : \beta(1) \in \{1,2\} \land \beta(2) \in \{1,2\}\}$. Demuestra que $H \leq S_n$, y determina
	$|H|$}

	\begin{itemize}
		\item $H \neq \varnothing$, ya que $\mathrm{Id} \in H$ \\
		\item $\beta_1,\beta_2 \in H$
			\begin{gather*}
				\beta_1\beta_2(1) = \beta_1(1 \text{ \'{o} } 2) = 1 \text{ \'{o} } 2 \\
				\beta_1\beta_2(2) = \beta_1(1 \text{ \'{o} } 2) = 1 \text{ \'{o} } 2
			\end{gather*}
			por lo que $\beta_1\beta_2 \in H$
	\end{itemize}
	y as\'{i}, $H \leq S_n$. Para determinar $|H|$, podemos ver que $H = H_1 \cup H_2$, donde
	\begin{gather*}
		H_1 = \{\beta \in S_n : \beta(1) = 1, \beta(2) = 2\} \cong S_{n-2} \\
		H_2 = \{\beta \in S_n : \beta(1) = 2, \beta(2) = 1\} \text{ que no es grupo, pero } |H_2| = |S_{n-2}|
	\end{gather*}
	luego $|H| = 2 \cdot |S_{n-2}| = 2(n-2)!$ \\

	\ej{48H5}{En $S_3$, encuentra elementos $\alpha,\beta$ tal que $|\alpha|=2,|\beta|=2,|\alpha\beta|=3$}

	Cualquier $(a \; b)(b \; c) = (a \; b \; c)$ \\

	\ej{51H5}{Prueba que $S_n$ es no abeliano para $n \geq 3$, y que $A_n$ es no abeliano para $n \geq 4$}

	Para $S_n$, vemos que $(12)(13) \neq (13)(12)$, y para $A_n$, $(124)$ no permuta con $(123)$ \\

	\ej{58H5}{Viendo $S_5$ como el grupo de los movimientos de un pent\'{a}gono regular con sus v\'{e}rtices marcados del 1 al 5, qu\'{e} simetr\'{i}a
	corresponde a $(14253)$, y cual a $(25)(34)$}

	A $(14253)$ le corresponde $R^{3}$, y $(25)(34)$ es la reflexi\'{o}n $H$. \\

	\ej{8H9}{Viendo $\gen{12}$ y $\gen{3}$ como subgrupos de $\mathbb{Z}$, prueba que $\faktor{\gen{3}}{\gen{12}}$ es isomorfo a $\mathbb{Z}_4$.
	De la misma forma, demuestra que $\faktor{\gen{8}}{\gen{48}} \cong \mathbb{Z}_6$. Generaliza para enteros $k,n$}

	Es claro que $\gen{12} \subseteq \gen{3}$. Adem\'{a}s, $\gen{12} \normleq \gen{3}$, ya que $\gen{12} \normleq \mathbb{Z}$ y $\gen{12} \leq \gen{3} \therefore$
	\begin{gather*}
		\faktor{\gen{3}}{\gen{12}} = \{\overline{0}, \pm\overline{3}, \pm\overline{6}, \pm\overline{9}, \pm\overline{12},\ldots\}
	\end{gather*}
	Sin embargo, este grupo colapsa hasta tener 4 elementos. Por ejemplo, $\pm\overline{12} = \overline{0}$, ya que $12 + \gen{12} = 0 + \gen{12}$.
	Esto pasa si y solo si $12 + (-0) \in \gen{12}$, que es el caso. As\'{i}, se quedan solo los positivos, ya que a base de sumar 12 a cualquier
	elemento $\overline{a}$, acaba llegando a uno positivo. Adem\'{a}s, cualquiera por encima de 12 se le puede restar 12 hasta llegar al grupo
	\begin{gather*}
		\{\overline{0},\overline{3},\overline{6},\overline{9}\}
	\end{gather*}
	y adem\'{a}s es c\'{i}clico, generado por $\gen{\overline{3}}$. As\'{i} $\faktor{\gen{3}}{\gen{12}} \cong \mathbb{Z}_4$. De igual forma,
	$\faktor{\gen{8}}{\gen{48}} \cong \mathbb{Z}_6$. En general, $\faktor{\gen{a}}{\gen{ak}} \cong \mathbb{Z}_k$, ya que $\gen{a} \leq \mathbb{Z}$,
	$\gen{ak} \normleq \mathbb{Z}$, y $\gen{ak} \subseteq \gen{a}$, por lo que $\gen{ak} \normleq \gen{a}$. Con el razonamiento anterior podemos
	llegar a colapsar las repeticiones en el grupo hasta quedarnos con $k$ de ellas distinas, y ser\'{a} un grupo generado por $\gen{\overline{a}}$. \\

	\ej{19H9}{Si $H$ es subgrupo normal de $G$ y $|H|=2$, prueba que $H \subseteq Z(G)$. Determina todos los subgrupos normales de $D_n$ de orden 2.}

	Sea $H = \{e,a\}$, $g \in G \implies gag^{-1} \in H$.

	\begin{gather*}
		\begin{cases}
			gag^{-1} = e \implies a = e \quad \bot \\
			gag^{-1} = a \implies ga = ag \; \forall g \in G \implies H \subseteq Z(G)
		\end{cases}
	\end{gather*}

	Los \'{u}nicos subgrupos de orden 2 de $D_n$ son los que, siendo $n$ par, $\gen{R^{n/2}}$. \\

	\ej{25H9}{Sea $G = U(32)$ y $H = \{1,15\}$. El grupo $\faktor{G}{H}$ es isomorfo a uno de los siguientes: $\mathbb{Z}_8,\mathbb{Z}_4 \times \mathbb{Z}_2,
	\mathbb{Z}_2 \times \mathbb{Z}_2 \times \mathbb{Z}_2$. Determina cual es por eliminaci\'{o}n.}

	\begin{gather*}
		U(32) = \{\overline{1},\overline{3},\overline{5},\ldots,\overline{31}\} \\
		\faktor{U(32)}{H} = \{[\overline{1}],[\overline{3}],[\overline{5}],\ldots,[\overline{31}]\}
	\end{gather*}

	Notamos que $\overline{15}^2 = 1$, luego $H \leq U(32)$, y como $U(32)$ es abeliano, $H$ es normal. Veamos repeticiones. Ya que encontrar inversos
	multiplicativos es tarea dif\'{i}cil, podemos ver que elemento es cada uno a base de calcular sus elementos, de la siguiente forma:
	\begin{alignat*}{3}
		& [\overline{3}] &&= \overline{3} \cdot H = \{\overline{3} \cdot 1, \overline{3} \cdot 15\} = \{\overline{3},\overline{13}\} &&= [\overline{13}] \\
		& [\overline{5}] &&= \overline{5} \cdot H = \{\overline{5} \cdot 1, \overline{5} \cdot 15\} = \{\overline{5},\overline{11}\} &&= [\overline{11}] \\
		& [\overline{7}] &&= \ldots &&= [\overline{9}] \\
		& [\overline{17}] &&= \ldots &&= [\overline{31}] \\
		& [\overline{19}] &&= \ldots &&= [\overline{29}] \\
		& [\overline{21}] &&= \ldots &&= [\overline{27}] \\
		& [\overline{23}] &&= \ldots &&= [\overline{25}] \\
	\end{alignat*}

	As\'{i}, nos queda el grupo
	\begin{gather*}
		\faktor{U(32)}{H} = \{[\overline{1}],[\overline{3}],[\overline{5}],[\overline{7}],[\overline{17}],[\overline{19}],[\overline{21}],[\overline{23}]\}
	\end{gather*}

	Si nos ponemos a calcular ordenes de elementos:
	\begin{gather*}
		[\overline{3}]^2 = [\overline{9}] \neq [\overline{1}] \\
		[\overline{3}]^{3} \neq [\overline{1}] \\
		[\overline{3}]^{4} \neq [\overline{1}]
	\end{gather*}

	Por lo que $|[\overline{3}]| \geq 5$, por lo que debe ser 8, y as\'{i}, tenemos que
	\begin{gather*}
		\faktor{U(32)}{H} \cong \mathbb{Z}_8
	\end{gather*}
	
	\ej{27H9}{Sea $G = U(16)$, $H = \{\overline{1},\overline{15}\}$ y $K = \{\overline{1},\overline{9}\}$. Son $H$ y $K$ isomorfos? Son $\faktor{G}{H}$
	y $\faktor{G}{K}$ isomorfos?}

	Primero, empezemos viendo que $H$ y $K$ son subgrupos normales de $G$.
	\begin{itemize}
		\item $H \normleq G$?
			\begin{gather*}
				\overline{15}^2 = \overline{1} \therefore H \normleq G
			\end{gather*}
		\item $K \normleq G$?
			\begin{gather*}
				\overline{9}^2 = \overline{1} \therefore K \normleq G
			\end{gather*}
	\end{itemize}
	Con esto, sabemos que son subgrupos normales. Adem\'{a}s, podemos afirmar que $G \not\cong \mathbb{Z}_8$, ya que hemos encontrado dos subgrupos
	distintos de orden 2, por lo que no puede ser c\'{i}clico. Por otra parte, $H \cong \mathbb{Z}_2$ y $K \cong \mathbb{Z}_2$, por lo que $H \cong K$. \\

	Ahora, veamos el caso de los cocientes. Vamos a ver las repeticiones de $\faktor{G}{H}$:
	\begin{gather*}
		\begin{cases}
			[\overline{1}] = \{\overline{1},\overline{15}\} = [\overline{15}] \\
			[\overline{3}] = \{\overline{3},\overline{13}\} = [\overline{13}] \\
			[\overline{5}] = \{\overline{5},\overline{11}\} = [\overline{11}] \\
			[\overline{7}] = \{\overline{7},\overline{9}\} = [\overline{9}] \\
		\end{cases} \implies \faktor{G}{H} = \{[\overline{1}],[\overline{3}],[\overline{5}],[\overline{7}]\}
	\end{gather*}
	Vamos a calcular \'{o}rdenes, para ver que tipo de subgrupo es $\faktor{G}{H}$.
	\begin{gather*}
		[\overline{3}], \quad [\overline{3}]^2 = [\overline{9}] = [\overline{7}] \implies |[\overline{3}]| > 2 \implies |[\overline{3}]| = 4
	\end{gather*}
	Luego $\faktor{G}{H}$ es un grupo c\'{i}clico de orden 4, al tener un elemento de orden 4, y ser \'{e}l mismo de orden 4. $\faktor{G}{H} \cong C_4$.
	Vamos con $\faktor{G}{K}$
	\begin{gather*}
		\begin{cases}
			[\overline{1}] = \{\overline{1},\overline{9}\} = [\overline{9}] \\
			[\overline{3}] = \{\overline{3},\overline{11}\} = [\overline{11}] \\
			[\overline{5}] = \{\overline{5},\overline{13}\} = [\overline{13}] \\
			[\overline{7}] = \{\overline{7},\overline{15}\} = [\overline{15}] \\
		\end{cases} \implies \faktor{G}{K} = \{[\overline{1}],[\overline{3}],[\overline{5}],[\overline{7}]\}
	\end{gather*}
	Veamos los \'{o}rdenes
	\begin{gather*}
		\begin{rcases}
			[\overline{3}]^2 = [\overline{1}] \\
			[\overline{5}]^2 = [\overline{25}] = [\overline{9}] = [\overline{1}] \\
			[\overline{7}]^2 = [\overline{49}] = [\overline{1}]
		\end{rcases} \implies \faktor{G}{K} \cong \mathbb{Z}_2 \times \mathbb{Z}_2
	\end{gather*}
	Con solo ver que $|[\overline{3}]| = |[\overline{5}]| = 1$, ya sabemos que no es c\'{i}clico, ya que para $C_n$ y $d|n$, $\exists \varphi(d)$ de haber
	1 elemento de orden 2, pero vemos que hay 2. As\'{i}, aunque $H \cong K$, no se da el caso de $\faktor{G}{H} \cong \faktor{G}{K}$. \\

	\ej{38H9}{Prueba que para cada $n \in \mathbb{N}$, $\faktor{\mathbb{Q}}{\mathbb{Z}}$ tiene un elemento de orden $n$}
	Miramos el elemento $[\frac{1}{n}]$:
	\begin{gather*}
		\underbrace{[\frac{1}{n}] + \ldots + [\frac{1}{n}]}_{k} = [\frac{k}{n}] = [0] \iff n | k
	\end{gather*}
	luego:
	\begin{gather*}
		|[\frac{1}{n}]| = n
	\end{gather*} \\

	\ej{39H9}{Sea $H$ subgrupo de $G$ con la propiedad de que para todo $a,b \in G$, se cumple que $aHbH = abH$. Prueba que $H \normleq G$}
	Sea $H \leq G \text{ t.q. } \forall \; a,b \in G, \; aHbH = abH$. Entonces, multiplicando por $a^{-1}$ a ambos lados, sacamos que
	\begin{gather*}
		HbH = bH \implies Hb \subseteq bH \implies H \subseteq bHb^{-1} \quad \forall \; b
	\end{gather*}
	Si ahora cogemos $b^{-1}$
	\begin{gather*}
		H \subseteq b^{-1}Hb \implies bHb^{-1} \subseteq H \implies H = bHb^{-1} \quad \forall \; b
	\end{gather*}
	demostrado por doble contenido. \\

	\ej{10H10}{Sea $G$ subgrupo de alg\'{u}n di\'{e}drico. Definimos para cada $x \in G$: $\phi(x) =
		\{1, \text{ si es rotaci\'{o}n}, -1, \text{ si es reflexi\'{o}n}\}$. Prueba que $\phi$ es homomirfismo multiplicativo de $G$}

	Vamos por casos:
	\begin{itemize}
		\item $x$ rotaci\'{o}n, $y$ rotaci\'{o}n:
			\begin{gather*}
				\phi(xy) = 1 = 1 \cdot 1 = \phi(x) \phi(y)
			\end{gather*}
		\item $x$ rotaci\'{o}n, $y$ reflexi\'{o}n
			\begin{gather*}
				\phi(xy) = -1 = 1 \cdot -1 = \phi(x) \phi(y)
			\end{gather*}
		\item $x$ reflexi\'{o}n, $y$ reflexi\'{o}n
			\begin{gather*}
				\phi(xy) = 1 = -1 \cdot -1 = \phi(x) \phi(y)
			\end{gather*}
	\end{itemize}
	El kernel de $\phi$ es $\ker \phi = \{x \in G : x \text{ es rotaci\'{o}n}\}$. As\'{i}:
	\begin{gather*}
		\phi^{-1}(-1) = \{x \in G : x = HR^{i}\} \\
		\phi^{-1}(1) = \{x \in G : x = R^{i}\}
	\end{gather*}
	As\'{i}, o bien $G$ no tiene reflexiones, es decir, $-1 \notin \phi(G)$, es decir, $G \subseteq \gen{R}$, o bien $G$ tiene reflexiones
	y tiene $|\phi^{-1}(-1)| = |\ker \phi| = |\phi^{-1}(1)|$, es decir, hay tantas reflexiones como rotaciones. \\

	\textit{Nota:} En general, si $f:G \to H$ homomorfismo de grupos y $h \in f(G)$. Intentemos ver que es $f^{-1}(h)$. Como $h \in f(G), \exists g_0 \text{ t.q. }
	\phi(g_0) = h$. Dado cualquier elemento de la preimagen $g \in f^{-1}(h)$, signfifica que $f(g) = h = f(g_0)$
	\begin{gather*}
		\left( f(g_0) \right)^{-1} f(g) = e \implies f(g_0^{-1}) f(g) = e \implies f(g_0^{-1}g) = e \implies g_0^{-1}g \in \ker f \implies \\
		g \in g_0 \ker f \implies f^{-1}(h) \subseteq g_) \ker f \subseteq f^{-1}(h) \therefore \\
		f^{-1}(h) = g_0 \ker f
	\end{gather*}
	donde $f(g_0) = h$. Vamos a usar esto para el 10H10, que es el anterior. Demuestra el enunaciado de este problema. Podemos sacar en claro, como trucazo,
	este resultado

	\begin{tcolorbox}[title=Truco para las preimagenes]
		Para un $G \leq D_n$ para alg\'{u}n $n \in \mathbb{N}$, y $f$ homomorfismo:
		\begin{gather*}
			|f^{-1}(h)| =
			\begin{cases}
				0, & h \notin f(G) \\
				|\ker f| & h \in f(G)
			\end{cases}
		\end{gather*}
	\end{tcolorbox}

	\ej{11H10}{Prueba que $\faktor{\mathbb{Z} \times \mathbb{Z}}{\gen{(a,0),(0,b)}} \cong \mathbb{Z}_a \times \mathbb{Z}_b$, con $a,b \geq 1$}

	Definimos
	\begin{align*}
		f: \mathbb{Z} \times \mathbb{Z} & \to \mathbb{Z}_a \times \mathbb{Z}_b \\
		(x,y) & \mapsto (\overline{x},\overline{y})
	\end{align*}
	Claramente, esto es un homomorfismo:
	\begin{gather*}
		f \left( (x_1,y_1) + (x_2,y_2) \right) = f \left( x_1+x_2,y_1+y_2 \right) = (\overline{x_1 + x_2},\overline{y_1+y_2}) = \\
		= (\overline{x_1} + \overline{x_2}, \overline{y_1} + \overline{y_2}) = (\overline{x_1},\overline{y_1}) + (\overline{x_2},\overline{y_2}) =
		f(x_1,y_1) + f(x_2,y_2)
	\end{gather*}
	Claramente, tambi\'{e}n es epimorfismo. Vamos a calcular el n\'{u}cleo
	\begin{align*}
		\ker f &= \{(x,y) \in \mathbb{Z} \times \mathbb{Z} : (\overline{x},\overline{y}) = (\overline{0},\overline{0})\} = \\
		       &= \{(x,y) \in \mathbb{Z} \times \mathbb{Z} : a|x, \; b|y\} = \\
		       &= \{(c_1a,c_2b) \in \mathbb{Z} \times \mathbb{Z} : c_1,c_2 \in \mathbb{Z}\} = \\
		       &= \{(c_1a,0) + (0,c_2b) : c_1,c_2 \in \mathbb{Z}\} = \\
		       &= \{c_1(1,0) + c_2(0,1) : c_1,c_2 \in \mathbb{Z}\} = \\
		       &= \gen{(a,0),(0,b)}
	\end{align*}
	As\'{i}, por el primer teorema de isomorf\'{i}a, inducimos el isomorfismo $\faktor{\mathbb{Z} \times \mathbb{Z}}{\gen{(a,0),(0,b)}} \cong \mathbb{Z}_a
	\times \mathbb{Z}_b$

	\ej{13H10}{Demuestra que $\faktor{A \times B}{A \times \{e\}} \cong B$}
	Definimos nuestra aplicaci\'{o}n
	\begin{align*}
		f: A \times B & \to B \\
		(a,b) & \mapsto b
	\end{align*}
	Que es homomorfismo de grupos
	\begin{gather*}
		f \left( (a_1,b_1) \cdot (a_2,b_2) \right) = f \left( (a_1a_2,b_1b_2) \right) = b_1b_2 = f(a_1,b_1) f(a_2,b_2)
	\end{gather*}
	Es epimorfismo. El n\'{u}cleo es
	\begin{gather*}
		\ker f = \{(a,b) \in A \times B : b=e\} = A \times \{e\}
	\end{gather*}
	Por el primer teorema de isomorfia, inducimos el isomorfismo deseado. \\

	\textit{Ejemplo:} Vamos a tratar cosas sobre conmutatividad y grupos derivados. \\

	Sea $G$ no abeliano. Definimos el subgrupo derivado
	\begin{gather*}
		G' := \gen{a^{-1}b^{-1}ab : a,b \in G}
	\end{gather*}
	Si $G$ no es abeliano, eso ser\'{a} algo m\'{a}s que el $\{e\}$. Resulta que $G' \normleq G$, y $\faktor{G}{G'}$ es conmutativo, ya que
	\begin{gather*}
		\overline{a^{-1}b^{-1}ab} = \overline{e} \implies \overline{a} \; \overline{b} = \overline{b} \; \overline{a}
	\end{gather*}
	Veamos el ejemplo con un di\'{e}drico. \\

	Sea $D_n = \gen{R,H}$.
	\begin{gather*}
		R^{i}H^{-1}R^{-1}H = R^{2} \implies R^{2i} \in D'_n \therefore \gen{R^{2}} \subseteq D'_n
	\end{gather*}
	Si nos ponemos a echar cuentas.
	\begin{gather*}
		(R^{i}H)^{-1}(R^{j}H)^{-1}(R^{i}H)(R^{j}H) = R^{2i -2j} \in \gen{R^2} \\
		(R^{i})^{-1}(R^{j}H)^{-1}R^{i}(R^{j}H) = R^{-2i} \in \gen{R^2} \\
		(R^{i}H)^{-1}(R^{j})^{-1}(R^{i}H)R^{j} = ... \in \gen{R^2} \\
		(R^{i})^{-1}(R^{j})^{-1}R^{i}R^{j} = e \in \gen{R^2}
	\end{gather*}
	Luego $D'_n = \gen{R^2}$. \\
	
	\ej{18H10}{¿Puede existir un epimorfismo de $\mathbb{Z}_4 \oplus \mathbb{Z}_4$ sobre $\mathbb{Z}_8$? ¿Puede existir un epimorfismo de $\mathbb{Z}_{16}$
		sobre $\mathbb{Z}_2 \oplus \mathbb{Z}_2$? ¿Puede existir un epimorfismo de $\mathbb{Z}_{18}$ sobre $\mathbb{Z}_3 \oplus
		\mathbb{Z}_2$? Explica tus respuestas.}

	Para cualquier homomorfismo, y elemento de un grupo:
	\begin{gather*}
		f: G \to H, \quad a \in G \\
		a^{|a|} = e \implies f(a^{|a|}) = e \implies f(a)^{|a|} = e \implies |f(a)|\; |\; |a|
	\end{gather*}
	Por esto, sabemos que no existe homomorfismo entre $\mathbb{Z}_4 \times \mathbb{Z}_4$ y $\mathbb{Z}_8$, ya que $\forall a \in \mathbb{Z}_4 \times
	\mathbb{Z}_4$, no hay elementos de orden 8 en un homomorfismo, luego $\mathbb{Z}_8$ no puede ser la imagen de un homomorfismo que salga de
	$\mathbb{Z}_4 \times \mathbb{Z}_4$. Tampoco puede existir un epimorfismo de $\mathbb{Z}_{16} \to \mathbb{Z}_2 \times \mathbb{Z}_2$, ya que
	$\mathbb{Z}_{16}$ es c\'{i}clico, y por ello, tambi\'{e}n lo ser\'{a} $\faktor{\mathbb{Z}_{16}}{\ker f}$, al ser subgrupo de $\mathbb{Z}_{16}$, pero
	la imagen no es c\'{i}clico. Eso viola el primer teorema de isomorfia. En el caso de $\mathbb{Z}_{18} \to \mathbb{Z}_3 \times \mathbb{Z}_2$.
	$\mathbb{Z}_{18}$ tiene exactamente un subgrupo normal de orden 3, que es $\gen{\overline{6}}$. Por ello, podemos ver que
	\begin{align*}
		\mathbb{Z}_{18} &\to \faktor{\mathbb{Z}_{18}}{\gen{\overline{6}}} \cong \mathbb{Z}_6 \cong \mathbb{Z}_3 \times \mathbb{Z}_2 \\
		\overline{x} &\mapsto [\overline{x}]
	\end{align*}
	Luego, te\'{o}ricamente, existe. Se puede construir con cuidado, al escoger el epimorfismo del primer paso. La composici\'{o}n total puede ser
	\begin{align*}
		f: \mathbb{Z}_{18} & \to \mathbb{Z}_3 \times \mathbb{Z}_2 \\
		\overline{x} & \mapsto (\overline{x},\overline{x})
	\end{align*}

	En general, existe epimorfismo entre $\mathbb{Z}_n \to \mathbb{Z}_{m_1} \times \mathbb{Z}_{m_2}$ si $m_1m_2 | n$, y $m_1,m_2$ con coprimos.
	Si existe, entonces
	\begin{gather*}
		\mathbb{Z}_{m_1} \times \mathbb{Z}_{m_2} = \phi(\mathbb{Z}_n) \cong \faktor{\mathbb{Z}_n}{\ker \phi} \therefore \\
		m_1m_2 = |\mathbb{Z}_{m_1} \times \mathbb{Z}_{m_2}| = \frac{|\mathbb{Z}_n|}{|\ker \phi|} = \frac{n}{|\ker \phi|} \therefore \\
		n = m_1m_2|\ker \phi|	
	\end{gather*}
	Si $n$ es m\'{u}ltiplo de $m_1m_2$, entonces existe un \'{u}nico $N \normleq \mathbb{Z}_n$ tal que
	\begin{gather*}
		|\faktor{\mathbb{Z}_n}{N}| = \frac{\;n\;}{\frac{n}{m_1m_2}} = m_1m_2 \therefore \faktor{\mathbb{Z}_n}{N} \cong \mathbb{Z}_{m_1m_2}
		\cong \mathbb{Z}_{m_1} \times \mathbb{Z}_{m_2}
	\end{gather*}
	
\end{document}
